\documentclass{sig-alternate}

\usepackage[usenames, dvipsnames]{color}
\usepackage{times}
\usepackage{xspace}
\usepackage{textcomp}
\usepackage{wrapfig}
\usepackage{graphicx}
\usepackage{url}
\usepackage{amsmath, amssymb}
\usepackage[protrusion=true,expansion=true]{microtype}
\usepackage{comment}
\usepackage{alltt}
\usepackage{appendix}
%\usepackage{algorithm}
%\usepackage{algorithmic}
\usepackage{booktabs}
\usepackage{color}
\usepackage{listings}
\lstset{ %
basicstyle=\ttfamily\scriptsize,       % the size of the fonts that are used for the code
numbers=left,                   % where to put the line-numbers
numberstyle=\ttfamily,      % the size of the fonts that are used for the line-numbers
%aboveskip=0pt,
%belowskip=0pt,
stepnumber=1,                   % the step between two line-numbers. If it is 1 each line will be numbered
%numbersep=10pt,                  % how far the line-numbers are from the code
breakindent=0pt,
firstnumber=0,
%backgroundcolor=\color{white},  % choose the background color. You must add \usepackage{color}
showspaces=false,               % show spaces adding particular underscores
showstringspaces=false,         % underline spaces within strings
showtabs=false,                 % show tabs within strings adding particular underscores
frame=leftline,
tabsize=2,  		% sets default tabsize to 2 spaces
captionpos=b,   		% sets the caption-position to bottom
breaklines=false,    	% sets automatic line breaking
breakatwhitespace=true,    % sets if automatic breaks should only happen at whitespace
%escapeinside={\%}{)}          % if you want to add a comment within your code
columns=fullflexible,
% are you fucking kidding me lstlistings?  who puts the line numbers outside the margin?
xleftmargin=6mm,
xrightmargin=-6mm,
numberblanklines=false,
language=Ruby,
morekeywords={declare,table,scratch,channel,interface,include,periodic}
}
\lstset{escapeinside={(*}{*)}}
%\renewcommand*\thelstnumber{\the\value{lstnumber}:}


%\linespread{0.975}

\usepackage{txfonts}

\newcommand{\jmh}[1]{{\textcolor{red}{#1 -- jmh}}}
 \newcommand{\paa}[1]{{\textcolor{blue}{#1 -- paa}}}
% \newcommand{\rcs}[1]{{\textcolor{green}{#1 -- rcs}}}
% \newcommand{\nrc}[1]{{\textcolor{magenta}{#1 -- nrc}}}
% \newcommand{\wrm}[1]{{\color{BurntOrange}{#1 -- wrm}}}
% \newcommand{\kc}[1]{{\textcolor{cyan}{#1 -- kc}}}
\newcommand{\smallurl}[1]{{\small \url{#1}}}

\CopyrightYear{2011}
\conferenceinfo{CIDR}{'11 Asilomar, California USA}

\begin{document}
\title{Consistency Analysis in Bloom: a CALM and Collected Approach}

\numberofauthors{4}
\author{
Peter Alvaro, Neil Conway, Joseph M. Hellerstein, William R. Marczak \vspace{12pt} \\
\{palvaro, nrc, hellerstein, wrm\}@cs.berkeley.edu \\
University of California, Berkeley
}

\maketitle

\begin{abstract}
  Distributed programming has become a topic of widespread interest, and many
  programmers now wrestle with tradeoffs between data consistency, availability
  and latency.  Distributed transactions are often rejected as an undesirable
  tradeoff today, but in the absence of transactions there are few concrete
  principles or tools to help programmers design and verify the correctness of
  their applications.

  We address this situation with the \emph{CALM} principle, which connects the
  idea of distributed consistency to program tests for logical monotonicity.  We
  then introduce \emph{Bloom}, a distributed programming language that is
  amenable to high-level consistency analysis and encourages order-insensitive
  programming.  We present a prototype implementation of Bloom as a
  domain-specific language in Ruby. We also propose a static analysis technique
  that identifies {\em points of order} in Bloom programs: code locations where
  programmers need to inject coordination logic to ensure consistency.  We
  illustrate these ideas in the context of two variants of a distributed
  ``shopping cart'' application in Bloom.  We also sketch the feasibility of
  code rewrites to support runtime annotation of data consistency, a topic for a
  longer paper.
\end{abstract}

\section{Introduction}
%Our research is motivated by two hard problems in distributed systems.  First,
\wrm{show examples of the problems (not necessarily code) -- evolving state and unreliable communication}

%Distributing any system introduces nondeterminism.  For example, one may
%distribute a computation over many inexpensive, but unreliable, commodity
%machines (e.g. RAID).  The status of internet links and widely distributed
%nodes is inherently more unreliable than multiple cores on a single die, or
%multiple CPUs in a single computer.  

%We present {\bf \lang}, a foundation language for programming and
%reasoning about distributed systems.  

%We correct deficiencies in earlier attempts, and introduce a compelling notion
%of non-determinism in the language.  We specifically use non-determinism to
%reason about {\em when} a deduction becomes visible, including the possibility
%that the deduction will never be visible.  Programmers can constrain this
%non-determinism by using well-studied techniques in distributed systems, such as
%Lamport Clocks 


Traditional database systems are based on declarative query languages that
specify transformations as dataflows over an updatable store.  Such query
languages are either not expressive enough to capture common programming
constructs \wrm{like what?}, or are at best awkward to use in this fashion.
\wrm{todo: transition that explains Datalog's birth from these languages... I
don't know enough to write it} The family of logic-based database languages, of
which Datalog is the progenitor, represent expressive programming languages
that produce similar dataflow representations.  Datalog is purely deductive: a
program specifies the rules by which the derived relations are populated based
on a static database, which is never updated.  Recent programming language
research has explored the use of Datalog-based languages for expressing
distributed systems.  Because the state of any complex system evolves with its
execution, these efforts were forced to extend the Datalog model by admitting
updates, additions and deletions of the EDB.  Unfortunately, these previous
attempts were plagued with ambiguities about how and when state changes occur
and become visible, putting a heavy burden on the programmer to ensure even
simple properties, such as atomicity of updates over time.

In contrast to reasoning about state change procdurally, \lang observes
that this concept is intuitively expressed as invariants over {\em time}.  In
this work, we present a formal model of Datalog augmented with time extensions.
By reifying time as data an introducing it into the logic, \lang eliminates
previous ambiguities, ensures atomicity of updates and makes it possible to
express system invariants that can guarantee liveness properties, a key
challenge in building distributed systems.

\section{Consistency and Logical \\ Monotonicity (CALM)}
In this section we present a strong connection between distributed consistency and logical monotonicity.  This discussion informs the language and analysis tools we develop in subsequent sections.

A key problem in distributed programming is reasoning about consistency in the face of {\em
temporal nondeterminism}: the delay and re-ordering of messages and data across
nodes.  Because delays can be unbounded, analysis typically focuses on ``eventual consistency'' after all messages have been delivered~\cite{vogels}.  A sufficient condition for eventual consistency is {\em order independence}: the independence of program execution from temporal
nondeterminism.

% \jmh{We need to pivot to a discussion of monotonicity and logic programming.  Trick: motivate declarative languages via disorderliness?  May be better to be more direct.}
% Programming models based on sets are attractive in this regard, because they keep programmers from making assumptions about the order of data arrival.  \nrc{Next sentence is confusing/vague to me.} But unordered inputs alone are not sufficient: there is still a question of handling delays as set contents stream (or stall) across a network. 
Order independence is a key attribute of declarative languages based on sets, which has led most notably to the success of parallel databases.  But even set-oriented languages can require a degree of ordering in their execution if they are sufficiently expressive.
The theory of relational databases and logic programming provides a framework to reason about these issues. \emph{Monotonic} programs---e.g., programs expressible via selection, projection and join---can be implemented by streaming algorithms that incrementally produce output elements as they receive input elements; the final order or contents of the input will never cause any earlier output to be ``revoked'' once it has been generated.\footnote{Formally, in a monotonic logic program any true statement continues to be true as new axioms---including new facts---are added to the program.}   
\emph{Non-monotonic} programs---e.g., those that contain aggregation or anti-join operations---can only be implemented correctly via blocking algorithms that do not produce any output until they have received all tuples in logical partitions of an input set. 
%%\nrc{The meaning of ``complete logical subset'' in the preceding sentence is unclear.} \jmh{switched to ``logical set partitions'', which is at least technically well-defined, though perhaps makes for vague english.  The example right after hopefully clears this up, no?}  
For example, aggregation queries need to receive entire ``groups'' before producing aggregates, which in general requires receiving the entire input set.

The implications for distributed programming and delayed messaging are clear. Monotonic programs are easy to distribute: they can be implemented via streaming set-based algorithms that produce actionable outputs to consumers while tolerating message reordering and delay from producers.  By contrast, even simple non-monotonic tasks like counting are difficult in distributed systems.  As a mnemonic, we say that \emph{counting requires waiting} in a distributed system: in general, a complete count of distributed data must wait for all its inputs, including stragglers, before producing the correct output.
% ; the alternative is to risk generating outputs that may be prove to be inconsistent with the full input once it arrives.    For example, an antijoin may output a row incorrectly if it does not wait for all the input records; similarly a SUM may be incorrectly reported as below a threshhold if only a subset of the records are summed.  

``Waiting'' is specified in a program via \emph{coordination logic}: code that (a) computes and transmits auxiliary information from producers to enable the recipient to determine when a set has completely arrived across the network, and (b) postpones production of results for consumers until after that determination is made.  Typical coordination mechanisms include sequence numbers, counters, and consensus protocols like Paxos or Two-Phase Commit.
% Our language, \emph{Bud}, omits coordination logic
% from non-monotonic operators by default, and forces the user to specify it.
% This is because badly-designed coordination logic can be a performance penalty,
% and this seems like a hard problem for an optimizer to approximate (an exact
% solution is undecidable).  In any case, this is the hard component of
% distributed systems programming, so a language should focus programmer effort
% on it.  
%Monotonic operators require no coordination logic.

Interestingly, these coordination mechanisms themselves typically involve counting.  For example, Paxos requires message counting to establish that a majority of the members have agreed to a proposal; Two-Phase Commit requires message counting to establish  that all members have agreed.  Hence we also say that \emph{waiting requires counting}, the converse of our earlier mnemonic.  

Our observations about waiting and counting illustrate the crux of what we call the \emph{CALM} principle: the tight relationship between Consistency And Logical Monotonicity.  Monotonic programs \emph{guarantee} eventual consistency under any interleaving of delivery and computation.  By contrast, non-monotonicity---the property that adding an element to an input set may revoke a previously-valid element of an output set---requires coordination schemes that ``wait'' until inputs can be guaranteed to be complete.  

Obviously we wish to minimize the use of coordination, because of well-known
concerns about latency and availability in the face of message delays during
coordination.  We can use the CALM principle to develop checks for distributed
consistency in logic languages, where conservative tests for monotonicity are
well understood. A simple syntactic check is often sufficient: if the program
does not contain any of the symbols in the language that correspond to
non-monotonic operators (e.g., \texttt{NOT IN} or aggregate symbols), then it is
monotonic and can be implemented without coordination, regardless of any
read-write dependencies in the code.  These conservative checks can be refined
further to consider semantics of predicates in the language. For example, the
expression ``\texttt{MIN(x) $< 100$}'' is monotonic despite containing an aggregate, by
virtue of the semantics of \texttt{MIN} and $<$: once a subset $S$ satisfies this
test, any superset of $S$ will also satisfy it.  Further refinements along
these lines exist, increasing the ability of program analyses to verify
monotonicity.
%More advanced whole-program analyses can take into account whether possibly
%non-monotonic conclusions can ever be consistent with other checks in the
%program; while undecidable in general, conservative tests with constraint
%propagation are often possible.  
% 
% It may also be useful to understand which portions of the program's execution
% or output may be affected by network non-determinism, and which are
% deterministic.  In this case, a static or runtime analysis could perform
% \emph{taint tracking} of network nondeterminism.  %Intuitively, a tuple is
% tainted if it is the transitive result of some %non-monotonic operator.

In cases where an analysis cannot guarantee monotonicity of a whole program, it can instead provide a conservative assessment of the points in the program where coordination may be required to ensure consistency.  For example, a shallow syntactic analysis could flag all non-monotonic predicates in a program (e.g., \texttt{NOT IN} tests or predicates with aggregate values as input).
We call the loci suggested by this analysis the program's \emph{points of
order}. A program with non-monotonicity can be made consistent by including coordination logic at its points of order.  

The reader may observe that because ``waiting requires counting,'' adding coordination logic actually increases the number of points of order in a program since it is non-monotonic.  To avoid this problem, the coordination logic must be hand-verified for consistency, after which annotations on the coordination logic can inform the analysis tool to (a) skip the coordination logic in its analysis, and (b) skip the point of order that the coordination logic handles.  

Because analyses based on the CALM principle operate with information about program semantics, they can avoid coordination logic in cases where traditional read/write analysis would require it.  Perhaps more importantly, as we will see in the next sections, logic languages and the analysis of points of order can help programmers redesign code to achieve goals of minimizing coordination.

%%\jmh{Should we say this: in the full paper, we will more directly show how CALM analysis relates to read/write analyses like serializability?}

% 
% In the general case, it is undecidable to verify whether they have done this
% correctly.  In fact, the introduction of coordination logic may involve
% additional non-monotonic operators (for example, two-phase commit has a
% non-monotonic "COUNT users = COUNT messages" statement).  There are a variety
% of possible solutions to this problem.  One option is for expert programmers
% (i.e. us) to encapsulate a suite of coordination protocols in a library, and
% manually verify the correctness of each protocol.  Another option is to develop
% various static analyses in the style of stratification tests for Datalog.
% \wrm{not sure how we can introduce additional detail here without pulling in
% tons of knowledge dependencies} 

% \wrm{addressed this} \jmh{At minimum, we should assert formal proof in one
% direction: purely Monotonic deduction can be done without coordination.  Start
% with explaining syntactic montonicity, perhaps via SQL, Datalog and MapReduce.
% Then enrich this by saying that single-node non-monotonicity is OK (if it's
% acyclic).} \wrm{i don't think it's okay if it's acyclic -- it's only okay if it
% does not depend on any network messages}  \jmh{  We should then enrich further
% by introducing the non-syntactic, ``instance-oriented'''' style monotonicity:
% saying that individual monotonic deductions---i.e. any monotonic tuple
% lineage---can be computed without coordination.  Give intuitive examples that
% are database instance-dependent (a la local stratification), and that are
% program-semantics dependent (a la universal constraint stratification).  The
% latter should do escrow transactions, and/or tee up shopping cart.}

% \wrm{addressed this} \jmh{Now you can talk at least in casual terms about a
% check for consistency: check the program for non-monotonicity.  Can probably do
% intuition of the data dependency here, and hint at how we'll do taint tracking
% later.}

% \wrm{not sure if this is true, also not sure what "universal-constraint-style
% analyses" you are referring to} \jmh{Now that we've said that ``counting
% requires waiting'', point out that ``waiting requires counting'': coordination
% protocols are basically threshhold tests on distributed count aggregates.
% Hence any check for consistency like the one above will flag the coordination
% logic as a problem.  Our solution: either expert programmers (us) verify the
% modules, or we develop universal-constraint-style analyses to validate them.}

% \wrm{i think we decided the other direction of CALM was false.  consider
% committed choice.  that's non-monotonic any way you look at it (except for the
% definition I tried to sell you durig the POPL submission that we ultimately
% rejected, because we believed monotonic should mean "has a monotonic
% representation in FOL")} \jmh{Finally, introduce the Bi-directional CALM
% Conjecture: that non-monotonic deduction (at the individual tuple level)
% requires coordination.  We can wave hands here about this requiring a crisper
% definition of coordination than we needed before.}

% \wrm{not sure what this means} \jmh{Occurs to me there's a disconnect here with
% the intro---in the intro we said that programming without coordination relates
% to ``loose'' consistency.  We should show a case where for the
% universal-constraint scenario, a read-write consistency analysis would flag
% this as bad, and a transaction manager would forbid it.}

\section{Adding Lattices to Bloom}
\label{sec:lang}

In this section, we introduce \lang, an extension of Bloom that allows monotonic
programs to be written using arbitrary lattices. We begin by reviewing the
algebraic properties of lattices, monotone functions, and morphisms. We then
introduce the basic concepts of \lang and detail the builtin lattices provided
by the language. We also show how users can define their own lattice types using
a simple Ruby API.

% is this the right place for this?
When designing \lang, we decided to extend Bloom to include support for
lattices rather than building a new language from scratch. Hence, \lang is
backward compatible with Bloom and was implemented with relatively minor
changes to the Bud runtime. This design decision also required that we consider
rules written over lattices should interoperate with rules that use traditional
Bloom relations; we added several \lang features to ease this interoperability,
which we describe in Section~\ref{sec:bloom-interop}.

\subsection{Definitions}
\label{sec:lattice-defn}
A \emph{bounded join semilattice} is a triple $\langle S, \lor, \bot\rangle$,
where $S$ is a poset, $\lor$ is a binary operator (called ``join'' or ``least
upper bound''), and $\bot \in S$. $\lor$ is associative, commutative, and
idempotent. For all $x, y \in S$, $x \lor y = z$, where $x \leq_S z, y \leq_S
z$, and there is no $z' \in S$ such that $z' <_S z$ (where $<_S$ is the partial
order associated with the poset $S$). Note that although the underlying set only
has a partial order, the least upper bound is defined for all elements $x,y \in
S$. The distinguished element $\bot$ is the smallest element in $S$; this
implies that $x \lor \bot = x$ for all $x \in S$. For brevity, we use the term
``lattice'' to mean ``bounded join semilattice'' in the rest of this paper.

% XXX: note that algebraic properties that must be satisfied by morphisms and
% monotone functions
A \emph{monotone function} from poset $S$ to poset $T$ is a function $g: S \to
T$ such that $\forall a,b \in S: a \leq_S b \Rightarrow g(a) \leq_T g(b)$. That
is, $g$ maps elements of $S$ to elements of $T$ in a manner that is consistent
with the partial orders of both posets.

% XXX: mention that morphisms must be distributive with respect to the lub of
% their domain, whereas monotone functions don't need to be?
A \emph{morphism} from lattice $\langle X, \lor_X, \bot_X\rangle$ to lattice
$\langle Y, \lor_Y, \bot_Y\rangle$ is a function $f$ such that, $\forall a,b \in
X: f(a \lor_X b) = f(a) \lor_Y f(b)$. That is, $f$ allows elements of $X$ to be
converted into elements of $Y$ in a way that preserves the lattice properties.
Note that all morphisms are monotone functions but the converse is not true in
general.

\subsection{Language concepts}
In \lang, state is represented using lattices and computation is expressed as
functions over lattices. A lattice in \lang is analogous to a collection type in
Bloom, while a lattice element corresponds to a particular collection value. For
example, the \texttt{lset} lattice provides similar functionality to the
set-oriented collections provided by Bloom; an element of the \texttt{lset}
lattice is a particular set value. In the terminology of object-oriented
programming, a lattice is a class that obeys a certain interface and an element
of a lattice is an instance of that class.

As in Bloom, the lattices used by a \lang program are declared in a
\texttt{state} block. More precisely, the declarations in the state block
introduce identifiers that are associated with lattice elements; over time, the
binding between identifiers and lattice elements is updated to reflect state
changes in the program. For example, line~\ref{line:quorum-lset-decl} of
Figure~\ref{fig:lattice-quorum} declares an identifier \texttt{votes} that is
mapped to an element of the \texttt{lset} lattice. As more votes are received,
the lattice element associated with the \texttt{votes} identifier
changes---specifically, it moves upward in the \texttt{lset} lattice.

\subsubsection{Computation in \lang}
Statements take the same form in both Bloom and \lang. The identifier on the lhs
of a statement can refer to either a set-oriented collection or a lattice. The
expression on the rhs can contain both traditional relational operators (applied
to Bloom collections) and method invocations (applied to lattice elements).

Note that if the lhs of a statement refers to a lattice, the statement's
operator must be either \verb|<=| or \verb|<+|, denoting instaneous or deferred
deduction. \lang does not currently support a notion of ``deletion'' for
lattices. Lattices do not directly support asynchronous communication (via the
\verb|<~| operator), but lattice elements can be embedded into channels (as
described in Section~\ref{sec:bloom-interop}).

\subsection{Builtin lattice types}
\label{sec:lattice-builtins}

\begin{table*}[t]
\begin{tabular}{|l|l|l|l|l|}
\hline
\textbf{Name} & \textbf{Description} & \textbf{Merge} & \textbf{Morphisms} &
\textbf{Monotone functions}\\
\hline
\texttt{lbool} & Boolean lattice (false $\to$ true) & & \texttt{when\_true} & \\
\texttt{lmax} & Max over an ordered domain & &\texttt{gt},
\texttt{gt\_eq}, \texttt{+}, \texttt{-} & \\
\texttt{lmin} & Min over an ordered domain & &\texttt{lt}, \texttt{lt\_eq},
\texttt{+}, \texttt{-} & \\
\texttt{lset} & Set of values & & \texttt{intersect}, \texttt{product},
\texttt{project} & \texttt{size} \\
\texttt{lpset} & Set of non-negative numbers & &
\texttt{intersect}, \texttt{product}, \texttt{project}& \texttt{size}, \texttt{sum} \\
\texttt{lbag} & Multiset of values & & \texttt{intersect},
\texttt{project}, \texttt{mult(k)}, \texttt{+} & \texttt{size}\\
\texttt{lmap} & Map from key to lattice values & &
\texttt{intersect}, \texttt{project}, \texttt{at(k)}, \texttt{key?(k)} & \texttt{size}\\
\hline
\end{tabular}
\caption{The builtin lattices in \lang.}
\label{tbl:builtin-lattices}
\end{table*}

Table~\ref{tbl:builtin-lattices} lists the builtin lattices provided by
\lang. Many common distributed protocols can be expressed using these lattices
(e.g., the causal delivery protocol described in Section~\ref{sec:causal}).

Note that \emph{size} is a monotone function provided by several lattices, but
it is not a morphism. This is because \texttt{size} cannot be distributed over
the merge functions of those lattices. For example, \ldots

\subsection{Lattice API}
\label{sec:lattice-api}
In \lang, each lattice has an associated Ruby class, which we call the
\emph{lattice class}. An instance of this class is called a \emph{lattice
  element}. A lattice element represents a single point in the lattice---i.e., an
element of the poset associated with the lattice.

A lattice class is a normal Ruby class that meets a certain API contract. Every
lattice class inherit from the builtin \texttt{Bud::Lattice} class, and
must also define two methods:
\begin{itemize}
\item \texttt{initialize(i)}: given a Ruby object \emph{i}, this method
  constructs a new lattice element that ``wraps'' \emph{i} (\texttt{initialize}
  is just the normal Ruby syntax for defining a constructor). By convention, the
  Ruby value wrapped by a lattice element is assigned to a Ruby member variable
  \texttt{@v}. If $i$ is the null reference, this method returns the least
  element of the lattice.

\item \texttt{merge(e)}: given a lattice element \emph{e}, this method returns the
  lattice element that is the least upper bound of $\{e, \textit{self}\}$. This method must
  satisfy the algebraic properties summarized in Section~\ref{sec:lattice-defn}---in
  particular, it must be commutative, associative, and idempotent. Note that
  \emph{e} must have the same class as \emph{self}.
\end{itemize}
Lattice elements are \emph{immutable} (e.g., \texttt{merge} functions should
construct a new lattice element rather than modifying one of their inputs
in-place). Efficient lattice implementations may \emph{share structure} on merge
operations, as is common practice for immutable data structures in functional
programming languages~\cite{Okasaki1999}. % XXX: maybe not the right place for this

\begin{figure}[t]
\begin{scriptsize}
\begin{lstlisting}
class Bud::SetLattice < Bud::Lattice
  wrapper_name :lset

  def initialize(x=[])
    # Reject invalid input (elided)
    @v = x.uniq # Remove duplicates from input
  end

  def merge(i)
    self.class.new(@v | i.reveal)
  end

  morph :intersect do |i|
    self.class.new(@v & i.reveal)
  end

  morph :pro do |&blk|
    @v.map(&blk)
  end

  ord_map :size do
    Bud::MaxLattice.new(@v.size)
  end
end
\end{lstlisting}
\end{scriptsize}
\caption{The implementation of the \texttt{lset} lattice in Ruby.}
\label{fig:lattice-set}
\end{figure}

\subsection{Integration with set-oriented logic}
\label{sec:bloom-interop}

\begin{figure}[t]
\begin{scriptsize}
\begin{lstlisting}
class ShortestPaths
  include Bud

  state do
    table :link, [:from, :to, :c]
    table :path, [:from, :to, :next_hop] => [:c]
    table :min_cost, [:from, :to] => [:c]
  end

  bloom do
    path <= link {|l| [l.from, l.to, l.to, MinLattice.new(l.c)]}
    path <= (link*path).pairs(:to => :from) do |l,p|
      [l.from, p.to, l.to, p.c + l.c]
    end
    min_cost <= path {|p| [p.from, p.to, p.c]}
  end
end
\end{lstlisting}
\end{scriptsize}
\caption{A \lang program to compute the all-pairs shortest paths of a
  graph.}
\label{fig:lattice-spaths}
\end{figure}

\lang provides two features to ease integration of lattice-based code with
traditional Bloom programs that manipulate set-oriented collections.

\subsubsection{Implicit fold}
% XXX: this ignores the fact that Bloom collections consist of sets of tuples,
% whereas implicit fold works for sets of singleton values
% XXX: refer to shortest paths program as practical example
This feature enables set-oriented collections to be more easily used as input to
lattices. If a \lang statement has a set-oriented collection on the rhs and a lattice
on the lhs, the lattice merge function is used to ``fold over'' the elements of
the collection. That is, each element of the collection is converted to a
lattice element (via the appropriate lattice constructor); then the set of
lattice elements are merged together (via repeated application of the
\texttt{merge} method). In our experience, this is typically the behavior
intended by the user.

\subsubsection{Collections with embedded lattice values}
It would be convenient to allow lattice elements to be stored as attributes of
tuples that appear in set-oriented Bloom collections. Furthermore, Bloom
provides several facilities (e.g., network communication, persistent storage,
module interfaces) as collections with special semantics; it would be
unfortunate if a redundant set of facilities would be necessary to support
lattice-based code. A simple solution would be to extract the underlying Ruby
value from the lattice element (e.g., using the \texttt{reveal} method), and
then store that value as a tuple attribute in a set-oriented
collection. Unfortunately, that would introduce needless non-monotonicity into
the program.

Storing lattice elements as attributes of tuples in set-oriented collections
introduces several challenges. Consider a simple \lang statement that derives tuples
with a lattice element as an attribute value:
\begin{verbatim}
    t1 <= t2 {|t| [t.x, lat_foo]}
\end{verbatim}
where \texttt{t1} and \texttt{t2} are Bloom relations and \texttt{lat\_foo}
identifies a lattice; suppose that the first column of \texttt{t1} is the
relation's key. The value associated with \texttt{lat\_foo} can change over the
course of the fixpoint computation (specifically, it can grow ``upward''
according to the lattice's partial order, as more values are merged into the
lattice). Implemented naively, this might result in multiple \texttt{t1} tuples
with different values for the second attribute, which would violate
\texttt{t1}'s key (the first column of \texttt{t1} would not functionally
determine a single value for the second column).

This problem could be avoided by placing constraints on the evaluation order of
statements: for example, we could require that all potential changes to
\texttt{lat\_foo} be completed before rules that embed \texttt{lat\_foo} could
be evaluated. This would effectively stratify the program according to lattice
embedding rules, which would disallow cycles through lattice
embeddings~\cite{Apt1988}. This would reject intuitively reasonable programs; it
also seems unsatisfying to require stratification of monotonic programs.

% Clarify this
Instead, \lang allows rules to produce multiple tuples that differ only in their
embedded lattice values. During the course of the fixpoint computation, those
values are merged together using the appropriate lattice merge function. This is
safe because Bud stratifies programs according to non-monotonic operators;
hence, any operators that might be applied to an embedded lattice value before
it has been determined exactly must be monotonic. Nevertheless, this solution is
somewhat counterintuitive because tuples in Datalog relations are traditionally
immutable: once a fact is known to be true, its value remains the
same.% \footnote{Bloom facts can be deleted, but this is an explicit non-monotonic
  % operation that can only occur between timesteps. Conceptually, Bloom models
  % update as the retraction of the previous version of a fact and the insertion
  % of a new fact~\cite{dedalus}.}
% Should we note that we might add an option to disable this behavior for
% particular attributes, or explain more about why this might be considered
% weird?

For similar reasons, we currently disallow lattice values from being used as
keys in Bloom collections. It might be possible to relax this restriction in
certain ``safe'' cases, but we have not found this limitation to be problematic
to date.

\subsection{Confluence in \lang}
\nrc{TODO: justify that CALM continues to hold for monotonic programs over
  lattices.}

\section{Case Study: Key-Value Store}
\label{sec:kvs}
In this section, we present two variants of a key-value store (KVS) implemented
using Bloom. We begin with an abstract protocol that any key-value store will
satisfy, and then provide both single-node and replicated implementations of
this protocol. We then introduce a graphical visualization of the dataflow in a
Bloom program, and use this visualization to reason about the \emph{points of
  order} in our programs: places in the program where additional coordination
may be required to guarantee consistent results (Section~\ref{sec:calm}).

\subsection{Abstract Key-Value Store Protocol}

\begin{figure}[t]
\begin{scriptsize}
\begin{lstlisting}
module KVSProtocol
  def state
    interface input, :kvput, 
      ['client', 'key', 'reqid'], ['value']
    interface input, :kvget, ['reqid'], ['key']
    interface output, :kvget_response, 
      ['reqid'], ['key', 'value']
  end
end
\end{lstlisting}
\centering
\vspace{-10pt}
\caption{Abstract key-value store protocol.}
\label{fig:kvs-proto}
\end{scriptsize}
\vspace{-2pt}
\end{figure}

Figure~\ref{fig:kvs-proto} specifies a protocol for interacting with an abstract
key-value store. The protocol comprises two input interfaces (representing
attempts to insert and fetch items from the store) and a single output interface
(which represents the outcome of a fetch operation). To use an implementation of
this protocol, a Bloom program can store key-value pairs by inserting facts into
\texttt{kvput}. To retrieve the value associated with a key, the client program
inserts a fact into \texttt{kvget} and then looks for the corresponding response
tuple in \texttt{kvget\_response}. In both cases, the client program must supply
a unique request identifier (\texttt{reqid}) to differentiate tuples in the
event of multiple concurrent requests.

A module which uses a key-value store but is indifferent to the specifics of the
implementation may simply mix in this protocol specification and postpone
committing to a particular implementation until runtime. As we will see shortly,
an implementation of the KVSProtocol is a collection of Bud rules that read
tuples from the protocol's input interfaces and send results to the output
interface.

\subsection{Single-Node Key-Value Store}
\label{sec:simple-kvs}
\begin{figure}[t]
\begin{scriptsize}
\begin{lstlisting}
module BasicKVS
  include KVSProtocol

  def state
    table :kvstate, ['key'], ['value'] (*\label{line:kvs-state}*)
  end

  declare
  def do_put
    kvstate <+ kvput.map{|p| [p.key, p.value]} (*\label{line:kvs-put}*)
    jst = join [kvstate, kvput], [kvstate.key, kvput.key] (*\label{line:kvs-join}*)
    kvstate <- jst.map{|b, p| b} (*\label{line:kvs-clean}*)
  end

  declare
  def do_get
    getj = join [kvget, kvstate], [kvget.key, kvstate.key] (*\label{line:kvs-getjoin}*)
    kvget_response <= getj.map do |g, t|
      [g.reqid, t.key, t.value]
    end
  end
end
\end{lstlisting}
\centering
%%\includegraphics[width=0.55\linewidth]{fig/basickvs.pdf}
\vspace{-10pt}
\caption{Single-node key-value store implementation.}
\label{fig:kvs-impl}
\end{scriptsize}
\vspace{-2pt}
\end{figure}

Figure~\ref{fig:kvs-impl} contains a simple single-node implementation of the
abstract key-value store protocol. Key-value pairs are stored in a persistent
table called \texttt{kvstate} (line~\ref{line:kvs-state}). When a \texttt{kvput}
tuple appears, its key-value pair is stored in \texttt{kvstate} at the immediately
following timestep (line~\ref{line:kvs-put}).  If the given key already exists in
\texttt{kvstate}, we want to replace the key's old value. This is done by
joining \texttt{kvput} against the current version of \texttt{kvstate}
(line~\ref{line:kvs-join}). If a matching tuple is found, the old key-value pair
is removed from \texttt{kvstate} at the beginning of the next timestep
(line~\ref{line:kvs-clean}). Note that we also insert the new key-value pair
into \texttt{kvstate} in the next timestep (line~\ref{line:kvs-put});
hence, we implement an overwriting update as an atomic deletion and insertion.

\subsection{Replicated Key-Value Store}
\label{sec:rep-kvs}

\begin{figure}[t]
\begin{scriptsize}
\begin{lstlisting}
module ReplicatedKVS
  include BasicKVS
  include MulticastProtocol

  def state
    interface input, :kvput, (*\label{line:rep-put-beg}*)
      ['client', 'key', 'reqid'], ['value']  (*\label{line:rep-put-end}*)
  end

  declare
  def replicate
    send_mcast <= kvput.map do |k| (*\label{line:send-mcast-beg}*)
      unless members.include? [k.client]  (*\label{line:not-rep}*)
        [k.reqid, [@addy, k.key, k.reqid, k.value]]   (*\label{line:marshall}*)            
      end
    end (*\label{line:send-mcast-end}*)
  end

  declare
  def apply_put
    kvput <= mcast_done.map{|m| m.payload}  (*\label{line:mcast-done}*)

    kvput <= pipe_chan.map do |d| (*\label{line:mcast-peer-beg}*)
      if d.payload.fetch(1) != @addy
        d.payload
      end
    end (*\label{line:mcast-peer-end}*)
  end
end
\end{lstlisting}
\vspace{-10pt}
\caption{Replicated key-value store implementation.}
\label{fig:kvs-repl}
\end{scriptsize}
\vspace{-2pt}
\end{figure}

Next, we extend the basic key-value store implementation to support replication
(Figure~\ref{fig:kvs-repl}). To communicate between replicas, we use a simple
multicast library implemented in Bloom; the complete source code for this
library can be found in Appendix~\ref{app:network-code}. To send a multicast, a
Bloom program inserts a fact into \texttt{send\_mcast}; a corresponding fact
appears in \texttt{mcast\_done} when the multicast has completed.\footnote{XXX:
  error handling.} The multicast library also provides the membership of the
multicast group in a table called \texttt{members}.

Our replicated key-value store is implemented on top of the single-node
key-value store described in the previous section. When a new key is inserted by
a client, we multicast the insertion to the other replicas
(lines~\ref{line:send-mcast-beg}--\ref{line:send-mcast-end}). To avoid repeated
multicasts of the same inserted key, we avoid multicasting updates we receive
from another replica (line~\ref{line:not-rep}). We apply an update to our local
\texttt{kvstate} table in two cases: (1) if a multicast succeeds at the node
that originated it (line~\ref{line:mcast-done}) (2) whenever a multicast is
received by a peer replica
(lines~\ref{line:mcast-peer-beg}--\ref{line:mcast-peer-end}).

Note that the implementation of ReplicatedKVS wants to ``intercept''
\texttt{kvput} events from clients, and only apply them to the underlying
BasicKVS module when certain conditions are met. To achieve this, we
``override'' the declaration of the \texttt{kvput} input interface
(lines~\ref{line:rep-put-beg}--\ref{line:rep-put-end}). In ReplicatedKVS,
references to \texttt{kvput} appearing in the LHS of rules are resolved to the
\texttt{kvput} provided by BasicKVS, while references in the RHS of rules
resolve to the local \texttt{kvput}.  Note that this is unambiguous, because a
module cannot insert into its own input or read from its own output
interfaces. Nevertheless, the current syntax for achieving this is somewhat
cryptic; it might be improved by the addition of a namespace-like concept.

\subsection{Predicate Dependency Graphs}
\begin{figure}[t]
\centering
\includegraphics[width=0.9\linewidth]{fig/mittalk_legend.pdf}
\vspace{-10pt}
\caption{Visual analysis legend. \jmh{Check on the $+/-$ label in the figure.  Is that right, do we ever use $-$?}}
\label{fig:analysis-legend}
\vspace{-2pt}
\end{figure}

Now that we have introduced two concrete implementations of the abstract
key-value store protocol, we now turn to analyzing the properties of these
programs. First, we introduce the graphical dataflow representation used by our
analysis; we then discuss the dataflow graphs generated for the key-value store
programs in the following section.

A Bloom program may be viewed as a dataflow graph with external input interfaces
as sources, external output interfaces as sinks, collections as internal nodes,
and rules as edges. This graph represents the dependencies between the
collections in a program, and can be generated automatically by the Bud
interpreter. A list of the different symbols and annotations in the graphical
visualization can be found in Figure~\ref{fig:analysis-legend}; we provide a
brief summary below.

Each node in the graph is either a collection or a cluster of collections;
tables are shown as rectangles, ephemeral collections (scratch, periodic and
channel) are depicted as ovals, and clusters (described below) as octagons. A
directed edge from node $A$ to node $B$ indicates that $B$ appears in the lhs of
a Bloom rule with $A$ referenced directly, or through a join expression, in the
rhs.  An edge is annotated based on the operator symbol in the rule. If the rule
uses the \texttt{$<$+} or \texttt{$<$-} operators, the edge is marked with
$+$. This indicates that facts traversing the edge ``spend'' a timestep to move
from the rhs to the lhs. Similarly, if the rule uses the \texttt{$<\sim$}
operator, the edge is a dashed line---this indicates that facts from the rhs
appear at the lhs at a non-deterministic future time. If the rule involves a
non-monotonic operation (aggregation, anti-join, or the \texttt{$<$-} operator),
then the edge is marked with a small circle.  To make the visualizations more
readable, any strongly connected component marked with both a circle and a $+$
edge is collapsed into an octagonal ``temporal cluster,'' which can be viewed
abstractly as a single, non-monotonic node in the dataflow. \emph{Points of
  order} are indicated in the graph by an edge with a white circle.  Any
non-monotonic edge in the graph is a point of order, as are all edges incident
to a temporal cluster, including any self-edges.

\subsection{Analysis}
Figure~\ref{fig:pdg-kvs-proto-analysis} presents a visual representation of the
abstract key-value store protocol. The diagram depicts the semantics implied by
the protocol's interfaces: data is inserted into the input interfaces
(\texttt{kvput} and \texttt{kvget}), and results eventually appear in the output
interface (\texttt{kvget\_response}). Naturally, the abstract protocol does not
specify a connection between the input and output events; this is indicated in
the diagram by the red diamond, which denotes an underspecified program. A
concrete realization of the key-value store protocol must, at minimum, supply a
dataflow that connects an input interface to the output interface.

Figure~\ref{fig:pdg-kvs-analysis} shows the visual analysis of the basic KVS
implementation, which supplies a concrete dataflow for the unspecified component
in the previous graph.  \texttt{kvstate} and \texttt{jst} are collapsed into a
red octagon because they are part of a strongly connected component in the graph
with both negative and temporal edges.  Any data flowing from \texttt{kvput} to
the sink must cross at least one non-monotonic point of order (at ingress to the
octagon) and possible an arbitrary number of them (via the octagon's self-edge),
and any path from \texttt{kvget} to the sink must join state potentially
affected by non-monotonicity (because \texttt{kvstate} is used to derive
\texttt{kvget\_response}).

Reviewing the code in Figure~\ref{fig:kvs-impl}, we see that this is
unavoidable.  The contents of \texttt{kvstate} at a given time are defined (in
part; line~\ref{line:kvs-clean}) in terms of its contents at the immediate
previous state and the current input.  Hence the state of \texttt{kvstate} at
any point in time may depend on the order of arrival of \texttt{kvput} tuples.
% We challenge the reader to implement the KeyValueProto interface in a way that
% has no such point of order.



\begin{figure}[t]
\centering
\includegraphics[width=0.4\linewidth]{fig/kvs_proto_pdg.pdf}
\vspace{-10pt}
\caption{Visualization of the abstract key-value store protocol.}
\label{fig:pdg-kvs-proto-analysis}
\vspace{-2pt}
\end{figure}

\begin{figure}[t]
\centering
\includegraphics[width=0.5\linewidth]{fig/basickvs.pdf}
\vspace{-10pt}
\caption{Visualization of the single-node key-value store.}
\label{fig:pdg-kvs-analysis}
\vspace{-2pt}
\end{figure}

\section{Case Study}
\label{sec:case}

\wrm{Re-do case studies in Bud} \wrm{Break cart development down into
iterations} \wrm{How does the language naturally lead us to an order
independent style?  Talk about inserting all sorts of exotic stuff like queues
if we want a highly order-dependent imperative style.}

\section{Tolerating Inconsistency}
\label{sec:inconsistency}
The analysis technique described in the preceding sections provides a
conservative test for when temporal non-determinism may lead to
non-deterministic results. One solution to this situation is to resolve the
non-determinism by introducing coordination. However, in many cases adding
additional coordination is undesirable (e.g., for latency or availability
reasons). Even here, we think that Bloom can improve distributed programming by
assisting programmers with the task of \emph{tolerating inconsistency}, rather
than resolving it via coordination.  A notable example of how to manage
inconsistency is presented by Helland and Campbell, who reflect on their
experience programming with patterns of ``memories, guesses and
apologies''~\cite{quicksand}.  We provide a sketch here of ideas for converting
these patterns into developer tools.

%Declarative programs typically assume ``guaranteed'' base facts, and use rules to define ``guaranteed'' derived monotonic.  
``Guesses''---facts that may not be true---may arise at the inputs to a program,
e.g., from noisy sensors or untrusted software or users.  But Helland and
Campbell's use of the term corresponds in our analysis to unresolved points of
order: non-monotonic logic that makes decisions without full knowledge of its
input sets.  We can rewrite the schemas of Bloom collections to include an
additional attribute marking each fact as a ``guarantee'' or ``guess,'' and
automatically augment user code to propagate those labels through program logic
in the manner of ``taint checking'' in program security~\cite{taint,asbestos}.
Moreover, by identifying unresolved points of order, we can identify when
program logic derives ``guesses'' from ``guarantees,'' and rewrite user code to
label data appropriately. By rewriting programs to log guesses that cross
interface boundaries, we can also implement Helland and Campbell's idea of
``memories'': a log of guesses that were sent outside the system.

Most of these patterns can be implemented as automatic program rewrites. We
envision building a system that facilitates running low-latency,
``guess''-driven decision making in the foreground, and expensive but consistent
logic as a background process. When the background process detects an
inconsistency in the results produced by the foreground system (e.g., because a
``guess'' turns out to be mistaken), it can then take corrective action by
generating an ``apology.'' Importantly, both of these subsystems are
implementations of the same high-level design, except with different consistency
and coordination requirements; hence, it should be possible to synthesize both
variants of the program from the same source code. Throughout this
process---making calculated ``guesses,'' storing appropriate ``memories,'' and
generating appropriate ``apologies''---we see significant opportunities to build
scaffolding and tool support to lighten the burden on the programmer.

Finally, we hope to provide analysis techniques that can prove the consistency
of the high-level workflow: i.e., prove that any combination of user behavior,
background guess resolution, and apology logic will eventually lead to a
consistent resolution of the business rules at both the user and system sides.

\section{Related Work}

\subsection{Concurrency control}
The importance of commutative operations has long been recognized by work in the
distributed systems, data management, and groupware communities.

\subsection{Non-monotonicity in deductive databases}
Adding non-monotonic operators (e.g., aggregation and negation) to Datalog
increases the expressiveness of the language but introduces significant
complexities: care must be taken to ensure that the resulting language has a
semantics that is well-defined, intuitive to the user, and amenable to efficient
evaluation. A straightforward approach is to disallow recursion through
aggregation or negation, which admits only the class of so-called ``stratified
programs''~\cite{Apt1988}. Many attempts have been made to assign a semantics to
larger classes of programs (e.g.,~\cite{Gelfond1988,Ross1990,VanGelder1991}).

The observation that many uses of aggregation and negation have a ``monotonic''
flavor has been made before.

% CRDTs
% Work on semantics-aware concurrency control
% Operational transformations
% Ross & Sagiv on lattices
% Kostler et al. on differential fixpoint + subsumption

\section{Conclusion}
\label{sec:conclusion}
And thus, we conclude.
\section*{Acknowledgments}
We would like to thank Peter Bailis, Ali Ghodsi, David Maier, and Matei Zaharia
for their helpful feedback on this paper.  This work was supported by the
Natural Sciences and Engineering Research Council of Canada.


\bibliographystyle{abbrv}
\bibliography{cidr11,declarativity}

\appendix
\section{Proof of Lemma 1}
\begin{proof}
%First, we prove an isomorphism between stable models, and finite prefixes of stable models.  Scan a stable model of a program timestamp by timestamp.  
We first present an algorithm for computing ultimate models, and argue that the algorithm computes exactly the stable models of the \lang program.  We then argue this algorithm can be run on our operational formalism, show how operational traces correspond with prefixes of stable models.

Any \lang program without asynchronous rules is a $\text{Datalog}_1S$ program, and the algorithm given in~\cite{tdd} computes an ultimate model in polynomial space\footnote{The class of {\em multi-seperable}~\cite{tdd-poly} \lang rules, which comprises all \lang programs $P$ with guarded asynchrony and persisted EDB, and their coordinations $\textsc{Coord}(P)$, can be executed in polynomial time.} in the size of the input.  The algorithm evalutes the program for $2^G + e$ consecutive timesteps, where $G$ is the number of instantiations of the non-temporal attributes of the program rules, using all combinations of constants in the Herbrand Universe, and $e$ is the maximum timestamp of any EDB fact.  At each step, the algorithm updates information on observed periodicities of facts.  When the algorithm terminates, any fact with a periodicity of 1 is regarded as part of the ultimate model.

For asynchronous rules, the natural distributed analog of the algorithm above simultaneously executes one instance for each node \dedalus{n}, using values of $G$ and $e$ computed from $E_{\text{\dedalus{{\scriptsize n}}}}$.  Each instance has its own local clock, which intuitively corresponds to the timestamp attribute in the model-theoretic semantics.  Nodes communicates over channels with arbitrary delay and message re-ordering.  When another node \dedalus{m} derives a fact at \dedalus{n}, it encloses its local clock value, \dedalus{t}; \dedalus{n} must consider this fact until time \dedalus{t}, in the style of Lamport Clocks.  Note that this behavior is implied by the model-theoretic semantics---remote asynchronous rules state that their deductions are visible at the destination at a time later than the body temporal attribute at the source.  Further, note that Lamport Clocks only introduce the constraint that if message $a$ ``happens before'' $b$, in other words $a$ directly or transitively causes $b$ to be sent, then $T(a) < T(b)$.  If $a$ and $b$ are concurrent, there is some execution where $T(a) \geq T(b)$.

When \dedalus{n} processes a received message, the number of constants available to \dedalus{n} may increase, and thus a node's $G$ may increase to $G'$.  Furthermore, the node may need to execute over this new fact for $2^{G'}$ additional timesteps.  If only finitely many messages are sent, this algorithm requires polynomial space.  In the case that infinitely many messages are sent, we only need to process each message $2^{G'}$ times: the maximum period of any fact is $2^{G'}$, as every incoming fact needs to have a chance (in some execution) to join with any deduction, at any time during its period, with which it is ``concurrent''.  Keeping track of the number of times we have seen each fact also requires polynomial space.  When the algorithm is done running for $2^{G'}$ steps, it pauses, waiting for new network input that it has not seen enough times.  If all nodes are paused and no outstanding messages exist, then the collection of all period 1 facts at all instances of the algorithm comprises an ultimate model.

We claim that the algorithm can generate every ultimate model---every message has the opportunity to join with another concurrent message or its transitive consequents at any point during their period, and has the opportunity to join with a causally related message during the range of times allowed by the model-theoretic asynchronous constraint (identical to the Lamport Clock condition used in the algorithm).

Note that we can execute this algorithm on our operational formalism.  Evaluating a single timestamp of a \lang program corresponds to the evaluation of a Datalog program, which is a polynomial time computation, and the Turing Machines can also maintain the necessary state about periods and message counts.
%2) Intuitively, the operational model is based on n Turing Machines, one per value of node(), which independently step sequentially through time and communicate via channels with
%non-deterministic delay.  At each timestep t they run a datalog fixpoint computation that evaluates P on ``projection(E_n, t)'' (notation needed); this takes polynomial
%time~\cite{immerman}.  At the end of this fixpoint there are three sets of relevant facts: local, synchronous facts that have timestep t+1 and become part of ``projection(E_n,
%t)'', local asynchronous facts whose timestep is chosen non-deterministically to be greater than t and become part of later timesteps, and remote asynchronous facts.  The
%timestamps in this third class of facts are chosen non-deterministically ``at the receiver'' to model delay, in a way that observes traditional causality
%restrictions~\cite{lamportclocks}.
%3) Any \lang program without  async rules is a Datalog_{1S} program, and the above intuition is captured by the algorithm given in~\cite{}, computing an ultimate model in
%polynomial space in the size of the input.  In the presence of asynchronous rules, this formalize needs to be expanded to account for the asynchronous advancement of time through
%\dedalus{successor} at each node.  The PSPACE guarantees of~\cite{} are not shown to hold for such programs, but in Appendix Foo we show that the following Lemma holds for all
%\lang programs under this model
\end{proof}

\section{Proof of Lemma 2}
\begin{proof}
We begin by assuming that \dedalus{node} contains the identifiers of each of the $n$ nodes.  Since the atemporal fragment of \lang is FO[LFP], we can represent a polynomial-time bounded Turing Machine using only atemporal rules in \lang~\cite{immerman-ptime}.  In addition to normal operations, the Turing Machine can place items into a queue---\cite{dedalus} shows how to model queues in \lang---or send messages to other nodes---modeled by an asynchronous communication rule with \dedalus{queue} in the head.  A node persists the contents of the tape across time if the queue is empty, using a rule like \dedalus{tape(\dbar{X})@next <- tape(\dbar{X}), !queue(\dbar{\_});}.  If the queue is non-empty, the computation skips a timestamp (leaving \dedalus{tape} empty), and then atomically copies the contents of \dedalus{queue} to \dedalus{tape}.  The ultimate model of this program is exactly the final contents of the tape on every node if the computation halts.  Otherwise, the program's ultimate model is empty: \dedalus{tape} facts only exist every other timestamp, and for any Turing Machine predicate \dedalus{r} we can create \dedalus{r'}, and create a mutual recursive cycle to ensure neither \dedalus{r} nor \dedalus{r'} contains facts at every timestamp:

\begin{Dedalus}
r(\dbar{X})@next <- r'(\dbar{X});
r'(\dbar{X})@next <- r(\dbar{X});
\end{Dedalus}

We can play a somewhat similar trick for \dedalus{queue} by having local messages alternate between going into \dedalus{queue} and \dedalus{queue'}.  Thus, no local queue message will be part of the ultimate model.  Remote messages will still go into \dedalus{queue}: this still leaves the case that the exact same message repeatedly arrives at a node at every timestamp forever, by chance.  We can dispense of this case by assuming the channels interconnecting the Turing Machines forbid it.
\end{proof}

\section{Proof of Lemma 3}
\begin{proof}
Our proof proceeds via construction of a two counter machine in \lang, inspired by the construction in~\cite{undecidable-datalog}. We briefly review two counter machines.  A two counter machine's state is captured in the state of its two counters (natural numbers), and in its control state.  A two counter machine has a transition function: $\delta: \Sigma \times \{=, >\} \times \{=, >\} \rightarrow \Sigma \times \{inc, dec\} \times \{inc, dec\}$

$\Sigma$ is a finite set of states (for simplicity we assume a finite subset of the natural numbers), $=$ indicates a counter is equal to zero, and $>$ indicates a counter is greater than zero.  $inc$ and $dec$ indicate that a counter should be incremented, or decremented respectively.

We represent the state of a two counter machine using the \linebreak \dedalus{cnfg(T,S,C1,C2)} relation, where \dedalus{T} represents ``time'' (note this is not the same as the timestamp attribute), \dedalus{S} is the state (in $\Sigma$), and \dedalus{C1} and \dedalus{C2} are the values of the two counters.  In order to support $inc$ and $dec$, we would like to make use of the \dedalus{succ} relation.  However, \lang conventions forbid the use of this infinite relation outside of the timestamp attribute.  Thus, we posit the \dedalus{fin\_succ(X,Y)} EDB relation, which represents a finite prefix of the successor relation.  Since it is EDB, its contents may be arbitrary.  If \dedalus{fin\_succ} is malformed, then the machine's execution may be incorrect.  In particular, our model of the machine may accept an input, whereas the actual machine would not have accepted that input.  We illustrate how to constrain the contents of \dedalus{fin\_succ} below:

\begin{Dedalus}
malformed() <- fin_succ(_,0);
malformed() <- fin_succ(X,Y), fin_succ(X,Z), Y != Z;
malformed() <- fin_succ(Y,X), fin_succ(Z,X), Y != Z;
malformed() <- fin_succ(X,Y), X >= Y;
\end{Dedalus}

For a given EDB, the two counter machine either halts in the accepting state or halts in a non-accepting state.  It cannot run forever since the EDB (in particular, the \dedalus{fin\_succ} relation) is finite.

We construct a \lang program that nondeterministically decides to either run the machine on the input provided (and for the length of \dedalus{fin\_succ} provided, or declare that the machine will never accept without running it.  If the machine ever accepts some input, then we would like this to induce two different ultimate models -- one generated by a trace where we run the machine and it accepts, and one generated by a trace where we decide to not run the machine, and thus we implicitly reject.  We describe the program below. 

Initially, we nondeterministically decide whether to run the machine or not, by sending two messages (0 and 1) to a remote node (\dedalus{decider}).  If both message arrive simultaneously, then the decider responds to run the machine.  Otherwise, the decider responds to declare failure:

%\jmh{should we use a hashmark for constants?  I would say no.}
\begin{Dedalus}
//send two messages to the decider
message(#D, 0)@async <- decider(D);
message(#D, 1)@async <- decider(D);

//decider responds to computer
run_machine(#computer)@async <- message(0),
                                message(1);
declare_failure(#computer)@async <- message(0),
                                    !message(1);
declare_failure(#computer)@async <- !message(0),
                                    message(1);
\end{Dedalus}

Each mapping in the transition function is expressed by a \lang rule with \dedalus{!malformed()} and \dedalus{!declare\_failure()} in its body.  For example, the rule $\delta(3, > =) = (7, inc, dec)$ would be represented as:

\begin{Dedalus}
cnfg(S,7,D1,D2) <- cnfg(T,3,C1,C2), C1 > 0, C2 == 0,
                   fin_suc(T, S), fin_succ(C1, D1),
                   fin_succ(D2, C2), !malformed(),
                   !declare_failure();
\end{Dedalus}

We declare success or failure as follows:

\begin{Dedalus}
reject() <- !accept();
accept() <- cnfg(20,_,_); //20 is the accepting state
accept()@next <- accept();
\end{Dedalus}

If we choose to declare failure, or the machine halts in a non-accepting state, whether it is due to incompleteness or malformedness \dedalus{fin\_succ}, or actual halting, then the ultimate model will contain \dedalus{reject}.  If the machine halts in an accepting state, then the ultimate model will contain \dedalus{accept}.  Thus, if we can decide confluence of this program, then we can decide whether a two-counter machine halts on any input.
\end{proof}


\end{document}
