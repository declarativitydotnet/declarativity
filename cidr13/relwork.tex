\section{Related Work}
\label{sec:relwork}

\wrm{Neil's lattice stuff}
\wrm{Bloom}
Our
``algebra'' is similar to the Bloom language for distributed systems.  In
Bloom, programmers react to an adversarial network that may reorder messages.  In contrast, our language induces programmers to proactively insert disorder into their code.

In the Bloom work, we focused on writing distributed systems, which are inherently non-deterministic by virtue of temporal non-determinism induced by communication between nodes in the system.  When specifying a distributed system, a programmer would use a static analysis to ensure that his code would be deterministic, regardless of this unavoidable non-determinism.~\cite{cidr11}

In this work, we start with well-known deterministic algorithms.  Many deterministic algorithms leverage non-determinism for increased performance.  For example, when one designs an asynchronous parallel algorithm, one must reason about the effects of temporal non-determinism between tasks on the output of the program.  Similarly, when one designs a Las Vegas algorithm---a deterministic algorithm that employs randomness for a speedup---one must reason about the effects of different random choices.

\wrm{SN and PSN evaluation}
The idea of choice functions can be viewed as a generalization of semi-naive~\cite{seminaive} evaluation in Datalog, and pipelined semi-naive~\cite{loo-sigmod06} evaluation proposed in distributed Datalog systems such as Overlog.  Indeed, one may think of Datalog as a particular instance of our ``algebra'' language where all data structures are sets, and all rule functions are composed of only selection, projection, and join.  Furthermore, all input identifiers have fixed contents before the evaluation starts.

In the semi-naive algorithm, evaluation proceeds in {\em rounds}.  The semi-naive algorithm is designed to be applied in cases where the input identifiers (so-called {\em relations} in Datalog) are sealed (unchanging over time).  In each round, every rule is ``fired once'' on all available elements in its body streams.  New elements derived are set aside for processing in the next round.

One may envision a tree whose nodes are elements of streams in the execution of an ``algebra'' program.  A link exists from node $n_1$ to node $n_2$ if the stream element represented by $n_2$ was the result of applying a function to element $n_1$.  Intuitively, semi-naive evaluation builds this graph in a breadth-first manner.

Pipelined semi-naive is a relaxation of semi-naive that supports input identifiers that change over time.  Pipelined semi-naive still proceeds in rounds, though there are several key differences.  First, in each round, all rules are executed to a {\em fixed point}, a state in which further application of rules no longer changes the value of any stream.  Their language is guaranteed to reach such a state, as, unlike ours, it operates over a fixed universe of values.  Second, there is no restriction that a round be applied to {\em all} available values in streams.  Some subset can be chosen.  Pipelined semi-naive corresponds to local breadth-first traversal.

Our canonical choice function is a relaxation of pipelined semi-naive, as we do not require application to a fixed point.  Further, we do not even require complete application of joins.


\wrm{Lindsey's lattice stuff}
