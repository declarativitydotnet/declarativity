\section{Stratification and Safety}
\label{sec:safety}
In the previous section we demonstrated that \slang can capture
intuitive notions of persistence and mutability of state via a
stylized use of Datalog.  However, the alert reader will note that
even simple \slang programs make for unusual Datalog: among other
concerns, persistence rules produce derivations for an infinite number
of values of the time suffix.  Traditional Datalog interpreters, which
work against static databases, would attempt to enumerate these
values, making this approach impractical.

However, in the context of distributed systems and networks, the need
for non-terminating ``services'' or ``protocols'' is very common.  In
this section we show that expressing distributed systems properties
such as persistence and mutable state in logic does not require
dispensing with familiar notions of safety and stratification: we take
traditional notions of acceptable Datalog programs, and extend them in
a way that admits sensible non-terminating programs.



\subsection{Stratification in {\large{\bf\slang}}}
\label{sec:strat}
We first turn our attention to the semantics of programs with negation.  As we
will see, the inclusion of time enables a syntactic stratification condition
for programs, and the existence of a unique model that corresponds to
intuition~\cite{local-strat}.






\begin{lemma} \label{lemma:no-neg-unique}
A \slang program without negation 
has a unique minimal model.
\end{lemma}

\begin{proof} 
A \slang program without negation 
is a pure Datalog
program.  Every pure Datalog program has a unique minimal model. 
\end{proof}



We define syntactic stratification of a \slang program the same way it is
defined for a Datalog program:

\begin{definition}
A \slang program is \emph{syntactically stratifiable} if there
exists no cycle with a negative edge 
in the program's
predicate dependency graph.
\end{definition}


We may evaluate such a program in {\em stratum order} as described in the
Datalog literature~\cite{ullmanbook}.
It is easy to see that any syntactically stratified \slang instance has a
unique perfect model~\cite{local-strat} because it is a syntactically stratified Datalog program.




However, many programs we are interested in expressing are not syntactically
stratifiable.  Fortunately, we are able to define a syntactically checkable
notion of {\em temporal stratifiability} of \slang programs that maps to a
subset of locally stratifiable Datalog programs.








\begin{definition} 
The \emph{deductive reduction} of a \slang program $P$ is
the subset of $P$ consisting of exactly the deductive rules in $P$.
\end{definition}

\begin{definition} 
A \slang program is \emph{temporally stratifiable} if its deductive
reduction is syntactically stratifiable.
\end{definition}

\begin{lemma}
\label{lemma:temp-strat-uniq}
Any temporally stratifiable \slang instance $P$ has a unique perfect model.
\end{lemma} 

\begin{proof}
Every temporally stratifiable \slang instance is locally
stratifiable~\cite{local-strat}, and thus has a unique perfect model.



\end{proof}


\begin{example}
A simple temporally stratifiable \slang program that is not syntactically stratifiable.

\begin{Dedalus}
persist[p\pos, p\nega, 3]  
  
p_pos(A, B, T) \(\leftarrow\)
  insert\_p(A, B, T);

p_neg(A, B, T) \(\leftarrow\)
  p_pos(A, B, T),
  delete\_p(T);
\end{Dedalus}

In the \slang program above, \emph{insert\_p} and \emph{delete\_p} are captured
in EDB relations.  This reasonable program is unstratifiable because $p\pos \succ
p\nega \land p\nega \succ p\pos$.  But because the successor relation is
constrained such that $\forall A,B, successor(A, B) \rightarrow B > A$, any
such program is locally stratified on time suffixes.  Therefore, we have
$p\pos_{n} \not\succ^+ p\_neg_{n} \not\succ^+ p\pos_{n+1}$; informally, earlier values
do not depend on later values.
\end{example}






\subsection{Temporal Safety}
Next we consider the issue of infinite results raised in the introduction to this section.
In traditional Datalog, this concern is well studied.
A Datalog program is considered {\em safe} if it has a finite minimal model, and hence has
a finite execution.  Safety in Datalog is traditionally ensured
through the following syntactic constraints:

\begin{enumerate}
\item No functions are allowed.
\item Variables are \emph{range restricted}: all attributes of the head goal
appear in a non-negated body subgoal.
\item The EDB is finite.
\end{enumerate}

These constraints ensure that the Herbrand Universe is finite: any atom that
may be deduced by a safe program may only take its attributes from the 
set of all constant symbols appearing in the program and EDB\@.
In fact, the set of all possible assignments of these constants to predicate
attributes, representing every possible interpretation, is itself finite. 

Since our definition of \dedalus{successor} violates these rules, and indeed
leads to an infinite set of facts, \slang programs violate this
definition of safety.  However, \dedalus{successor} models time, not computation;
later sections explain how \lang implementations avoid enumerating the contents
of \dedalus{successor} at runtime.
This section
introduces a notion of termination that allows us to reason about the safety of
\slang programs.


  



A \slang program containing only deductive rules is informally equivalent to a
Datalog program in which all predicates have no time suffix.  If all the rules
in such a program meet the conditions above, then clearly we would like them to meet \slang's definition of safety. 



\begin{definition}
A rule is \emph{instantaneously safe} if it is deductive,  function-free and range-restricted.
A \slang program is instantaneously safe if its deductive reduction is instantaneously safe.
\end{definition}


The \dedalus{successor} relation complicates the discussion of safety, as it
introduces the countably infinite set $\mathbb{Z}$ to our
universe of constants.

Consider the following \slang program, which derives a single, persistent fact:









\begin{example}
\label{ex:tempsafe}
An unsafe \slang instance?
\\
\begin{Dedalus}
persist[p\pos, p\nega, 2]
p(1, 2)@123;
\end{Dedalus}


The single ground fact will cause one deduction for each tuple in
\dedalus{successor}.  Since \dedalus{successor} is infinite, the corresponding
Datalog
program is unsafe.  
\end{example}

However, observe that each of these deductions produces a tuple that changes
only in its time suffix.  We find it useful to distinguish between unsafe
programs and programs that, given a finite EDB, eventually derive only tuples
that are equivalent except in their time suffixes.












\begin{definition}
Two sets of ground atoms $\Gamma$ and $\Gamma'$ are \emph{equivalent modulo
time} if each atom $\gamma \in \Gamma$ has a corresponding atom $\gamma' \in
\Gamma'$ such that $\gamma$ and $\gamma'$ have the same predicate symbol, and
the same assignment of constants to attributes for all attributes except the
time suffix.
\end{definition}


\begin{definition}
We say a \slang instance is \emph{quiescent at time $T$} if the set of all
atoms with time suffix $T$ is equivalent modulo time to the set of all atoms
with time suffix $T-1$.
\end{definition}


\begin{lemma}
A \slang instance that is quiescent at time $T$ will be quiescent until
timestamp of the next EDB fact $V$, i.e.\ for all $U \in \mathbb{Z}: V > U \ge
T$.  If no EDB fact has a timestamp greater than $T$, then the instance will be
henceforth quiescent.
\end{lemma}
\begin{proof}
A \slang program admits only deductive and inductive rules, which derive
new tuples at the same time as their ground tuples or in the immediate next
timestep.  Thus, the set of tuples true at time $T$ is completely determined by
any tuples true at time $T-1$, and any EDB facts true at time $T$.  Observe
that the integer value of the timestep does not influence the derivation.

If the instance is quiescent at $T$, then given ${\bf A}$, the set of atoms
with timestamp $T-1$, and the EDB at $T$, the program entails ${\bf
A}$ at timestamp $T$.  Thus in the absence of EDB facts at $T+1$, it entails
${\bf A}$ at $T+1$.
\end{proof}

\begin{definition}
A \slang instance with finite EDB is \emph{temporally safe} if it is henceforth
quiescent after some time $T$.
\end{definition}

\begin{definition}
Given the depends-on relation $\succ$ and its transitive closure $\succ^{*}$,
an intensional predicate $e$ in a program $P$ is called an \emph{instantaneous
predicate} if for every predicate $p$ for which $e \succ^{*} p$ (ie, $e$
depends transitively on $p$), either $p$ appears in the head of no inductive rules, or the body
of each inductive rule with head $p$  contains at least one positive instantaneous 
predicate.

\end{definition}

We propose the following conservative test for temporal safety.  A program is
guaranteed to be temporally safe if every rule is either:

\begin{enumerate}
\item An instantaneously safe rule, or
\item An inductive rule in which the head predicate occurs also in the
body with the same variable bindings for all attributes save the time suffix,
or
\item An inductive rule that has at least one instantaneous predicate as a
positive subgoal in the body.
\end{enumerate}


While a temporally safe program is henceforth quiescent after some time $T$,
a temporally unsafe program changes infinitely.  Note that
the \slang program in Example~\ref{ex:tempsafe} is temporally safe because
the basic persistence macro creates a rule that satisfies the second condition above.

\begin{lemma} 
A temporally stratifiable \slang instance is temporally safe if it has a finite EDB and every rule is one of the kinds 1-3 above.
\end{lemma}
\begin{proof}
Assume the program is temporally unsafe.  That is, there exists no time $T$
such that $\forall U \ge T$, the set of all atoms with timestamp $U$ is
equivalent modulo time to the set of all atoms with timestamp $T-1$.  Let $E$
be the maximum timestamp of any fact in the EDB.

Observe that every rule $r$ of kind 3 may only entail a finite number
of facts---as the EDB is finite---and thus may entail no facts at a
timestamp greater than some maximum timestamp $V_r \le E+1 \in
\mathbb{Z}$.  Since a \slang program has a finite set of rules we know
$\exists V \in \mathbb{Z} : \forall r: V \ge V_r$, and thus $V \le E+1$.

We now consider times $T$ such that $T > E+1$.  By the argument above, no rules
of kind 3 entail any facts with a timestamp greater than $E+1$.  Recall that
no EDB atoms are true at any timestamp greater than $E$.  Thus, any facts with
timestamp greater than $E+1$ must be entailed by rules of kind 1 or 2.

Consider the set of equivalence classes modulo time of all possible atoms, {\bf
A}, given the Herbrand Universe.  We know {\bf A} is finite, as the Herbrand
Universe is finite.  Therefore, if the program is temporally unsafe, then {\bf B}, the
set of atoms entailed by the program, both contains and excludes
infinitely many members of at least one equivalence class in {\bf A} (i.e.,
something ``infinitely oscillates in time'' between being true and false).
Since the program has finitely many rules, at least one rule must entail
infinitely many atoms (from at least one of the equivalence classes from {\bf
A}). Thus, it is easy to see that there must be a cycle that contains some predicate $P$ and $\lnot P$.

We show there exists such a cycle containing only rules of kind 1, which
implies that the program is temporally unstratifiable.  In order for such a
cycle to exist, $P$ must transitively depend on $\lnot P$, and $\lnot P$ must
transitively depend on $P$.  Thus, the program contains a rule $J_1$ with
$\lnot P$ in its body, and some predicate $R$ in its head, as well as a rule
$J_2$ that is transitively dependent on $R$, with $P$ in its head.

{\bf Case 1: }$P \neq R$.  In this case, $J_1$ must be of kind 1, as for any $Q
\neq P$, a rule of kind 2 with $P$ in the head may not directly entail $Q$
given $P$.  $J_2$ must also be of kind 1---if it is of kind 2, then it
necessarily contains $P$ in its body, so it cannot entail $P$ unless $P$ is
entailed by some other rule.  If $J_2$ contains $R$ in its body, then the
program is syntactically unstratifiable.  But if $J_2$ does not contain $R$ in
its body, then it contains some predicate $S$ transitively entailed by $R$;
without loss of generality, the body contains $R$.  Thus, the program is syntactically unstratifiable.

{\bf Case 2: }$P = R$.  In this case, $J_1$ and $J_2$ are the same rule: $P
\leftarrow \lnot P$.  Thus, the program is syntactically unstratifiable.

Thus, the program is temporally unstratifiable, which contradicts our
assumption.
\end{proof}

\begin{example}
A \slang instance with a temporally unsafe deductive rule.


\begin{Dedalus}
p(A, B) \(\leftarrow\) q(A);
\end{Dedalus}

The program above has a temporally unsafe deductive rule that corresponds to an
unsafe rule in Datalog: it is not range-restricted.  The head variable $B$
could range over an infinite set of constants.
\end{example}


\begin{example} 
A \slang instance that is temporally unsafe due to infinite oscillation.

\begin{Dedalus}
flip\_flop(B, A)@next \(\leftarrow\) flip\_flop(A, B);
flip\_flop(0, 1)@1;
\end{Dedalus}

The above program exemplifies temporally unsafe induction. Even though it
contains no function symbols and all variables are range-restricted, it
entails infinite oscillation of the \emph{flip\_flop} predicate.  
\end{example}

We can imagine interesting examples of temporally unsafe programs, and do not forbid them
in \slang. 


















