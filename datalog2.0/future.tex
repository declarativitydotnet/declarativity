\section{Conclusion}
Datalog has inspired a variety of recent applied work, which touts the benefits
of declarative specifications for practical implementations.  We have developed
substantial experience building distributed
systems~\cite{boom-eurosys,Alvaro2009I-Do-Declare:-C,Chu:2007,Loo2009-CACM}
using hybrid declarative/imperative languages such as
Overlog~\cite{Loo2009-CACM}.  While our experience with those languages was
largely positive, the combination of Datalog and imperative constructs often
clouded our understanding of the ``correct'' execution of single-node programs
that performed state updates.  This work developed in large part as a reaction
to the semantic difficulties presented by these distributed logic languages.

Through its reification of time as data, \lang allowed us to achieve the goal of
a purely declarative language, without sacrificing the ability to express
two critical features of practical distributed systems: mutable state and
asynchronous communication. We believe that \lang is as expressive as Overlog,
but formalizing this intuition is difficult because the semantics of Overlog are
not well specified.  Instead, we are currently validating the practicality of
our work by ``porting'' many of our Overlog programs to \lang.

In \lang, state update and communication differ from logical deduction only in
terms of timing.  In the local case, this allows us to express state update
without giving up the clean semantics of Datalog; unlike Datalog extensions that
use imperative constructs to provide such functionality, each \lang rule
expresses a logical invariant that will hold over all program executions.
However, interactions with external processes and asynchronous communication
introduce nondeterminism which \lang models with \dedalus{choose}.  Our hope is
that modeling external processes and events with a single primitive will
simplify efforts to formally verify the correctness of distributed systems
implemented using \lang. Two natural directions in this vein are to determine
for a given \lang program whether Church-Rosser confluence holds for all models
produced by \dedalus{choice}, or to capture finer-grained notions like
serializability of such models with respect to transaction identifiers embedded
in EDB facts.
