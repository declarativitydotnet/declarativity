\documentclass{vldb}
\usepackage{color}
\usepackage{graphicx}
\usepackage{url}
\usepackage{balance}  % for  \balance command ON LAST PAGE  (only there!)
\usepackage{xspace}
\usepackage[T1]{fontenc}
\usepackage{times}
%\usepackage{mathptmx}    % use "times" font, including for math mode
\usepackage{txfonts}  % apparently needed to fixup the formatting of lstlistings
\usepackage{textcomp}
\usepackage[protrusion=true,expansion=true]{microtype}
\usepackage{paralist}
\usepackage{comment}

\frenchspacing

\usepackage{listings}
\lstset{
basicstyle=\ttfamily\scriptsize,       % the size of the fonts that are used for the code
numbers=left,                   % where to put the line-numbers
numberstyle=\ttfamily,      % the size of the fonts that are used for the line-numbers
%aboveskip=0pt,
%belowskip=0pt,
stepnumber=1,                   % the step between two line-numbers. If it is 1 each line will be numbered
%numbersep=10pt,                  % how far the line-numbers are from the code
breakindent=0pt,
firstnumber=1,
%backgroundcolor=\color{white},  % choose the background color. You must add \usepackage{color}
showspaces=false,               % show spaces adding particular underscores
showstringspaces=false,         % underline spaces within strings
showtabs=false,                 % show tabs within strings adding particular underscores
frame=leftline,
tabsize=2,  		% sets default tabsize to 2 spaces
captionpos=b,   		% sets the caption-position to bottom
breaklines=false,    	% sets automatic line breaking
breakatwhitespace=true,    % sets if automatic breaks should only happen at whitespace
%escapeinside={\%}{)}          % if you want to add a comment within your code
columns=fixed,
basewidth=0.52em,
% are you fucking kidding me lstlistings?  who puts the line numbers outside the margin?
xleftmargin=6mm,
xrightmargin=-6mm,
numberblanklines=false,
language=Ruby,
morekeywords={table,scratch,channel,interface,periodic,bloom,state,bootstrap,morph,monotone,lset,lbool,lmax,lmap}
}
\lstset{escapeinside={(*}{*)}}

\def\lang{Bloom$^L$\xspace}

% \newcommand{\nrc}[1]{{\textcolor{magenta}{#1 -- nrc}}}
% \newcommand{\wrm}[1]{{\textcolor{blue}{#1 -- wrm}}}
% \newcommand{\jmh}[1]{{\textcolor{red}{#1 -- jmh}}}

\begin{document}

\title{Logic and Lattices for Distributed Programming}

\numberofauthors{4}

\author{
\alignauthor
Neil Conway\\
       \affaddr{UC Berkeley}\\
       \email{nrc@cs.berkeley.edu}
\alignauthor
William R.\ Marczak\\
       \affaddr{UC Berkeley}\\
       \email{wrm@cs.berkeley.edu}
\alignauthor
Peter Alvaro\\
       \affaddr{UC Berkeley}\\
       \email{palvaro@cs.berkeley.edu}
\and
\alignauthor
Joseph M.\ Hellerstein\\
       \affaddr{UC Berkeley}\\
       \email{hellerstein@cs.berkeley.edu}
}

\maketitle

\begin{abstract}
  Monotonic logic has recently been identified as a useful building block for
  consistent distributed programs because it allows eventual consistency without
  the need for coordination. However, many common distributed programs cannot
  naturally be expressed as monotonic programs over sets. We present \lang, a
  variant of Datalog that allows monotonic programs to use arbitrary lattice
  structures, not just the set lattice. The ability to use multiple lattices
  allows \lang to verify a much wider range of programs as monotonic and hence
  eventually consistent.  We show how to implement lattice-based code
  efficiently, in the style of semi-naive evaluation. Finally, we show how \lang
  can be used to implement practical distributed programs.
\end{abstract}

\section{Introduction}
%Our research is motivated by two hard problems in distributed systems.  First,
\wrm{show examples of the problems (not necessarily code) -- evolving state and unreliable communication}

%Distributing any system introduces nondeterminism.  For example, one may
%distribute a computation over many inexpensive, but unreliable, commodity
%machines (e.g. RAID).  The status of internet links and widely distributed
%nodes is inherently more unreliable than multiple cores on a single die, or
%multiple CPUs in a single computer.  

%We present {\bf \lang}, a foundation language for programming and
%reasoning about distributed systems.  

%We correct deficiencies in earlier attempts, and introduce a compelling notion
%of non-determinism in the language.  We specifically use non-determinism to
%reason about {\em when} a deduction becomes visible, including the possibility
%that the deduction will never be visible.  Programmers can constrain this
%non-determinism by using well-studied techniques in distributed systems, such as
%Lamport Clocks 


Traditional database systems are based on declarative query languages that
specify transformations as dataflows over an updatable store.  Such query
languages are either not expressive enough to capture common programming
constructs \wrm{like what?}, or are at best awkward to use in this fashion.
\wrm{todo: transition that explains Datalog's birth from these languages... I
don't know enough to write it} The family of logic-based database languages, of
which Datalog is the progenitor, represent expressive programming languages
that produce similar dataflow representations.  Datalog is purely deductive: a
program specifies the rules by which the derived relations are populated based
on a static database, which is never updated.  Recent programming language
research has explored the use of Datalog-based languages for expressing
distributed systems.  Because the state of any complex system evolves with its
execution, these efforts were forced to extend the Datalog model by admitting
updates, additions and deletions of the EDB.  Unfortunately, these previous
attempts were plagued with ambiguities about how and when state changes occur
and become visible, putting a heavy burden on the programmer to ensure even
simple properties, such as atomicity of updates over time.

In contrast to reasoning about state change procdurally, \lang observes
that this concept is intuitively expressed as invariants over {\em time}.  In
this work, we present a formal model of Datalog augmented with time extensions.
By reifying time as data an introducing it into the logic, \lang eliminates
previous ambiguities, ensures atomicity of updates and makes it possible to
express system invariants that can guarantee liveness properties, a key
challenge in building distributed systems.

\section{Background}
\label{sec:background}
\begin{table}
\begin{tabular}{|c|l|p{1.85in}|}
\hline
\textbf{Op} & \textbf{Name} & \textbf{Meaning} \\
\hline
\verb|<=| & \emph{merge} & lhs includes the content of rhs in the
current timestep \\
\hline
\verb|<+| & \emph{deferred merge} & lhs will include the content of rhs in the
next timestep \\
\hline
\verb|<-| & \emph{deferred delete} & lhs will not include the content of the rhs
in the next timestep \\
\hline
\verb|<|$\sim$ & \emph{async merge} & (remote) lhs will include the content of the
rhs at some non-deterministic future timestep\\
\hline
\end{tabular}
\caption{Bloom operators.}
\label{tbl:bloom-ops}
\end{table}

\begin{figure}[t]
\begin{scriptsize}
\begin{lstlisting}
class AllPaths
  include Bud

  state do
    table :link, [:from, :to, :cost]
    table :path, [:from, :to, :next_hop, :cost]
  end

  bloom :make_paths do
    path <= link {|l| [l.from, l.to, l.to, l.cost]}
    path <= (link*path).pairs(:to => :from) do |l,p|
      [l.from, p.to, l.to, l.cost + p.cost]
    end
  end
end
\end{lstlisting}
\end{scriptsize}
\caption{A Bloom program to compute the transitive closure of the
  \emph{link} relation.}
\label{fig:bloom-spaths}
\end{figure}

In this section, we briefly review the Bloom programming language and the CALM
program analysis technique.  We highlight a simple distributed protocol for
which the CALM analysis produces unsatisfactory results.

\subsection{Bloom}
\label{sec:bg-bloom}

\subsection{CALM Analysis}
\label{sec:bg-calm}

\section{Adding Lattices to Bloom}
\label{sec:lang}

In this section, we introduce \lang, an extension of Bloom that allows monotonic
programs to be written using arbitrary lattices. We begin by reviewing the
algebraic properties of lattices, monotone functions, and morphisms. We then
introduce the basic concepts of \lang and detail the builtin lattices provided
by the language. We also show how users can define their own lattice types using
a simple Ruby API.

% is this the right place for this?
When designing \lang, we decided to extend Bloom to include support for
lattices rather than building a new language from scratch. Hence, \lang is
backward compatible with Bloom and was implemented with relatively minor
changes to the Bud runtime. This design decision also required that we consider
rules written over lattices should interoperate with rules that use traditional
Bloom relations; we added several \lang features to ease this interoperability,
which we describe in Section~\ref{sec:bloom-interop}.

\subsection{Definitions}
\label{sec:lattice-defn}
A \emph{bounded join semilattice} is a triple $\langle S, \lor, \bot\rangle$,
where $S$ is a poset, $\lor$ is a binary operator (called ``join'' or ``least
upper bound''), and $\bot \in S$. $\lor$ is associative, commutative, and
idempotent. For all $x, y \in S$, $x \lor y = z$, where $x \leq_S z, y \leq_S
z$, and there is no $z' \in S$ such that $z' <_S z$ (where $<_S$ is the partial
order associated with the poset $S$). Note that although the underlying set only
has a partial order, the least upper bound is defined for all elements $x,y \in
S$. The distinguished element $\bot$ is the smallest element in $S$; this
implies that $x \lor \bot = x$ for all $x \in S$. For brevity, we use the term
``lattice'' to mean ``bounded join semilattice'' in the rest of this paper.

% XXX: note that algebraic properties that must be satisfied by morphisms and
% monotone functions
A \emph{monotone function} from poset $S$ to poset $T$ is a function $g: S \to
T$ such that $\forall a,b \in S: a \leq_S b \Rightarrow g(a) \leq_T g(b)$. That
is, $g$ maps elements of $S$ to elements of $T$ in a manner that is consistent
with the partial orders of both posets.

% XXX: mention that morphisms must be distributive with respect to the lub of
% their domain, whereas monotone functions don't need to be?
A \emph{morphism} from lattice $\langle X, \lor_X, \bot_X\rangle$ to lattice
$\langle Y, \lor_Y, \bot_Y\rangle$ is a function $f$ such that, $\forall a,b \in
X: f(a \lor_X b) = f(a) \lor_Y f(b)$. That is, $f$ allows elements of $X$ to be
converted into elements of $Y$ in a way that preserves the lattice properties.
Note that all morphisms are monotone functions but the converse is not true in
general.

\subsection{Language concepts}
In \lang, state is represented using lattices and computation is expressed as
functions over lattices. A lattice in \lang is analogous to a collection type in
Bloom, while a lattice element corresponds to a particular collection value. For
example, the \texttt{lset} lattice provides similar functionality to the
set-oriented collections provided by Bloom; an element of the \texttt{lset}
lattice is a particular set value. In the terminology of object-oriented
programming, a lattice is a class that obeys a certain interface and an element
of a lattice is an instance of that class.

As in Bloom, the lattices used by a \lang program are declared in a
\texttt{state} block. More precisely, the declarations in the state block
introduce identifiers that are associated with lattice elements; over time, the
binding between identifiers and lattice elements is updated to reflect state
changes in the program. For example, line~\ref{line:quorum-lset-decl} of
Figure~\ref{fig:lattice-quorum} declares an identifier \texttt{votes} that is
mapped to an element of the \texttt{lset} lattice. As more votes are received,
the lattice element associated with the \texttt{votes} identifier
changes---specifically, it moves upward in the \texttt{lset} lattice.

\subsubsection{Computation in \lang}
Statements take the same form in both Bloom and \lang. The identifier on the lhs
of a statement can refer to either a set-oriented collection or a lattice. The
expression on the rhs can contain both traditional relational operators (applied
to Bloom collections) and method invocations (applied to lattice elements).

Note that if the lhs of a statement refers to a lattice, the statement's
operator must be either \verb|<=| or \verb|<+|, denoting instaneous or deferred
deduction. \lang does not currently support a notion of ``deletion'' for
lattices. Lattices do not directly support asynchronous communication (via the
\verb|<~| operator), but lattice elements can be embedded into channels (as
described in Section~\ref{sec:bloom-interop}).

\subsection{Builtin lattice types}
\label{sec:lattice-builtins}

\begin{table*}[t]
\begin{tabular}{|l|l|l|l|l|}
\hline
\textbf{Name} & \textbf{Description} & \textbf{Merge} & \textbf{Morphisms} &
\textbf{Monotone functions}\\
\hline
\texttt{lbool} & Boolean lattice (false $\to$ true) & & \texttt{when\_true} & \\
\texttt{lmax} & Max over an ordered domain & &\texttt{gt},
\texttt{gt\_eq}, \texttt{+}, \texttt{-} & \\
\texttt{lmin} & Min over an ordered domain & &\texttt{lt}, \texttt{lt\_eq},
\texttt{+}, \texttt{-} & \\
\texttt{lset} & Set of values & & \texttt{intersect}, \texttt{product},
\texttt{project} & \texttt{size} \\
\texttt{lpset} & Set of non-negative numbers & &
\texttt{intersect}, \texttt{product}, \texttt{project}& \texttt{size}, \texttt{sum} \\
\texttt{lbag} & Multiset of values & & \texttt{intersect},
\texttt{project}, \texttt{mult(k)}, \texttt{+} & \texttt{size}\\
\texttt{lmap} & Map from key to lattice values & &
\texttt{intersect}, \texttt{project}, \texttt{at(k)}, \texttt{key?(k)} & \texttt{size}\\
\hline
\end{tabular}
\caption{The builtin lattices in \lang.}
\label{tbl:builtin-lattices}
\end{table*}

Table~\ref{tbl:builtin-lattices} lists the builtin lattices provided by
\lang. Many common distributed protocols can be expressed using these lattices
(e.g., the causal delivery protocol described in Section~\ref{sec:causal}).

Note that \emph{size} is a monotone function provided by several lattices, but
it is not a morphism. This is because \texttt{size} cannot be distributed over
the merge functions of those lattices. For example, \ldots

\subsection{Lattice API}
\label{sec:lattice-api}
In \lang, each lattice has an associated Ruby class, which we call the
\emph{lattice class}. An instance of this class is called a \emph{lattice
  element}. A lattice element represents a single point in the lattice---i.e., an
element of the poset associated with the lattice.

A lattice class is a normal Ruby class that meets a certain API contract. Every
lattice class inherit from the builtin \texttt{Bud::Lattice} class, and
must also define two methods:
\begin{itemize}
\item \texttt{initialize(i)}: given a Ruby object \emph{i}, this method
  constructs a new lattice element that ``wraps'' \emph{i} (\texttt{initialize}
  is just the normal Ruby syntax for defining a constructor). By convention, the
  Ruby value wrapped by a lattice element is assigned to a Ruby member variable
  \texttt{@v}. If $i$ is the null reference, this method returns the least
  element of the lattice.

\item \texttt{merge(e)}: given a lattice element \emph{e}, this method returns the
  lattice element that is the least upper bound of $\{e, \textit{self}\}$. This method must
  satisfy the algebraic properties summarized in Section~\ref{sec:lattice-defn}---in
  particular, it must be commutative, associative, and idempotent. Note that
  \emph{e} must have the same class as \emph{self}.
\end{itemize}
Lattice elements are \emph{immutable} (e.g., \texttt{merge} functions should
construct a new lattice element rather than modifying one of their inputs
in-place). Efficient lattice implementations may \emph{share structure} on merge
operations, as is common practice for immutable data structures in functional
programming languages~\cite{Okasaki1999}. % XXX: maybe not the right place for this

\begin{figure}[t]
\begin{scriptsize}
\begin{lstlisting}
class Bud::SetLattice < Bud::Lattice
  wrapper_name :lset

  def initialize(x=[])
    # Reject invalid input (elided)
    @v = x.uniq # Remove duplicates from input
  end

  def merge(i)
    self.class.new(@v | i.reveal)
  end

  morph :intersect do |i|
    self.class.new(@v & i.reveal)
  end

  morph :pro do |&blk|
    @v.map(&blk)
  end

  ord_map :size do
    Bud::MaxLattice.new(@v.size)
  end
end
\end{lstlisting}
\end{scriptsize}
\caption{The implementation of the \texttt{lset} lattice in Ruby.}
\label{fig:lattice-set}
\end{figure}

\subsection{Integration with set-oriented logic}
\label{sec:bloom-interop}

\begin{figure}[t]
\begin{scriptsize}
\begin{lstlisting}
class ShortestPaths
  include Bud

  state do
    table :link, [:from, :to, :c]
    table :path, [:from, :to, :next_hop] => [:c]
    table :min_cost, [:from, :to] => [:c]
  end

  bloom do
    path <= link {|l| [l.from, l.to, l.to, MinLattice.new(l.c)]}
    path <= (link*path).pairs(:to => :from) do |l,p|
      [l.from, p.to, l.to, p.c + l.c]
    end
    min_cost <= path {|p| [p.from, p.to, p.c]}
  end
end
\end{lstlisting}
\end{scriptsize}
\caption{A \lang program to compute the all-pairs shortest paths of a
  graph.}
\label{fig:lattice-spaths}
\end{figure}

\lang provides two features to ease integration of lattice-based code with
traditional Bloom programs that manipulate set-oriented collections.

\subsubsection{Implicit fold}
% XXX: this ignores the fact that Bloom collections consist of sets of tuples,
% whereas implicit fold works for sets of singleton values
% XXX: refer to shortest paths program as practical example
This feature enables set-oriented collections to be more easily used as input to
lattices. If a \lang statement has a set-oriented collection on the rhs and a lattice
on the lhs, the lattice merge function is used to ``fold over'' the elements of
the collection. That is, each element of the collection is converted to a
lattice element (via the appropriate lattice constructor); then the set of
lattice elements are merged together (via repeated application of the
\texttt{merge} method). In our experience, this is typically the behavior
intended by the user.

\subsubsection{Collections with embedded lattice values}
It would be convenient to allow lattice elements to be stored as attributes of
tuples that appear in set-oriented Bloom collections. Furthermore, Bloom
provides several facilities (e.g., network communication, persistent storage,
module interfaces) as collections with special semantics; it would be
unfortunate if a redundant set of facilities would be necessary to support
lattice-based code. A simple solution would be to extract the underlying Ruby
value from the lattice element (e.g., using the \texttt{reveal} method), and
then store that value as a tuple attribute in a set-oriented
collection. Unfortunately, that would introduce needless non-monotonicity into
the program.

Storing lattice elements as attributes of tuples in set-oriented collections
introduces several challenges. Consider a simple \lang statement that derives tuples
with a lattice element as an attribute value:
\begin{verbatim}
    t1 <= t2 {|t| [t.x, lat_foo]}
\end{verbatim}
where \texttt{t1} and \texttt{t2} are Bloom relations and \texttt{lat\_foo}
identifies a lattice; suppose that the first column of \texttt{t1} is the
relation's key. The value associated with \texttt{lat\_foo} can change over the
course of the fixpoint computation (specifically, it can grow ``upward''
according to the lattice's partial order, as more values are merged into the
lattice). Implemented naively, this might result in multiple \texttt{t1} tuples
with different values for the second attribute, which would violate
\texttt{t1}'s key (the first column of \texttt{t1} would not functionally
determine a single value for the second column).

This problem could be avoided by placing constraints on the evaluation order of
statements: for example, we could require that all potential changes to
\texttt{lat\_foo} be completed before rules that embed \texttt{lat\_foo} could
be evaluated. This would effectively stratify the program according to lattice
embedding rules, which would disallow cycles through lattice
embeddings~\cite{Apt1988}. This would reject intuitively reasonable programs; it
also seems unsatisfying to require stratification of monotonic programs.

% Clarify this
Instead, \lang allows rules to produce multiple tuples that differ only in their
embedded lattice values. During the course of the fixpoint computation, those
values are merged together using the appropriate lattice merge function. This is
safe because Bud stratifies programs according to non-monotonic operators;
hence, any operators that might be applied to an embedded lattice value before
it has been determined exactly must be monotonic. Nevertheless, this solution is
somewhat counterintuitive because tuples in Datalog relations are traditionally
immutable: once a fact is known to be true, its value remains the
same.% \footnote{Bloom facts can be deleted, but this is an explicit non-monotonic
  % operation that can only occur between timesteps. Conceptually, Bloom models
  % update as the retraction of the previous version of a fact and the insertion
  % of a new fact~\cite{dedalus}.}
% Should we note that we might add an option to disable this behavior for
% particular attributes, or explain more about why this might be considered
% weird?

For similar reasons, we currently disallow lattice values from being used as
keys in Bloom collections. It might be possible to relax this restriction in
certain ``safe'' cases, but we have not found this limitation to be problematic
to date.

\subsection{Confluence in \lang}
\nrc{TODO: justify that CALM continues to hold for monotonic programs over
  lattices.}

\section{Implementation}
\label{sec:impl}

In this section, we describe how \lang programs can be evaluated. First, we
detail a variant of semi-naive evaluation that supports lattices. We validate
that our implementation of semi-naive evaluation results in significant
performance gains and is competitive with a traditional set-oriented version of
semi-naive evaluation. We also detail the engineering effort required to extend
Bud to support \lang. % XXX: last sentence is weak

\subsection{Evaluation strategy}
\label{sec:lattice-eval-strat}
\emph{Naive} evaluation is a simple but inefficient approach to evaluating
recursive Datalog programs. Evaluation proceeds in ``rounds.'' In each round, all
the rules in the program are evaluated over the entire database (including all
derivations made in previous rounds). This process stops when a round makes no
new derivations. Naive evaluation is inefficient because it makes many redundant
derivations: once a fact has been derived in round $i$, it is rederived in every
subsequent round.

\nrc{NOT DONE!}
\emph{Semi-naive} evaluation improves upon naive evaluation by making fewer
redundant derivations~\cite{Balbin1987}. Let $\Delta_0$ represent the initial
database state. In the first round, all the rules are evaluated over $\Delta_0$;
let $\Delta_1$ represent the new facts derived in this round. In the second
round, we only need to compute derivations that are dependent on
$\Delta_1$---everything that can be derived purely from $\Delta_0$ has already
been computed.

A similar evaluation strategy works for \lang statements that invoke lattice
morphisms. For each lattice identifier $i$, we record two lattice elements:
$v_i$ and $\Delta^r_i$. $v_i$ is the least upper bound of all the lattice
elements ever produced by any statement with $i$ on the lhs---that is, it is the
current lattice element associated with $i$. $\Delta^r_i$ represents the new
derivations for $i$ that have been made in evaluation round $r$. During round
one, the program's statements are evaluated and $i$ is mapped to $v_i$; this
computes $\Delta^1_i$. In round two, $i$ is now mapped to $\Delta^1_i$ and
evaluating the program's statements yields $\Delta^2_i$. This process continues
until $\Delta^j_i = \Delta^{j+1}_i$ for all identifiers $i$.

This optimization can be used for morphisms, but not ordinary monotone
functions. This is because seminaive evaluation requires the ability to invoke a
method on two different lattice values $\Delta^j_i$ and $\Delta^k_i$

 relies on the ability to apply a
method to $\Delta^r_i$, and then later merge $\Delta^r_i$ with $v_i$ to obtain
the final value for $i$. This is safe for morphisms:

 During the first round, the program's statements are evaluated and
lattice identifiers are mapped to lattice elements in the normal way. At the end
of the round, each identifier also has a ``delta'' value that represents new
derivations made for statements with that identifier on the lhs. In round 2, the
lattice identifier is mapped to the ``delta'' value from round 1, rather than
the lattice identifiers are mapped to the delta element of the respective la

 over
the current value associated with each lattice element. Rather than merging rhs
values directly into the lhs lattice, 

A similar idea can be applied to \lang programs that apply morphisms to
lattices. However, semi-naive evaluation \emph{cannot} be used for \lang
statements that invoke monotone functions. This is because seminaive evaluation
requires the ability to split the input into pieces (``deltas''), and apply the
function to each piece, then merge together the output of the function over the
pieces. Morphisms allow this, but monotone functions do not. For example,
consider how we might compute the \texttt{size} monotone function over an
\texttt{lset} lattice.

% The basic insight is that if a fact is derived for the first time in round $i$,
% it must somehow depend on a fact that was derived for the first time in round
% $i-1$; otherwise, $f$ would have been derived earlier. Hence, we can rewrite the
% program to evaluate

\subsection{Performance validation}
\label{sec:lattice-perf}
\begin{figure}[t]
\includegraphics[width=\linewidth]{fig/sn_perf}
\caption{Performance comparison of three different methods for computing the
  transitive closure of a graph.}
\label{fig:tc-perf-graph}
\end{figure}

To validate the effectiveness of semi-naive evaluation for \lang programs, we
wrote two versions of a program to compute the transitive closure of a directed
acyclic graph. One version was written in Bloom and used traditional
set-oriented collections. The other version was written in \lang using morphisms
over the \texttt{lset} lattice. For the \lang version, we ran the program both
with and without semi-naive evaluation enabled. As input, we used synthetic
graphs of various sizes---in a graph with $n$ nodes, each node had $O(\log_2 n)$
outgoing edges. We ran the experiment on a late 2010 MacBook Air with a 2.13 Ghz
Intel Core 2 Duo processor and 4GB of RAM, running Mac OS X 10.7.3 and Ruby
1.8.7-p352. We ran each program variant five times on each graph and report the
mean elapsed wall-clock time.

Figure~\ref{fig:tc-perf-graph} shows how the runtime of each program varied with
the size of the graph. Note that we only report results for the naive \lang
strategy on small input sizes because this variant ran very slowly as the graph
size increased. The poor performance of naive evaluation is not surprising:
after deriving all paths of length $n$, naive evaluation will then rederive all
those paths at every subsequent ``step'' of the fixpoint computation. In
contrast, after computing length $n$ paths, a semi-naive strategy will only
generate length $n+1$ paths in the next step. Bloom and semi-naive \lang achieve
similar results. We instrumented Bud to count the number of derivations made by
the Bloom and semi-naive lattice variants---as expected, both programs made about
the same number of derivations. These results suggest that our implementation of
semi-naive evaluation for \lang is effective and performs comparably with a
traditional Datalog system.

For large inputs, Bloom began to outperform the semi-naive lattice variant. We
suspect this is because our current implementation of lattices requires more
data copies. Lattices are immutable, so the \texttt{lset} merge function
allocates a new object to hold the result of the merge. In contrast, Bloom
collections are modified in-place. We plan to address this by allowing the
lattice runtime to avoid copies when it can determine that in-place updates are
safe.

\subsection{Modifying Bud}
We were able to extend Bud to support \lang with relatively minor changes. Bud
initially had about 7200 lines of Ruby source code (LOC). The core lattice
features (the \texttt{Bud::Lattice} base class and the mapping from identifiers
to lattice elements) required about 300 LOC. Modifying Bud's fixpoint logic to
include lattices required only 10 LOC, while the program rewriting required to
enable semi-naive evaluation required 100 LOC. Modifying Bud's collection
classes to support merging of embedded lattice values required modifying about
125 LOC. The builtin lattice classes constituted an additional 300 LOC. In
total, adding support for \lang required less than 900 lines of added or
modified code, and took about two man-months of engineering time.%  This
% experience suggests that support for lattices can be added to an existing
% Datalog engine in a relatively straightforward manner.

\section{Case Study: Shopping Cart}
\label{sec:case}
In this section, we develop two different designs for a distributed
shopping-cart service in Bloom. In a
shopping cart system, clients add and remove items from their shopping cart. To
provide fault tolerance and persistence, the content of the cart is stored by a
collection of server replicas. Once a client has finished shopping, they perform
a ``checkout'' request, which returns the final state of their cart.

After presenting the abstract shopping cart protocol and a simple client
program, we implement a ``destructive,'' state-modifying shopping cart service
that uses the key-value store introduced in Section~\ref{sec:kvs}. Second, we
illustrate a ``disorderly'' cart that accumulates updates in a set-wise fashion,
summarizing updates at checkout into a final result.  These two different
designs illustrate our analysis tools and the way they inform design decisions
for distributed programming.

\subsection{Shopping Cart Client}
An abstract shopping cart protocol is presented in
Figure~\ref{fig:cart-protocol}. Figure~\ref{fig:cart-client} contains a simple
shopping cart client program: it takes client operations (represented as
\texttt{client\_action} and \texttt{client\_checkout} facts) and sends them to
the shopping cart service using the CartProtocol. We omit logic for clients to choose a cart server replica; this can be based on simple policies like round-robin or random selection, or via more explicit load balancing.

\begin{figure}[t]
\begin{scriptsize}
\begin{lstlisting}
module CartProtocol
  def state
    channel :action_msg, (*\label{line:dec_chan_beg}*)
      ['@server', 'client', 'session', 'reqid'],
      ['item', 'action'] (*\label{line:dec_chan_end}*)
    channel :checkout_msg,
      ['@server', 'client', 'session', 'reqid']
    channel :response_msg,
      ['@client', 'server', 'session'], ['contents']
  end
end

module CartClientProtocol
  def state
    interface input, :client_action, (*\label{line:dec_in_interface}*)
      ['server', 'session', 'reqid'], ['item', 'action'] 
    interface input, :client_checkout,
      ['server', 'session', 'reqid']
    interface output, :client_response, 
      ['client', 'server', 'session'], ['contents']
  end
end
\end{lstlisting}
\vspace{-10pt}
\caption{Abstract shopping cart protocol.}
\label{fig:cart-protocol}
\end{scriptsize}
\vspace{-2pt}
\end{figure}

\begin{figure}[t]
\begin{scriptsize}
\begin{lstlisting}
module CartClient
  include CartProtocol
  include CartClientProtocol

  declare
  def client
    action_msg <~ client_action.map do |a| 
      [a.server, @local_addr, a.session, a.reqid, a.item, a.action] (*\label{line:des_client_action}*)
    end
    checkout_msg <~ client_checkout.map do |a| 
      [a.server, @local_addr, a.session, a.reqid]
    end
    client_response <= response_msg
  end
end
\end{lstlisting}
\vspace{-10pt}
\caption{Shopping cart client implementation.}
\label{fig:cart-client}
\end{scriptsize}
\vspace{-2pt}
\end{figure}

\subsection{``Destructive'' Shopping Cart Service}

\begin{figure}[t]
\begin{scriptsize}
\begin{lstlisting}
module DestructiveCart
  include CartProtocol
  include KVSProtocol

  declare (*\label{line:des_declare}*)
  def do_action
    kvget <= action_msg.map{|a| [a.reqid, a.key]}

    kvput <= action_msg.map do |a|
      if a.action == "A" (*\label{line:des_kvstate_gen_beg}*)
        unless kvget_response.map{|b| b.key}.include? a.session (*\label{line:des_kvstate_check}*)
          [a.server, a.client, a.session, a.reqid, [a.item]]
        end
      end (*\label{line:des_kvstate_gen_end}*)
    end

    old_state = join [kvget_response, action_msg],
                     [kvget_response.key, action_msg.session]  (*\label{line:des_join_kvstate_action_msg}*)
    kvput <= old_state.map do |b, a|
      if a.action == "A" (*\label{line:des_replace_beg}*)
        [a.server, a.client, a.session, 
         a.reqid, b.value.push(a.item)]
      elsif a.action == "D"
        [a.server, a.client, a.session, 
         a.reqid, delete_one(b.value, a.item)]
      end (*\label{line:des_replace_end}*)
    end
  end

  declare
  def do_checkout
    kvget <= checkout_msg.map{|c| [c.reqid, c.session]} (*\label{line:kvget-beg}*)
    lookup = join [kvget_response, checkout_msg],
                  [kvget_response.key, checkout_msg.session]
    response_msg <~ lookup.map do |r, c|
      [c.client, c.server, c.session, r.value] (*\label{line:kvget-end}*)
    end
  end
end
\end{lstlisting}
\vspace{-10pt}
\caption{Destructive cart implementation.}
\label{fig:dest-cart}
\end{scriptsize}
\vspace{-2pt}
\end{figure}

We begin with a shopping cart service built on a key-value store.  Each cart is
a \texttt{(key,value)} pair, where \texttt{key} is a unique session identifier
and \texttt{value} is an object containing the session's state, including a Ruby
array that holds the items currently in the cart. Adding or deleting items from
the cart result in ``destructive'' updates: the value associated with the key is
replaced by a new value that reflects the effect of the update. Deletion
requests are ignored if the item they refer to does not exist in the cart.

Figure~\ref{fig:dest-cart} shows the Bloom code for this design. The
\texttt{kvput} collection is provided by the abstract KVSProtocol described in
Section~\ref{sec:kvs}. Our shopping cart service would work with any concrete
realization of the KVSProtocol; we will choose to use the replicated key-value
store (Section~\ref{sec:rep-kvs}) to provide fault-tolerance.

When client actions arrive from the CartClient, the cart service checks to see
if there is a record in the key-value store associated with the client's
session. If no record is found (i.e., this is the first operation for a new
session), then
lines~\ref{line:des_kvstate_gen_beg}--\ref{line:des_kvstate_gen_end} generate an
entry for the new session in \texttt{kvstate}.  Otherwise, the join conditions
in line~\ref{line:des_join_kvstate_action_msg} are satisfied and
lines~\ref{line:des_replace_beg}--\ref{line:des_replace_end} ``replace'' the
value in the key-value store with an updated set of items for this session; this
uses the built-in overwriting capability provided by the key-value store.  When
a \texttt{checkout\_msg} appears at a server replica, the key-value store is
queried to retrieve the cart state associated with the given session
(lines~\ref{line:kvget-beg}--\ref{line:kvget-end}), and the results are returned
to the client.

\subsection{``Disorderly'' Shopping Cart Service}

\begin{figure}[t]
\begin{scriptsize}
\begin{lstlisting}
module DisorderlyCart
  include CartProtocol

  def state
    table :cart_action, ['session', 'item', 'action', 'reqid']
    table :action_cnt, ['session', 'item', 'action'], ['cnt']
    scratch :status, ['server', 'client', 'session', 'item'],
                     ['cnt']
  end

  declare
  def do_action
    cart_action <= action_msg.map do |c| (*\label{line:dis_action_msg_beg}*)
      [c.session, c.item, c.action, c.reqid] 
    end (*\label{line:dis_action_msg_end}*)
    action_cnt <= cart_action.group(  (*\label{line:dis_count_reqid_beg}*)
      [cart_action.session, cart_action.item, cart_action.action],
      count(cart_action.reqid))  (*\label{line:dis_count_reqid_end}*)
  end

  declare
  def do_checkout
    del_items = action_cnt.map{|a| a.item if a.action == "Del"} (*\label{line:dis_checkout_no_item_beg}*)
    status <= join([action_cnt, checkout_msg]).map do |a, c|
      if a.action == "Add" and not del_items.include? a.item  (*\label{line:dis_checkout_negation}*)
        [c.client, c.server, a.session, a.item, a.cnt] (*\label{line:dis_status_add}*)
      end
    end (*\label{line:dis_checkout_no_item_end}*)

    status <= join([action_cnt, action_cnt, (*\label{line:dis_checkout_item_beg}*)
                    checkout_msg]).map do |a1, a2, c|
      if a1.session == a2.session and a1.item == a2.item and
         a1.session == c.session and
         a1.action == "A" and a2.action == "D"
        [c.client, c.server, c.session, a1.item, a1.cnt - a2.cnt] (*\label{line:dis_status_diff}*)
      end
    end (*\label{line:dis_checkout_item_end}*)

   response_msg <~ status.group( (*\label{line:dis_checkout_response_beg}*)
      [status.client, status.server, status.session],
      accum(status.cnt.times.map{status.item})) (*\label{line:dis_checkout_response_end}*)
  end
end
\end{lstlisting}
\vspace{-10pt}
\caption{Disorderly cart implementation.}
\label{fig:dis-cart}
\end{scriptsize}
\vspace{-2pt}
\end{figure}

Figure~\ref{fig:dis-cart} shows an alternative shopping cart implementation, in
which updates are monotonically accumulated in a set, and summed up only at
checkout.  Lines~\ref{line:dis_action_msg_beg}--\ref{line:dis_action_msg_end}
insert client updates into the persistent table \texttt{cart\_action}.
Lines~\ref{line:dis_count_reqid_beg}--\ref{line:dis_count_reqid_end} define
\texttt{action\_cnt} as an aggregate over \texttt{cart\_action}, in the style of
an SQL \texttt{GROUP BY} statement: for each item associated with a cart, we
separately count the number of times it was added and the number of times it was
deleted.
%Lines 5--16 define the collection \texttt{status} as the union of the session,
%item and action fields resulting from a 3-way join between the
%\texttt{checkout} message and two copies of \texttt{action\_cnt}---one
%corresponding to additions and one to deletions -- and the set of
%\texttt{action\_cnt}  tuples corresponding to items for which there is no
%deletion in a given session.
Lines~\ref{line:dis_checkout_no_item_beg}--\ref{line:dis_checkout_no_item_end}
ensure that when a \texttt{checkout\_msg} tuple arrives, \texttt{status}
contains a record for every added item for which there was no corresponding
deletion in the session.
Lines~\ref{line:dis_checkout_item_beg}--\ref{line:dis_checkout_item_end}
additionally define \texttt{status} as the 3-way join of the
\texttt{checkout\_msg} message and two copies of \texttt{action\_cnt}---one
corresponding to additions and one to deletions.  Thus, for each item,
\texttt{status} contains its final quantity: the difference between the number
of additions and deletions (line~\ref{line:dis_status_diff}), or simply the
number of additions if there are no deletions (line~\ref{line:dis_status_add}).
Upon the appearance of a \texttt{checkout\_msg}, the replica returns a
\texttt{response\_msg} to the client containing the final quantity
(lines~\ref{line:dis_checkout_response_beg}--\ref{line:dis_checkout_response_end}).
Because the CartClient expects the cart to be returned as an array of items on
checkout, we use the \texttt{accum} aggregate function to nest the set of
items into an array.

%Because the disorderly implementation does not use a separate storage system,
%it includes its own logic for state replication
%(lines~\ref{line:dis_join_action_msg_member}--\ref{line:dis_rep_end}).
%Line~\ref{line:dis_multicast} ensures that only the replica contacted by the
%client performs a multicast, by probing the table \texttt{member} to
%see if the \texttt{action\_msg} originated inside the quorum.

\subsection{Analysis}

\begin{figure}[t]
\centering
\includegraphics[width=1.1\linewidth]{fig/destructive_kvs.pdf}
\vspace{-10pt}
\caption{Visualization of the destructive cart program.}
\label{fig:pdg-destructive-kvs-analysis}
\vspace{-2pt}
\end{figure}

\begin{figure}[t]
\centering
\includegraphics[width=0.8\linewidth]{fig/disorderly_base.pdf}
\vspace{-10pt}
\caption{Visualization of the core logic for the disorderly cart.}
\label{fig:pdg-disorderly-base}
\vspace{-2pt}
\end{figure}

\begin{figure}[t]
\centering
\includegraphics[width=1\linewidth]{fig/disorderly_complete.pdf}
\vspace{-10pt}
\caption{Visualization of the complete disorderly cart program.}
\label{fig:pdg-disorderly-analysis}
\vspace{-2pt}
\end{figure}


%%Intuitively, the ``destructive'' cart based on
%%array mutation should be non-monotonic due to the transience of its state; this should make it sensitive to the order
%%of its inputs.
Figure~\ref{fig:pdg-destructive-kvs-analysis} presents the analysis of the 
``destructive'' shopping cart variant.  Note that
because all dependencies are analyzed, collections defined in mixins
but not referenced in the code sample
(e.g., \texttt{pipe\_chan}, \texttt{member}) also appear in the graph.
Although there is no syntactic
non-monotonicity in 
Figure~\ref{fig:dest-cart}, the underlying key-value store
uses the non-monotonic \texttt{$<$-} operator to model updateable state.
Thus, while the details of the implementation are encapsulated by the key-value
store's abstract interface, its points of order resurface in the full-program analysis.
Figure~\ref{fig:pdg-destructive-kvs-analysis}
indicates that there are
points of order between \texttt{action\_msg}, \texttt{member}, and the temporal cluster.
%%This means that the arrival 
%%timing and ordering of both client updates (via \texttt{iaction}) and
%%server replication (via the meta-edge in the temporal cycle from 
%%\texttt{kvput} to itself) may affect the
%%end results.  
This figure also tells the (sad!) story of how we could ensure consistency of
the destructive cart implementation: introduce coordination between client and
server---and between the chosen server and all its replicas---for {\em every
  client action or kvput update}.  The programmer can achieve this
coordination by supplying a ``reliable'' implementation of multicast that awaits
acknowledgements from all replicas before reporting completion: this
fine-grained coordination is akin to ``eager replication''~\cite{dangers}.
Unfortunately, it would incur the latency of a round of messages per server per
client update, decrease throughput, and reduce availability in the face of
replica failures.

%One coordination technique would require the client to wait for all replicas to
%successfully apply an update before sending a subsequent update.  This would
%involve changing the key-value store to inherit from a reliable delivery
%superclass (implemented in 6 LOC).
%%We can easily achieve coordination without modifying the cart implementation
%%by redefining the key-value store to extend
%%a reliable or quorum-based delivery module instead of the best-effort module
%%that the basic key-value store extends, and require acknowledgement from a server or a %%quorum of servers, respectively.
%%The simplest (and least performant) approach is to require unanimous quorum,
%%approximating ``eager replication''~\cite{dangers} via two-phase commit.
%A programmer could resolve this point of order by adapting the key-value store
%to require reliable delivery to all replicas before acknowledging a client
%update, by making the key-value store inherit from a superclass providing a
%reliable delivery abstraction (6 LOC).
%Such ``eager replication''~\cite{dangers} would incur a cost of a round of
%messages per server per client update, and substantially decrease system
%throughput.  However, it is possible to achieve consistency with far less
%coordination, by writing the shopping cart in a disorderly, rather than
%destructive fashion.
%\wrm{The following is confusing because people might not have this temptation.
%Maybe rephrase to ``a need for synchronization arises, because deletions and
%insertions do not commute''} Because we only care about the set of elements
%contained in the value array and not its order, we might be tempted to argue
%that the shopping cart application is eventually consistent when
%asynchronously updated, and forego the synchronization.  Unfortunately, such
%informal reasoning can hide serious bugs: consider what happens if a delete
%update is received before the addition it was intended to cancel.

% A programmer could resolve this point of order by adapting the key-value store
% to inherit from a reliable delivery superclass (implemented in 6 LOC), requiring 
% acknowledgements from replicas before acknowledging a client update.
% Such ``eager replication''~\cite{dangers} would incur a cost of a round of messages
% per server per client update, and substantially decrease system throughput.  
% \wrm{i don't quite see the point of presenting this straw-man scenario to people.  i think we should just explain that coordination is intuitively needed because deletions don't commute with insertions.}
% \paa{personally I am inclined to keep it.  neil?}
Because we only care about the \emph{set} of elements contained in the value array
and not its order, we might be tempted to argue that 
the shopping cart application is eventually consistent
when asynchronously updated and forego the coordination logic.  Unfortunately,
such informal reasoning can hide serious bugs. For example, consider what would happen if a delete action for an item arrived at some replica
before any addition of that item: the delete would be ignored, leading to
inconsistencies between replicas.

A happier story emerges from our analysis of the disorderly cart service.
Figure~\ref{fig:pdg-disorderly-base} shows a visualization of the core logic of
the disorderly cart module presented in Figure~\ref{fig:dis-cart}.  This program
is not complete: its inputs and outputs are channels rather than interfaces, so
the dataflow from source to sink is not completed.  To complete this program, we
must mixin code that connects input and output interfaces to
\texttt{action\_msg}, \texttt{checkout\_msg}, and \texttt{response\_msg}, as the
CartClient does (Figure~\ref{fig:cart-client}).  Note that the disorderly cart
has points of order on all paths but there are no cycles.

Figure~\ref{fig:pdg-disorderly-analysis} shows the analysis for a complete
implementation that mixes in both the client code and logic to replicate the
\texttt{cart\_action} table via best-effort multicast (see
Figure~\ref{fig:complete-dis-cart} in Appendix~\ref{app:network-code} for the
corresponding source code).  Note that communication (via \texttt{action\_msg})
between client and server---and among server replicas---crosses no points of
order, so all the communication related to shopping actions converges to the
same final state without coordination.  However, there are points of order upon
the appearance of \texttt{checkout\_msg} messages, which must be joined with an
\texttt{action\_cnt} aggregate over the set of updates.  Additionally, using the
\texttt{accum} aggregate adds a point of order to the end of the dataflow,
between \texttt{status} and \texttt{response\_msg}.  Although the accumulation
of shopping actions is monotonic, summarization of the cart state requires us to
ensure that there will be no further cart actions.

Comparing Figure~\ref{fig:pdg-destructive-kvs-analysis} and
Figure~\ref{fig:pdg-disorderly-analysis}, we can see that the disorderly cart
requires less coordination than the destructive cart: to ensure that the
response to the client is deterministic and consistently replicated, we need to
coordinate once per {\em session} (at checkout), rather than once per shopping
action.  This is analogous to the desired behavior in practice~\cite{quicksand}.
% =======
%
% indicates that no points of order are crossed when a client sends an
% \texttt{action\_msg} to a replica, and when replicas exchange
% \texttt{action\_msg} messages during replication.  Thus, all replicas will
% eventually have the same contents in their \texttt{cart\_action} collections,
% even in the absence of coordination.  There is, however, a point of order at
% checkout, when a \texttt{checkout\_msg} message is joined with an aggregate
% over the set of updates.  While the accumulation of \texttt{cart\_action} is
% monotonic, an aggregate cannot be returned until \texttt{cart\_action} is
% complete.
% %requires us to assume (or prove) that there will be no further updates.
% %Consider a checkout message and a final update message racing from a client to
% %one of the replicas: the order in which they arrive will surely affect the
% %contents of the response message.  
% We need only coordinate once per session to ensure that the response to the
% client is deterministic.
% >>>>>>> .r5792

\subsection{Discussion}
Strictly monotonic programs are rare in practice, so adding some amount of
coordination is often required to ensure consistency. In this running example we
studied two candidate implementations of a simple distributed application with
the aid of our program analysis. Both programs have points of order, but the
analysis tool helped us reason about their relative coordination costs.
Deciding that the disorderly approach is ``better'' required us to apply domain
knowledge: checkout is a coarser-grained coordination point than cart actions
and their replication.

By providing the programmer with a set of abstractions that are predominantly
order-independent, Bloom encourages a style of programming that minimizes
coordination requirements. But as we see in the destructive cart program, it is
nonetheless possible to use Bloom to write code in an imperative,
order-sensitive style. Our analysis tools provide assistance in this regard.
Given a particular implementation with points of order, Bloom's dataflow
analysis can help a developer iteratively refine their program---either to
``push back'' the points to as late as possible in the dataflow, as we did in
this example, or to ``localize'' points of order by moving them to locations in
the program's dataflow where the coordination can be implemented on individual
nodes without communication.

\section{Case Study: Vector Clocks and Causal Delivery}
\label{sec:causal}
Understanding the causal relationship between events in a distributed system has
many applications.

A causal delivery protocol ensures that messages are delivered in an order that
is consistent with the ``happens before'' relation between
events~\cite{Lamport1978}. Providing a causal order over events has been
identified as a useful building block for loosely consistent distributed
applications~\cite{Lloyd2011}.

In this section, we first show how \lang can be used to build vector
clocks~\cite{Fidge1988,Mattern1989}, a common tool for tracking causal
relationships in a distributed system. We then use \lang to implement a
classical algorithm for point-to-point causal delivery~\cite{Schiper1989}. The
implementation of both protocols in \lang is concise and readable; perhaps more
significantly, both protocols are monotonic. Since this agrees with the
intuitive notion that both protocols make ``progress'' over time, this gives us
more confidence in the correctness of our designs.

\subsection{Vector clocks}
\begin{figure}[t]
\begin{scriptsize}
\begin{lstlisting}
class VectorClock
  include Bud

  state do
    lmap :my_vc
    lmap :next_vc
    interface input, :in_msg, [:addr, :payload, :clock]
    interface input, :out_msg, [:addr, :payload]
    interface output, :out_msg_vc, [:addr, :payload, :clock]
  end

  bootstrap do
    my_vc <= [ {ip_port => Bud::MaxLattice.new(0)} ]
  end

  bloom do
    next_vc <= my_vc
    next_vc <= out_msg { {ip_port => my_vc.at(ip_port) + 1} } (*\label{line:vc-out-increment}*)
    next_vc <= in_msg  { {ip_port => my_vc.at(ip_port) + 1} } (*\label{line:vc-in-increment}*)
    next_vc <= in_msg  {|m| m.clock} (*\label{line:vc-in-merge}*)
    my_vc <+ next_vc

    out_msg_vc <= out_msg {|m| [m.addr, m.payload, next_vc]} (*\label{line:vc-out-stamp}*)
  end
end
\end{lstlisting}
\end{scriptsize}
\caption{Vector clocks in \lang.}
\label{fig:vector-clock-src}
\end{figure}

In a distributed system with $n$ nodes, a \emph{vector clock} is an array of $n$
logical clock values. Each node keeps a local vector clock and updates it as it
processes events; the node's vector clock reflects how up-to-date its local
state is with respect to the other nodes in the system. Vector clocks can be
used to assign a partial order over the events in the system that agrees with
the causal relationship between events. A vector clock can be implemented with
three simple rules that are followed when sending and receiving events:
\begin{enumerate}
\item
  Each node $n$ initializes its vector clock to $\{n \rightarrow 0\}$.
\item
  Before sending a message, $n$ increments the value in the vector clock
  associated with its own 
\end{enumerate}

Figure~\ref{fig:vector-clock-src} contains a complete \lang implementation of
vector clocks. A vector clock is represented as a map lattice that associates
node IDs with \texttt{lmax} values; each \texttt{lmax} value represents the
logical clock of a single node (which can only increase over
time). \texttt{ip\_port} returns the IP address and port number of the current
\lang instance; we use that as a node ID. Incoming messages are represented as
tuples in the \texttt{in\_msg} collection. Outbound messages begin as tuples in
\texttt{out\_msg}; outbound messages that have been stamped with the local clock
value appear in \texttt{out\_msg\_vc}.

The key vector clock logic appears in lines \ref{line:vc-out-increment},
\ref{line:vc-in-increment}, and \ref{line:vc-in-merge}---each \lang statement is
a direct translation of the three rules for updating a vector clock described
above. Line~\ref{line:vc-out-stamp} stamps outgoing messages with the updated
vector clock value.

Note that Figure~\ref{fig:vector-clock-src} is a monotonic \lang program. Hence,
the statements can be executed in any order.

\subsection{Causal delivery}

\begin{figure}[t]
\begin{scriptsize}
\begin{lstlisting}
module CausalDelivery
  state do
    channel :chn, [:@dst, :src, :ident] => [:payload, :clock, :ord_buf]

    # Local vector clock: map from node_id => lmax
    lmap :my_vc
    lmap :next_vc

    # Our knowledge of the vector clocks at other nodes:
    # map from node_id => {map from node_id => lmax}
    lmap :ord_buf

    # Received messages that haven't yet been delivered
    table :recv_buf, chn.schema
    scratch :buf_chosen, recv_buf.schema
  end

  bootstrap do
    my_vc <= [ {ip_port => Bud::MaxLattice.new(0)} ]
  end

  bloom :update_vc do
    my_vc <+ next_vc
    next_vc <= my_vc

    # On outgoing messages:
    next_vc <= pipe_in { {ip_port => my_vc.at(ip_port) + 1} }
    # On incoming messages:
    next_vc <= buf_chosen { {ip_port => my_vc.at(ip_port) + 1} }
    next_vc <= buf_chosen {|m| m.clock}
  end

  bloom :outbound_msg do
    chn <~ pipe_in {|p| [p.dst, p.src, p.ident, p.payload, next_vc, ord_buf]}
    ord_buf <+ pipe_in {|p| {p.dst => next_vc} }
  end

  bloom :inbound_msg do
    recv_buf <= chn
    buf_chosen <= recv_buf {|m| m.ord_buf.at(ip_port, Bud::MapLattice).lt_eq(my_vc).when_true { m } }
    recv_buf <- buf_chosen

    pipe_out <= buf_chosen {|m| [m.dst, m.src, m.ident, m.payload]}
    ord_buf <+ buf_chosen {|m| m.ord_buf}
  end
end
\end{lstlisting}
\end{scriptsize}
\caption{Point-to-point causal delivery in \lang.}
\label{fig:causal-delivery-src}
\end{figure}

\section{Related Work}

\subsection{Concurrency control}
The importance of commutative operations has long been recognized by work in the
distributed systems, data management, and groupware communities.

\subsection{Non-monotonicity in deductive databases}
Adding non-monotonic operators (e.g., aggregation and negation) to Datalog
increases the expressiveness of the language but introduces significant
complexities: care must be taken to ensure that the resulting language has a
semantics that is well-defined, intuitive to the user, and amenable to efficient
evaluation. A straightforward approach is to disallow recursion through
aggregation or negation, which admits only the class of so-called ``stratified
programs''~\cite{Apt1988}. Many attempts have been made to assign a semantics to
larger classes of programs (e.g.,~\cite{Gelfond1988,Ross1990,VanGelder1991}).

The observation that many uses of aggregation and negation have a ``monotonic''
flavor has been made before.

% CRDTs
% Work on semantics-aware concurrency control
% Operational transformations
% Ross & Sagiv on lattices
% Kostler et al. on differential fixpoint + subsumption

\section{Discussion and Future Work}
\label{sec:discussion}

\lang simplifies the design of loosely consistent distributed systems by
enabling \emph{composability}. Rather than reasoning about the consistency of an
entire application, the programmer can instead ensure that individual lattices
satisfy \emph{local} correctness properties (e.g., commutativity, associativity,
and idempotence). The CALM analysis then verifies that when these modules are
composed to form an application, the complete program satisfies the desired
consistency properties.

Nevertheless, designing a correct lattice can still be difficult. To address
this, we plan to develop tools to give programmers more confidence in the
correctness of lattice implementations. For example, we plan to build a test
data generation framework that can efficiently cover the space of possible
inputs to lattice merge functions. We also plan to explore a restricted DSL for
writing lattice methods, which would make formal verification of correctness an
easier task.

Each lattice includes $\bot$, a distinguished ``smallest element.'' In many
programs there is also a notion of a ``largest element'': a value that, once
reached, will never be exceeded. \texttt{lbool} has a largest value
(\texttt{true}), as does the \texttt{lcart} lattice presented in
Section~\ref{sec:monotone-checkout}. We are investigating how our analysis
techniques could be improved by introducing more direct support for $\top$
values into \lang. For example, a lattice that has reached $\top$ is immutable,
and hence any function can safely be applied to it, whether monotone or not.

By allowing a more general notion of monotonicity, \lang significantly increases
the number of programs that can be shown to be confluent. However, many
distributed protocols are not intended to be confluent---rather, they exhibit
\emph{controlled non-determinism}, in which timing conditions affect the choice
of one among several acceptable outcomes. Vector clocks and causal delivery are
both examples of this kind of behavior. Although \lang is still useful---the
local correctness properties of lattices help programmers to reason about the
behavior of individual program values---we are also investigating\ldots

% Support a notion of sealing?

\section{Conclusion}
\label{sec:conclusion}
And thus, we conclude.
\section*{Acknowledgments}
We would like to thank Peter Bailis, Ali Ghodsi, David Maier, and Matei Zaharia
for their helpful feedback on this paper.  This work was supported by the
Natural Sciences and Engineering Research Council of Canada.


\balance
\begin{small}
\bibliographystyle{abbrv}
\bibliography{vldb}
\end{small}

% \newpage
% \begin{appendix}
% \section{Proof of Lemma 1}
\begin{proof}
%First, we prove an isomorphism between stable models, and finite prefixes of stable models.  Scan a stable model of a program timestamp by timestamp.  
We first present an algorithm for computing ultimate models, and argue that the algorithm computes exactly the stable models of the \lang program.  We then argue this algorithm can be run on our operational formalism, show how operational traces correspond with prefixes of stable models.

Any \lang program without asynchronous rules is a $\text{Datalog}_1S$ program, and the algorithm given in~\cite{tdd} computes an ultimate model in polynomial space\footnote{The class of {\em multi-seperable}~\cite{tdd-poly} \lang rules, which comprises all \lang programs $P$ with guarded asynchrony and persisted EDB, and their coordinations $\textsc{Coord}(P)$, can be executed in polynomial time.} in the size of the input.  The algorithm evalutes the program for $2^G + e$ consecutive timesteps, where $G$ is the number of instantiations of the non-temporal attributes of the program rules, using all combinations of constants in the Herbrand Universe, and $e$ is the maximum timestamp of any EDB fact.  At each step, the algorithm updates information on observed periodicities of facts.  When the algorithm terminates, any fact with a periodicity of 1 is regarded as part of the ultimate model.

For asynchronous rules, the natural distributed analog of the algorithm above simultaneously executes one instance for each node \dedalus{n}, using values of $G$ and $e$ computed from $E_{\text{\dedalus{{\scriptsize n}}}}$.  Each instance has its own local clock, which intuitively corresponds to the timestamp attribute in the model-theoretic semantics.  Nodes communicates over channels with arbitrary delay and message re-ordering.  When another node \dedalus{m} derives a fact at \dedalus{n}, it encloses its local clock value, \dedalus{t}; \dedalus{n} must consider this fact until time \dedalus{t}, in the style of Lamport Clocks.  Note that this behavior is implied by the model-theoretic semantics---remote asynchronous rules state that their deductions are visible at the destination at a time later than the body temporal attribute at the source.  Further, note that Lamport Clocks only introduce the constraint that if message $a$ ``happens before'' $b$, in other words $a$ directly or transitively causes $b$ to be sent, then $T(a) < T(b)$.  If $a$ and $b$ are concurrent, there is some execution where $T(a) \geq T(b)$.

When \dedalus{n} processes a received message, the number of constants available to \dedalus{n} may increase, and thus a node's $G$ may increase to $G'$.  Furthermore, the node may need to execute over this new fact for $2^{G'}$ additional timesteps.  If only finitely many messages are sent, this algorithm requires polynomial space.  In the case that infinitely many messages are sent, we only need to process each message $2^{G'}$ times: the maximum period of any fact is $2^{G'}$, as every incoming fact needs to have a chance (in some execution) to join with any deduction, at any time during its period, with which it is ``concurrent''.  Keeping track of the number of times we have seen each fact also requires polynomial space.  When the algorithm is done running for $2^{G'}$ steps, it pauses, waiting for new network input that it has not seen enough times.  If all nodes are paused and no outstanding messages exist, then the collection of all period 1 facts at all instances of the algorithm comprises an ultimate model.

We claim that the algorithm can generate every ultimate model---every message has the opportunity to join with another concurrent message or its transitive consequents at any point during their period, and has the opportunity to join with a causally related message during the range of times allowed by the model-theoretic asynchronous constraint (identical to the Lamport Clock condition used in the algorithm).

Note that we can execute this algorithm on our operational formalism.  Evaluating a single timestamp of a \lang program corresponds to the evaluation of a Datalog program, which is a polynomial time computation, and the Turing Machines can also maintain the necessary state about periods and message counts.
%2) Intuitively, the operational model is based on n Turing Machines, one per value of node(), which independently step sequentially through time and communicate via channels with
%non-deterministic delay.  At each timestep t they run a datalog fixpoint computation that evaluates P on ``projection(E_n, t)'' (notation needed); this takes polynomial
%time~\cite{immerman}.  At the end of this fixpoint there are three sets of relevant facts: local, synchronous facts that have timestep t+1 and become part of ``projection(E_n,
%t)'', local asynchronous facts whose timestep is chosen non-deterministically to be greater than t and become part of later timesteps, and remote asynchronous facts.  The
%timestamps in this third class of facts are chosen non-deterministically ``at the receiver'' to model delay, in a way that observes traditional causality
%restrictions~\cite{lamportclocks}.
%3) Any \lang program without  async rules is a Datalog_{1S} program, and the above intuition is captured by the algorithm given in~\cite{}, computing an ultimate model in
%polynomial space in the size of the input.  In the presence of asynchronous rules, this formalize needs to be expanded to account for the asynchronous advancement of time through
%\dedalus{successor} at each node.  The PSPACE guarantees of~\cite{} are not shown to hold for such programs, but in Appendix Foo we show that the following Lemma holds for all
%\lang programs under this model
\end{proof}

\section{Proof of Lemma 2}
\begin{proof}
We begin by assuming that \dedalus{node} contains the identifiers of each of the $n$ nodes.  Since the atemporal fragment of \lang is FO[LFP], we can represent a polynomial-time bounded Turing Machine using only atemporal rules in \lang~\cite{immerman-ptime}.  In addition to normal operations, the Turing Machine can place items into a queue---\cite{dedalus} shows how to model queues in \lang---or send messages to other nodes---modeled by an asynchronous communication rule with \dedalus{queue} in the head.  A node persists the contents of the tape across time if the queue is empty, using a rule like \dedalus{tape(\dbar{X})@next <- tape(\dbar{X}), !queue(\dbar{\_});}.  If the queue is non-empty, the computation skips a timestamp (leaving \dedalus{tape} empty), and then atomically copies the contents of \dedalus{queue} to \dedalus{tape}.  The ultimate model of this program is exactly the final contents of the tape on every node if the computation halts.  Otherwise, the program's ultimate model is empty: \dedalus{tape} facts only exist every other timestamp, and for any Turing Machine predicate \dedalus{r} we can create \dedalus{r'}, and create a mutual recursive cycle to ensure neither \dedalus{r} nor \dedalus{r'} contains facts at every timestamp:

\begin{Dedalus}
r(\dbar{X})@next <- r'(\dbar{X});
r'(\dbar{X})@next <- r(\dbar{X});
\end{Dedalus}

We can play a somewhat similar trick for \dedalus{queue} by having local messages alternate between going into \dedalus{queue} and \dedalus{queue'}.  Thus, no local queue message will be part of the ultimate model.  Remote messages will still go into \dedalus{queue}: this still leaves the case that the exact same message repeatedly arrives at a node at every timestamp forever, by chance.  We can dispense of this case by assuming the channels interconnecting the Turing Machines forbid it.
\end{proof}

\section{Proof of Lemma 3}
\begin{proof}
Our proof proceeds via construction of a two counter machine in \lang, inspired by the construction in~\cite{undecidable-datalog}. We briefly review two counter machines.  A two counter machine's state is captured in the state of its two counters (natural numbers), and in its control state.  A two counter machine has a transition function: $\delta: \Sigma \times \{=, >\} \times \{=, >\} \rightarrow \Sigma \times \{inc, dec\} \times \{inc, dec\}$

$\Sigma$ is a finite set of states (for simplicity we assume a finite subset of the natural numbers), $=$ indicates a counter is equal to zero, and $>$ indicates a counter is greater than zero.  $inc$ and $dec$ indicate that a counter should be incremented, or decremented respectively.

We represent the state of a two counter machine using the \linebreak \dedalus{cnfg(T,S,C1,C2)} relation, where \dedalus{T} represents ``time'' (note this is not the same as the timestamp attribute), \dedalus{S} is the state (in $\Sigma$), and \dedalus{C1} and \dedalus{C2} are the values of the two counters.  In order to support $inc$ and $dec$, we would like to make use of the \dedalus{succ} relation.  However, \lang conventions forbid the use of this infinite relation outside of the timestamp attribute.  Thus, we posit the \dedalus{fin\_succ(X,Y)} EDB relation, which represents a finite prefix of the successor relation.  Since it is EDB, its contents may be arbitrary.  If \dedalus{fin\_succ} is malformed, then the machine's execution may be incorrect.  In particular, our model of the machine may accept an input, whereas the actual machine would not have accepted that input.  We illustrate how to constrain the contents of \dedalus{fin\_succ} below:

\begin{Dedalus}
malformed() <- fin_succ(_,0);
malformed() <- fin_succ(X,Y), fin_succ(X,Z), Y != Z;
malformed() <- fin_succ(Y,X), fin_succ(Z,X), Y != Z;
malformed() <- fin_succ(X,Y), X >= Y;
\end{Dedalus}

For a given EDB, the two counter machine either halts in the accepting state or halts in a non-accepting state.  It cannot run forever since the EDB (in particular, the \dedalus{fin\_succ} relation) is finite.

We construct a \lang program that nondeterministically decides to either run the machine on the input provided (and for the length of \dedalus{fin\_succ} provided, or declare that the machine will never accept without running it.  If the machine ever accepts some input, then we would like this to induce two different ultimate models -- one generated by a trace where we run the machine and it accepts, and one generated by a trace where we decide to not run the machine, and thus we implicitly reject.  We describe the program below. 

Initially, we nondeterministically decide whether to run the machine or not, by sending two messages (0 and 1) to a remote node (\dedalus{decider}).  If both message arrive simultaneously, then the decider responds to run the machine.  Otherwise, the decider responds to declare failure:

%\jmh{should we use a hashmark for constants?  I would say no.}
\begin{Dedalus}
//send two messages to the decider
message(#D, 0)@async <- decider(D);
message(#D, 1)@async <- decider(D);

//decider responds to computer
run_machine(#computer)@async <- message(0),
                                message(1);
declare_failure(#computer)@async <- message(0),
                                    !message(1);
declare_failure(#computer)@async <- !message(0),
                                    message(1);
\end{Dedalus}

Each mapping in the transition function is expressed by a \lang rule with \dedalus{!malformed()} and \dedalus{!declare\_failure()} in its body.  For example, the rule $\delta(3, > =) = (7, inc, dec)$ would be represented as:

\begin{Dedalus}
cnfg(S,7,D1,D2) <- cnfg(T,3,C1,C2), C1 > 0, C2 == 0,
                   fin_suc(T, S), fin_succ(C1, D1),
                   fin_succ(D2, C2), !malformed(),
                   !declare_failure();
\end{Dedalus}

We declare success or failure as follows:

\begin{Dedalus}
reject() <- !accept();
accept() <- cnfg(20,_,_); //20 is the accepting state
accept()@next <- accept();
\end{Dedalus}

If we choose to declare failure, or the machine halts in a non-accepting state, whether it is due to incompleteness or malformedness \dedalus{fin\_succ}, or actual halting, then the ultimate model will contain \dedalus{reject}.  If the machine halts in an accepting state, then the ultimate model will contain \dedalus{accept}.  Thus, if we can decide confluence of this program, then we can decide whether a two-counter machine halts on any input.
\end{proof}

% \end{appendix}

\end{document}
