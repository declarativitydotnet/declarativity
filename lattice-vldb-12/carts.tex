\section{Case Study: Shopping Carts}
\label{sec:carts}
\begin{figure}[t]
\includegraphics[width=\linewidth]{fig/cart_arch.pdf}
\label{fig:cart-system}
\caption{Architecture of a simple shopping cart system.}
\end{figure}

The next two sections contain case studies that demonstrate how \lang can be
used to build practical distributed programs. Our goal is to demonstrate that
non-trivial distributed protocols can be expressed using lattices and monotone
functions. If written in Bloom, these programs would be considered non-monotonic
because they contain aggregation. Presenting monotonic implementations in \lang
is equivalent to showing that the use of aggregation in these protocols is
``safe'' and produces consistent results without needing additional
coordination.

In the first case study, we consider a simple e-commerce system in which clients
interact with a shopping cart service by adding and removing items over the
course of a shopping session (Figure~\ref{fig:cart-system}). The shopping cart
service is replicated to improve fault tolerance; client requests can be routed
to any of the cart replicas. Eventually, a client submits a ``checkout''
operation, at which point the cumulative effect of their shopping session should
be summarized and returned to the client. In a practical system, the result
of the checkout operation might be presented to the client for confirmation or
submitted to a payment processor to complete the e-commerce transaction. This
case study is based on the cart system from Alvaro et al.~\cite{Alvaro2011},
which was in turn inspired by the discussion of replicated shopping carts in the
Dynamo paper~\cite{DeCandia2007}.

Alvaro et al.\ discuss two different designs: a ``disorderly'' version in which
the cart state is represented as a set of operations (allowing monotonic
accumulation of adds and removes) and a ``destructive'' version in which the
cart state is managed by a key-value store, which requires a non-monotonic
update on each cart action. In both designs, Alvaro et al.\ argue that the
checkout operation is non-monotonic because it requires aggregating over all
previous operations applied to the cart.

\subsection{Monotonic Checkout}
\begin{figure}[t]
\begin{scriptsize}

\begin{lstlisting}
module CartProtocol
  state do
    channel :action_msg,
      [:@server, :session, :reqid] => [:item, :action]
    channel :checkout_msg,
      [:@server, :client, :session, :reqid] => [:lbound]
    channel :response_msg,
      [:@client, :server, :session] => [:items]
  end
end

module MonotoneReplica
  include CartProtocol

  state do
    lmap :sessions
  end

  bloom do
    sessions <= action_msg {|c|
      cart = CartLattice.new({c.reqid => [ACTION_OP,
                                          c.item, c.action]})
      { c.session => cart }
    }
    sessions <= checkout_msg {|c|
      cart = CartLattice.new({c.reqid => [CHECKOUT_OP,
                                          c.lbound, c.client]})
      { c.session => cart }
    }

    response_msg <~ sessions {|s_id, c| (*\label{line:cart-response-start}*)
      c.is_complete.when_true {
        [c.checkout_addr, ip_port, s_id, c.summary]
      }
    } (*\label{line:cart-response-end}*)
  end
end
\end{lstlisting}
\end{scriptsize}
\caption{\lang program for a server replica that allows monotonic checkout.}
\label{fig:monotone-cart}
\end{figure}

Alvaro et al.\ argue that processing a checkout request is non-monotonic because
it requires aggregating over an asynchronously computed data set---in general,
coordination might be required to ensure that all inputs have been received
before the checkout response can be returned to the client. However, observe
that the client knows exactly which add/remove operations should be reflected in
the result of the checkout. If that information can be propagated to the cart
service, any server replica can decide if it has enough information to process
the checkout operation without needing additional coordination. This design is
monotonic: once a checkout response is produced, it will never change or be
retracted. Hence, our goal is to translate this design into a monotonic \lang
program.

% This can be done by assigning IDs to each message sent by the client. Each
% client has a session ID; within a session, operation IDs are assigned in
% increasing numeric order without gaps. Hence, if the client sends a ``lower
% bound'' message ID along with the checkout message, any replica of the cart
% service can independently ensure that it only produces a response message once
% it has received all the operations in the ID range indicated by the client. This
% essentially requires a threshold test over the operation IDs received by a given
% replica, which can easily be implemented using \lang.

We represent the state of a server replica using an \texttt{lmap} lattice that
associates session IDs with \texttt{lcart} lattice elements. \texttt{lcart} is a
custom lattice that represents the state of a single shopping cart. An
\texttt{lcart} contains a set of client operations. Each operation has a unique
ID; operation IDs are assigned by the client in increasing numeric order without
gaps. An \texttt{lcart} contains two kinds of operations: \emph{actions} and
\emph{checkouts}. An action describes the addition or removal of an item from
the cart. An \texttt{lcart} contains at most one checkout operation---the
checkout specifies the smallest operation ID that must be reflected in the
result of the checkout, along with the address where the checkout response
should be sent.

An \texttt{lcart} is \emph{complete} if it contains a checkout operation along
with all the actions in the ID range identified by the checkout. Hence, testing
whether an \texttt{lcart} is complete is a monotone function: it is equivalent
to testing whether an accumulating set has crossed a threshold. Hence, if a
server replica determines that it has a complete cart, it can send a response to
the client without risking inconsistency. Note that the client might receive
multiple responses, but they will all reflect the same cart contents.

Figure~\ref{fig:monotone-cart} contains the server code for this design (we omit
the client code for the sake of brevity). Note that the statement that produces
a response to the client
(lines~\ref{line:cart-response-start}--\ref{line:cart-response-end}) is
contingent on having a complete cart. The \texttt{summary} function returns the
summarized state of the cart---if \texttt{summary} is called before the cart is
complete, an exception is raised.

% Note that because each replica determines when the cart is ``complete''
% independently, multiple response messages may be produced. However, they will
% all be consistent, because ...

% \subsection{Performance Study}
% \begin{itemize}
% \item
%   goal: demonstrate that removing coordination from a distributed protocol can
%   significantly reduce its running time
% \item
%   benchmark: destructive cart w/ coordination on each action vs.\ destructive
%   cart in \lang without coordination, disorderly cart with coordination on
%   checkout vs.\ disorderly cart with monotonic checkout
% \end{itemize}
