\section{Case Study: Key-Value Store}
\label{sec:kvs}
The next two sections contain case studies that show how \lang can be used to
build practical distributed programs. Both case studies are monotonic programs:
that is, both programs consist of monotone functions applied to lattices. This
ensures that the values computed by the programs ``grow'' over time (according
to some notion of ``growth'') --- these examples show the various kinds of
forward progress that can be encoded using \lang.

In the first case study, we show that a complex program can be \emph{composed}
via a series of monotone mappings between simple lattices. This demonstrates
that \lang's builtin lattices are useful and results in a very concise
implementation. Moreover, it gives us confidence in the correctness of our
implementation, because much of the program's complexity is handled by the
behavior of the builtin lattices, which are likely to be correct.

\subsection{Basic Architecture}

\subsection{Object Versioning}

\subsubsection{Vector Clocks}

\subsection{Quorum Replication}
