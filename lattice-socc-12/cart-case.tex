\section{Case Study: Shopping Carts}
\label{sec:carts}

\begin{figure}[t]
\centering
\includegraphics[width=\linewidth]{fig/cart_arch.pdf}
\caption{Shopping cart system architecture.}
\label{fig:cart-system-arch}
\end{figure}

In the previous section, we showed how a complete, consistent distributed
program can be composed via monotonic mappings between simple lattice types. In
this section, we describe how \lang overcomes the ``type dilemma'' of Bloom. In
prior work, we introduced a case study in Bloom that seemed to require
coordination because of the use of distributed aggregation~\cite{Alvaro2011}. By
using custom lattice types, the \lang CALM analysis can verify that our revised
design is eventually consistent without need for coordination.

Figure~\ref{fig:cart-system-arch} depicts a simple e-commerce system in which
clients interact with a shopping cart service by adding and removing items over
the course of a shopping session. The cart service is replicated to improve
fault tolerance; client requests can be routed to any server
replica. Eventually, a client submits a ``checkout'' operation, at which point
the cumulative effect of their shopping session should be summarized and
returned to the client. In a practical system, the result of the checkout
operation might be presented to the client for confirmation or submitted to a
payment processor to complete the e-commerce transaction. This case study is
based on the cart system from Alvaro et al.~\cite{Alvaro2011}, which was in turn
inspired by the discussion of replicated shopping carts in the Dynamo
paper~\cite{DeCandia2007}.

\subsection{Monotonic Checkout}
\label{sec:monotone-checkout}

\begin{figure}[t]
\begin{scriptsize}
\begin{lstlisting}
module CartProtocol
  state do
    channel :action_msg,
      [:@server, :session, :op_id] => [:item, :cnt]
    channel :checkout_msg,
      [:@server, :session, :op_id] => [:lbound, :addr]
    channel :response_msg,
      [:@client, :session] => [:summary]
  end
end

module MonotoneReplica
  include CartProtocol

  state { lmap :sessions }

  bloom do
    sessions <= action_msg do |m|
      c = LCart.new({m.op_id => [ACTION, m.item, m.cnt]}) (*\label{line:cart-action-op}*)
      { m.session => c }
    end
    sessions <= checkout_msg do |m|
      c = LCart.new({m.op_id => [CHECKOUT, m.lbound, m.addr]}) (*\label{line:cart-checkout-op}*)
      { m.session => c }
    end

    response_msg <~ sessions do |session, cart| (*\label{line:cart-response-start}*)
      cart.is_complete.when_true {
        [cart.checkout_addr, session, cart.summary]
      }
    end (*\label{line:cart-response-end}*)
  end
end
\end{lstlisting}
\end{scriptsize}
\caption{Cart server replica in \lang that supports a monotonic
  (coordination-free) checkout operation.}
\label{fig:monotone-cart}
\end{figure}

Alvaro et al.\ argue that processing a checkout request is non-monotonic because
it requires aggregation over an asynchronously computed data set---in general,
coordination might be required to ensure that all inputs have been received
before the checkout response can be returned to the client. However, observe
that the client knows exactly which add and remove operations should be
reflected in the result of the checkout. If that information can be propagated
to the cart service, any server replica can decide if it has enough information
to safely process the checkout operation without needing additional
coordination. This design is monotonic: once a checkout response is produced, it
will never change or be retracted. Our goal is to translate this design into a
monotonic \lang program.

% This can be done by assigning IDs to each message sent by the client. Each
% client has a session ID; within a session, operation IDs are assigned in
% increasing numeric order without gaps. Hence, if the client sends a ``lower
% bound'' message ID along with the checkout message, any replica of the cart
% service can independently ensure that it only produces a response message once
% it has received all the operations in the ID range indicated by the client. This
% essentially requires a threshold test over the operation IDs received by a given
% replica, which can easily be implemented using \lang.

Figure~\ref{fig:monotone-cart} contains the server code for this design (we omit
the client code for the sake of brevity). Communication with the client occurs
via the channels declared in the \texttt{CartProtocol} module. Each server
replica stores an \texttt{lmap} lattice that associates session IDs with
\texttt{lcart} lattice elements. An \texttt{lcart} is a custom lattice that
represents the state of a single shopping cart. An \texttt{lcart} contains a set
of client operations. Each operation has a unique ID; operation IDs are assigned
by the client in increasing numeric order without gaps. An \texttt{lcart}
contains two kinds of operations, \emph{actions} and \emph{checkouts}. An action
describes the addition or removal of $k$ copies of an item from the cart. An
\texttt{lcart} contains at most one checkout operation---the checkout specifies
the smallest operation ID that must be reflected in the result of the checkout,
along with the address where the checkout response should be sent. The
\texttt{lcart} merge function takes the union of the operations in both input
carts (operation IDs ensure idempotence). In Figure~\ref{fig:monotone-cart},
lines~\ref{line:cart-action-op} and \ref{line:cart-checkout-op} construct
\texttt{lcart} elements that contain a single action or checkout operation,
respectively. These singleton carts are then merged with the previous
\texttt{lcart} associated with the client's session, if any.

An \texttt{lcart} is \emph{complete} if it contains a checkout operation as well
as all the actions in the ID range identified by the checkout. Hence, testing
whether an \texttt{lcart} is complete is a monotone function: it is similar to
testing whether an accumulating set has crossed a threshold. Hence, if any
server replica determines that it has a complete cart, it can send a response to
the client without risking inconsistency.\footnote{Without coordination, the
  client might receive multiple responses but they will all reflect the same
  cart contents.} Because this program contains only monotonic operations,
CALM analysis can verify that this design is consistent without requiring
additional coordination.

Note that the statement that produces a response to the client
(lines~\ref{line:cart-response-start}--\ref{line:cart-response-end}) is
contingent on having a complete cart. The monotone \texttt{summary} method
returns a summary of the actions in the cart---an exception is raised if
\texttt{summary} is called on an incomplete cart. Similarly, attempting to
construct an ``illegal'' \texttt{lcart} instance (e.g., an \texttt{lcart} that
contains multiple checkout operations or actions that are outside the ID range
specified by the checkout) also produces an exception, since this likely
indicates a logic error in the program.
% Implementing the \texttt{lcart} lattice required 58 lines of Ruby using the
% lattice API described in Section~\ref{sec:lattice-api}.

\subsection{Discussion}
This design is possible because a single client has complete knowledge of the
shopping actions in its associated session. Hence, there is no need for
additional distributed coordination---the server replicas accumulate knowledge
but do not contribute new information themselves. If multiple clients could
operate on a single shopping cart, some form of coordination between clients
would be needed to ensure a consistent checkout result.

Note that the threshold test for completeness is a crucial part of this
design. Until a cart is complete, its content changes in a ``non-monotonic''
fashion as items are added and removed. However, these non-monotonic changes are
hidden inside the \texttt{lcart} type and are not directly visible to
clients. Clients can only observe the cart's state once the cart is complete; at
that point, the cart state is immutable and hence will not change in a
non-monotonic fashion. \lang enables \texttt{lcart} to expose a limited ``safe''
interface and to hide transient non-monotonic changes from direct visibility.

% Note that because each replica determines when the cart is ``complete''
% independently, multiple response messages may be produced. However, they will
% all be consistent, because ...

% \subsection{Performance Study}
% \begin{itemize}
% \item
%   goal: demonstrate that removing coordination from a distributed protocol can
%   significantly reduce its running time
% \item
%   benchmark: destructive cart w/ coordination on each action vs.\ destructive
%   cart in \lang without coordination, disorderly cart with coordination on
%   checkout vs.\ disorderly cart with monotonic checkout
% \end{itemize}
