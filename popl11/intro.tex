\section{Introduction}
\rcs{I think we've been thinking about Dedalus too narrowly.  It's not just a language for distributed programming atop some model of asynchrony.  Instead, it's an efficiently executable temporal logic that can be used to model time-varying systems.  Distributed systems are just a case study.  (And even there, we can't agree on a set of sane temporal semantics, even within our group...)}
\nrc{Another possible approach to the first para: observe that declarative logic langs are seeing increased use in a wide variety of domains, from games to modular robotics, because of succinctness, formal analysis, etc. Then argue that distributed systems are a particularly promising domain for declarative logic.}

Logic languages succinctly and precisely express programs in a way
that is amenable to automatic verification and formal specification
of practical systems\rcs{cite some static analysis wins} and
non-trivial algorithms\rcs{cite lamport, and a few other things.}.
Traditionally, such languages operate over static databases
(i.e. Datalog, SQL), or admit constructs that cannot be efficiently
evaluated (i.e. TLA).\rcs{we could more explicit about which primitives they admit...}

Recent work has extended languages based upon first order logic,
adding primitives such as state mutation and communication.  Such
languages are increasingly being used in a wide variety of fields,
including games, modular robotics, and distributed systems.  However,
these language extensions depart from their purely logical
foundations; their semantics are defined in terms of concepts such as
queuing disciplines, network communication models, node failures, and
transactional updates.  Although such operational constructs have
little impact on the length or style of programs written in these
languages, they lead to semantic ambiguities that undermine program
readability and analysis techniques~\rcs{cite some overlog dig(s)}.

\rcs{the old intro mentioned least fixpoint operators, which seemed out of place, so I cut it.  It should probably go somewhere else...}

\rcs{kill following para
Distributed logic languages \wrm{cite stuff here, to make it clear that what we mean by a ``distributed logic language'' are the following instances of prior work} promise to significantly raise the level of
abstraction at which distributed systems are currently implemented.
Commonly, this class of languages allow
programmers to specify systems as a set of invariants over local state,
%%\wrm{okay, maybe say ``commonly'' or ``one popular design choice'' or something before %%``these languages allow programs to specify systems as invariants over local state and rules 
%%that describe how state changes and moves across a network,'' because i think we all would 
%%consider something to be a distributed logic language that defines invariants over global or 
%%remote state}
%% \wrm{local isn't important, it's just a design choice of dedalus}
and rules that describe how state changes and moves across a network. 
%%\wrm{again this detail isn't important, it's just a design choice of dedalus}.  
This approach leads to succinct, executable specifications whose faithfulness to the original pseudocode may be visually verified.
\nrc{Which ``original pseudocode'' are you referring to? Seems like a much crisper pitch for distributed logic langs should be possible.}\rcs{This would be a good place to say that we're a bridge between easily-verifiable languages like TLA, and efficiently executable languages like datalog}
\paa{fair enough, hack away}
Moreover, many 
distributed logic languages are extensions of first-order logic enhanced with
a {\em least fixpoint operator}, which lends itself to powerful
formal verification techniques~\cite{wang, wang2}.
RCS: resume not-killed text} 

We conjecture that logic languages are a
better fit for the specification of concurrent and distributed systems
than traditional imperative approaches.  Imperative programs are built
up from sequences of explicitly ordered, reliable operations; modern
distributed hardware provides neither ordering nor reliability.
Imperative techniques address this mismatch by explicitly partitioning
computation across threads of execution, then executing the threads
concurrently across multiple CPUs or machines.  We argue that such
programs are overspecified in two important ways.  First, explicitly
partitioning computation hardcodes the unit of concurrency to be
exposed to the hardware.  Second, imperative programs commonly store
state in data structures such as lists and trees; each such structure
must be carefully manipulated in accordance with some protocol, which
is (at best) specified using an API that, in turn, imposes
additional ordering and partitioning constraints upon the program.

\rcs{more dead text:  applications of a sequencing primitive and whose state is commonly stored in data structures like lists and trees, ordering is implicit and ubiquitous, 
while concurrency is achieved by duplicating the sequential control structures (i.e., stacks)
or by partitioning computation across CPUs,
and allowing the programs to run in an interleaved or truly concurrent fashion, respectively.}

%%\nrc{I don't follow what ``duplicating the sequential control structures'' means.}\rcs{Two 
%%options: something about duplicating the stack, or ``... by explicitly partitioning computation 
%%across CPUs, and allowing them to ...''}
%\wrm{what about concurrent data structures?  maybe we cut the ``concurrency is achieved...'' 
%and replace it with ``the programmer must take specific steps to achieve concurrency''}.   
%whoops -- thought you said "data structures" instead of "control structures"

In contrast, declarative language constructs, built up from the
implication primitive and abstracting program state into relations,
are in general order-independent.  This allows the programmer to avoid
specifying or even reasoning about the ordering of data elements or
operations, except when such details are fundamentally relevant to the
program at hand.  In turn, this increases the amount of concurrency
exposed at runtime, especially for ``embarassingly parallel''
operations.

\rcs{neil: Better?}
%high degree
\nrc{This seems a confusing claim: the programmer must also ``reason about ordering of data elements and operations'' in an imperative language}
%but in return allowing a high degree of concurrency and often ``embarrassing'' 
%parallelism~\cite{podskey}. 
\nrc{People usually speak of ``embarrassing parallelism'' as a property of a particular problem/algorithm, not a language.} 

Instead of reasoning about the possible interleavings of sequential
instructions or the serial order of message arrival as one must when writing threaded or 
event-driven code, respectively, a distributed logic language allows the programmer to focus
on the admissible, atomic transformations over state that can in general be carried out in
parallel.  This inversion of concerns -- implicit concurrency and explicit order, instead of
implicit order and explicit concurrency -- can both ease the programmer's cognitive burden as
concurrency and parallelism increase, and help to highlight the few (but critical) 
portions of a distributed algorithm for which order is essential to correctness.

\paa{briefly deal with the shortcomings of previous logic languages.  text from
outline: ``they get squirrelly when you introduce events, which you NEED to do so to implement constructs that express mutable state (eg state machines, update) or preserve or reestablish order in the face of asynchrony (eg sequences, queues).  dedalus admits such ordered constructs (via reification of time) without sacrificing powerful static analysis capabilities -- in fact, it admits new analysis that are unique to the asynchronous domain''}

\rcs{Kill this para, redundant with new text above:
In order to support mutable state (for example, to express state transitions) and 
sequencing constructs (to restore the order lost by asynchronous communication),
many previous distributed logic languages introduced a small set of operational
features that are familiar in the domain, such as events, counters or sequences, 
and implicit queues.  
These operational aspects often lead to semantic ambiguitities when combined
with the underlying declarative constructs: while the resulting languages {\em looked}
like Datalog, it was usually not possible to leverage the rich set of static analyses 
available in a ``pure'' logic language.}

In this paper we reconsider declarative logic languages for distributed
systems from a purely model-theoretic perspective. We introduce a declarative
language called \lang
\begin{comment}
\footnote{\small \lang is intended as a precursor
  language for \textbf{Bloom}, a high-level language for programming
  distributed systems that will replace Overlog in the \textbf{BOOM}
  project~\cite{boom-eurosys}.  As such, it is derived from the
  character Stephen Dedalus in James Joyce's \emph{Ulysses}, whose
  dense and precise chapters precede those of the novel's hero,
  Leopold Bloom.  The character Dedalus, in turn, was partly derived
  from Daedalus, the greatest of the Greek engineers and father of
  Icarus.  Unlike Overlog, which flew too close to the sun, Dedalus
  remains firmly grounded.  } 
\end{comment}
that enables the specification of rich distributed systems concepts
without recourse to operational constructs.  Concretely, \lang can
succinctly model characteristics of the runtime environment such as
queuing disciplines, network communication models and atomicity in
terms of simple logical specifications.  This allows restricted
variants of Dedalus to provide semantics that are unambiguous, but
otherwise similar to existing operational approaches.\rcs{the
  restricted variants are lamport clocks (causal) and tuple at a time
  (serializable)} We describe a number of such variants below;
unrestricted Dedalus allows programs to send messages to the past; addition of a
Lamport clock restricts us to execution traces that obey causality,
and ensuring atomic message processing reveals connections to
traditional database concurrency control and existing compiler
optimization techniques.

\wrm{assignment of timestamps is operational.  i think the point we want to make is that we elegantly minimize the operational semantics}
\lang is simply
Datalog \rcs{It's not a subset!}, enhanced with negation, aggregate functions, choice,
and a successor relation.  
\lang provides a model-theoretic 
foundation
for the two key features of distributed systems: mutable state and
asynchronous processing and communication.  We show how these features
are captured in \lang using logical time. 

\paa{three features: default unordered, order analysis and flex, reconfigurability}.
\lang makes three principal contributions to the implementation and analysis
of distributed systems.  First, like other logic languages, it presents an
alternative to the traditional sequential programming model,  encouraging
programmers to compose expressions that are order-independent and hence admit
both parallel execution and rewrite-based optimizations.
Second, \lang enables static analyses that can detect where in a distributed system
order constructs may be necessary to overcome indeterminacy caused by arbitrary
communication delay.  \lang is expressive enough to implement a rich set of ordering
constructs, including events, sequences, queues,
distributed clocks and barriers\rcs{I added barriers.  Does it subsume sequence/queue?},
which may be employed to restore necessary orders to such systems.
Because the explicit inclusion of time allows a purely logical interpretation of 
these constructs, their incorporation in \lang does not force us to sacrifice 
the many powerful analyses from the Datalog literature.
In combination, these features lead to ... \paa{how to sell reconfigurability?}\rcs{I don't buy/understand the reconf stuff yet, so I'm not touching the rest of this para.}

Throughout, we demonstrate the expressivity
and practical utility of our work with specific examples, including a
number of building-block routines from classical distributed
computing, such as sequences, queues, and distributed clocks.  
%%We also discuss the correspondence between \lang and our
%%prior work implementing full-featured distributed services in somewhat
%%more operational Datalog variants~\cite{boom-eurosys,Loo2009-CACM}.
As a running example, this paper will discuss various approaches to 
implementing and analyzing a distributed ``shopping cart'' application.

\begin{comment}
\paa{ possibly reuse the below in the 1st para as neil suggests (but as ras recommends
against}
\rcs{ i say kill it; we now talk about this stuff more concisely above.  I didn't mention non-state mutating applications in the intro, so we might want to put those cites somewhere.}

In recent years, there has been a resurgence of interest in Datalog as
the foundation for applied, domain-specific languages in a wide
variety of areas, including networking~\cite{Loo2009-CACM},
distributed systems~\cite{Belaramani:2009,Chu:2007}, natural language
processing~\cite{Eisner:2004}, robotics~\cite{Ashley-Rollman:2007},
compiler analysis~\cite{Lam:2005}, security~\cite{sd3,Li:2003,Zhou:2009}
and computer games~\cite{White:2007}.  The resulting languages have
been promoted for their compact and natural representations of tasks
in their respective domains, in many cases leading to code that is
orders of magnitude shorter than equivalent imperative programs.
Another stated advantage of these languages is the ability to
directly capture intuitive specifications of protocols and programs as
executable code.

While most of these efforts were intended to be ``declarative''
languages, many chose to extend Datalog with operational features
natural to their application domain.  These operational aspects,
though familiar, limit the ability of the language designers to
leverage the rich literature on Datalog: program checks like safety
and stratifiability, and optimizations like magic sets and
materialized recursive view maintenance.  In addition, combining
operational and declarative constructs often
leads to semantic ambiguities.  This is of particular interest to us
in the context of networking and other distributed systems, both
because we have considerable practical experience with these
languages~\cite{boom-eurosys,Loo2009-CACM}, and because others have
examined the semantic ambiguities of these languages in some
depth~\cite{Mao2009,navarro}.\rcs{Furthermore, a wide range of network protocols have already been proven correct in higher level logic languages, such as TLA.  Unfortunately, while these languages are ``executable'' in some sense, they include constructs that cannot be automatically mapped into efficient implementations.}


Sometimes order -- ordering of data elements, as well as ordered computational steps -- 
is required by an algorithm, and in a logic language this order must be explicitly specified and 
managed.  Previous languages 

\paa{story: admitting order (that is, time) into logic is tricky, and retaining a purely logical interpretation for how the order is established and respected is downright hard.  as you'll see in the next section, nearly all of the useful idioms for DS construction are problematic/ambiguous precisely because they describe composite operations across timesteps, without the vocabulary to make clear what things co-occur atomically.}
\end{comment}

% Traditional database systems are based on declarative query languages that
% specify transformations as dataflows over an updatable store.
% \jmh{Not usually thought of as dataflows.  Rel Alg is kinda like dataflow but declarative Rel Calc isn't. Stick with Calc/Declarative as your reference.}
%   Such query
% languages are either not expressive enough to capture common programming
% constructs \wrm{like what?}, or are at best awkward to use in this fashion.
% \wrm{todo: transition that explains Datalog's birth from these languages... I
% don't know enough to write it} The family of logic-based database languages, of
% which Datalog is the progenitor, represent expressive programming languages
% that produce similar dataflow representations.  Datalog is purely deductive: a
% program specifies the rules by which the derived relations are populated based
% on a static database, which is never updated.  Recent programming language
% research has explored the use of Datalog-based languages for expressing
% distributed systems.  Because the state of any complex system evolves with its
% execution, these efforts were forced to extend the Datalog model by admitting
% updates, additions and deletions of the EDB.  Unfortunately, these previous
% attempts were plagued with ambiguities about how and when state changes occur
% and become visible, putting a heavy burden on the programmer to ensure even
% simple properties, such as atomicity of updates over time.
% 
% In contrast to reasoning about state change procdurally, \lang observes
% that this concept is intuitively expressed as invariants over {\em time}.  In
% this work, we present a formal model of Datalog augmented with time extensions.
% By reifying time as data an introducing it into the logic, \lang eliminates
% previous ambiguities, ensures atomicity of updates and makes it possible to
% express system invariants that can guarantee liveness properties, a key
% challenge in building distributed systems.
