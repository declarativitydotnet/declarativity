\appendix


\subsection{Proof of Lemma~\ref{lemma:temp-strat-uniq}}

\begin{proof}
%
{\bf Case 1:} $P$ consists of only deductive rules.  In this case, $P$'s
deductive reduction is $P$ itself.  We know $P$ is syntactically stratifiable,
thus it has a unique minimal model.

{\bf Case 2:} $P$ consists of both deductive and inductive rules.  Assume that
$P$ does not have a unique minimal model.  This implies that $P$ is not
syntactically stratifiable.  Thus, there must exist some cycle through at least
one predicate $q$ involving negation.
%%or aggregation.  
Furthermore, this cycle must involve an inductive rule, as $P$ is temporally
stratified.

Since the time suffix in the head of an inductive rule is strictly greater than
the time suffix of its body, no atom may depend negatively on itself---it may
only depend negatively on atoms in the previous timestep.  Thus, $P'$ is
modularly stratified over time, using the definition of modular stratification
according to Ross et al.~\cite{modular}.  This guarantees a unique minimal model
achievable via standard bottom-up fixpoint execution.
%
%does not have a unique minimal model.  This implies that $P$ is not
%syntactically stratifiable, thus there must exist some cycle through at least
%one predicate $q$ involving a negation or aggregation edge in $P$'s predicate
%dependency graph, and furthermore this cycle must include at least one
%inductive rule.  Since an inductive rule has a time suffix $S := N+1$, where
%$N$ is the timestamp of its body, and $P$'s deductive reduction is
%syntactically stratifiable, we know that the aggregate or negation of $q$ must
%always occur in a strictly earlier or later timestamp than that of the
%positive $q$ atom.  Since the timestep in the cycle increases monotonically
%with each iteration, $q$ will never, in practice, depend on a negation or
%aggregate of itself.  Thus, $P$ is locally stratifiable, and by Lemma XXX
%above, $P$ has a unique minimal model.  This contradicts our assumption that
%$P$ does not have a unique minimal model.  Thus, $P$ has a unique minimal
%model.
%
\end{proof}

\section{Proof of Observation 1}

\begin{proof}
%%%\wrm{sightly ???}
A \slang program admits only instantaneous and inductive rules, which derive
new tuples at the same time as their ground tuples, or in the immediate next
timestep.  Thus, the set of tuples true at time $T$ is completely determined by
any tuples true at time $T-1$, and any EDB facts true at time $T$.  Observe
that the integer value of the timestep does not influence the derivation.

If the instance is quiescent at $T$, then given ${\bf A}$, the set of atoms
with timestamp $T-1$, and the EDB at $T$, the program entails ${\bf
A}$ at timestamp $T$.  Thus in the absence of EDB facts at $T+1$, it entails
${\bf A}$ at $T+1$.
%
\end{proof}



\section{Proof of Lemma~\ref{lemma:temp-safety}}





\begin{proof}
%
Assume the program is temporally unsafe.  That is, there exists no time $T$
such that $\forall U >= T$, the set of all atoms with timestamp $U$ is
equivalent modulo time to the set of all atoms with timestamp $T-1$.  Let $E$
be the maximum timestamp of any fact in the EDB.

Observe that every rule $r$ of kind 3 may only entail a finite number
of facts---as the EDB is finite---and thus may entail no facts at a
timestamp greater than some maximum timestamp $V_r <= E+1 \in
\mathbb{Z}$.  Since a \slang program has a finite set of rules we know
$\exists V \in \mathbb{Z} : \forall r: V >= V_r$, and thus $V <= E+1$.

We now consider times $T$ such that $T > E+1$.  By the above argument, no rules
of kind 3 entail any facts with a timestamp greater than $E+1$.  Recall that
no EDB atoms are true at any timestamp greater than $E$.  Thus, any facts with
timestamp greater than $E+1$ are entailed by rules of kind 1 or 2.

Consider the set of equivalence classes modulo time of all possible atoms, {\bf
A}, given the Herbrand universe.  We know {\bf A} is finite, as the Herbrand
Universe is finite.  Therefore, if the program is temporally unsafe, then {\bf B}, the
set of atoms entailed by the program, both contains and excludes
infinitely many members of at least one equivalence class in {\bf A} (i.e.,
something ``infinitely oscillates in time'' between being true and false).
Since the program has finitely many rules, at least one rule must entail
infinitely many atoms (from at least one of the equivalence classes from {\bf
A}). Thus, it is easy to see that there must be a cycle that contains some predicate $P$ and $\lnot P$.

We show there exists such a cycle containing only rules of kind 1, which
implies that the program is temporally unstratifiable.  In order for such a
cycle to exist, $P$ must transitively depend on $\lnot P$, and $\lnot P$ must
transitively depend on $P$.  Thus, the program contains a rule $J_1$ with
$\lnot P$ in its body, and some predicate $R$ in its head, as well as a rule
$J_2$ that is transitively dependent on $R$, with $P$ in its head.

{\bf Case 1: }$P \neq R$.  In this case, $J_1$ must be of kind 1, as for any $Q
\neq P$, a rule of kind 2 with $P$ in the head may not directly entail $Q$
given $P$.  $J_2$ must also be of kind 1---if it is of kind 2, then it
necessarily contains $P$ in its body, so it cannot entail $P$ unless $P$ is
entailed by some other rule.  If $J_2$ contains $R$ in its body, then the
program is syntactically unstratifiable.  But if $J_2$ does not contain $R$ in
its body, then it contains some predicate $S$ transitively entailed by $R$;
wlog the body contains $R$.  Thus, the program is syntactically unstratifiable.

{\bf Case 2: }$P = R$.  In this case, $J_1$ and $J_2$ are the same rule: $P
\leftarrow \lnot P$.  Thus, the program is syntactically unstratifiable.

Thus, the program is temporally unstratifiable, which contradicts our
assumption.
%
\end{proof}
