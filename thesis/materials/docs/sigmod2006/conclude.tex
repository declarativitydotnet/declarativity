

\section{Additional Related Work}
We mentioned most of the related work in the context of our discussion
above.  Here, we briefly mention some other related efforts.

% Optimization and execution of recursive queries is a rich area of
% research; Ramakrishnan and Ullman's survey~\cite{ramakrishnan93survey}
% provides a variety of references to the literature.  

We are not alone in our renewed enthusiasm for applications of
recursive queries.  There are other contemporary examples from outside the
traditional database ``market'', including software
analysis~\cite{whaley04}, trust management~\cite{Cassandra} and
diagnosis of distributed systems~\cite{discreteEventDatalog}.  Our
concept of {\em link-restricted} rules is similar in spirit to {\em
  d3log}~\cite{jimsuciu}, a query language based on Datalog proposed
for dynamic site discovery along web topologies.

Much research in the
parallel execution of recursive queries~\cite{cacace93survey} has
focused on high throughput within a cluster. In contrast, our
strategies and optimizations are geared towards bandwidth efficiency
and fast convergence in a distributed setting.  Instead of hash-based
partitioning schemes that assume full connectivity among nodes, we are
required to perform query execution only along physical network links
and deal with network changes during query execution.  There is also
previous empirical work on the performance of parallel pipelined
execution of recursive queries~\cite{pipelinedRecursive}. Our results
extend that work by providing new, provably correct pipelining
variants of semi-\naive evaluation.

In terms of distributed systems, the closest analog is the recent work
by Abiteboul \textit{et al.}~\cite{discreteEventDatalog}.  They adapt
the QSQ~\cite{qsr} technique to a distributed domain in order to
diagnose distributed systems.  An important limitation of their
approach is that they do not consider partitioning of relations across
sites as we do; they assume each relation is stored in its entirety in
one network location. Further, they assume full connectivity and do not
consider updates concurrent with query processing.


%There are proposals for instantiating networking protocols
%from high-level specifications from the formal methods community.
%For example,~\cite{maude} applies Maude, an executable specification
%language based on rewriting logic to the field of network security
%protocols.  Within the traditional networking, there is also renewed
%interest in high-level specification.  For example,
%Metarouting~\cite{metarouting} deals with ``routing algebras'' which
%specify routing protocol behavior globally.  We are investigating
%implementing Metarouting over \Sys and \Dlog. 

% Apart from declarative networks, there have been other recent proposals
% on interesting applications of Datalog and recursive queries outside of traditional data
% management domains. Abiteboul \textit{et~al.}~\
% \cite{discreteEventDatalog}  demonstrate how the diagnosis of distributed systems
% can be modeled using Datalog programs, and can benefit from optimization
% techniques such as Query-Sub-Query (QSQ)~\cite{qsr}, which is related to
% our use of the magic sets optimizations. It has also been shown that
% many program analysis can be expressed naturally and easily in logic
% programming languages~\cite{logicProgramAnalysis}. In a recent work, Whaley \textit{et~al.}~\ \cite{whaley04} demonstrated the advantages (in terms of programmability and
% performance) of having a declarative interface for pointer analysis of
% large programs.  

\section{Conclusion}
\label{sec:conclusion}
\label{sec:futurework}

Our goal in this paper was twofold: to provide a solid database
foundation for recent developments in declarative networking, and to
open a number of database research directions in the area.  We believe
that our contributions here are significant on both fronts.

We started with the concept of \emph{link-restricted rules}, which
capture syntactically in \Dlog the notion that query
messages are constrained to travel along direct links between nodes in
a network.   This in turn led to successive refinements of
semi-\naive evaluation that deal efficiently
with the asynchrony and delays intrinsic to a wide-area networking
environment.  We introduced techniques to incorporate updates
immediately during execution, capturing the reactive nature of
typical network protocols while offering meaningful semantic
guarantees.  We  also discussed a number of
query optimization techniques, and their applicability to the networking
domain.  Finally, we presented evaluation results from a distributed deployment
  involving 100 machines on the Emulab~\cite{emulab} network testbed, running prototypes of our optimization techniques implemented as
  modifications to the \Pitu system.  

%the
%adaptation of cost-based magic-sets~\cite{costMagic} for
%the distributed setting, 

%Declarative networking is a promising area with some interesting
%challenges for database researchers.  
Our ongoing research is proceeding in several directions. First, we are
exploring a complete query optimization architecture, as well as
specific techniques beyond those of
Section~\ref{sec:queryOpt}: additions to the cost-based
optimizations of Section~\ref{subsec:hybridRewrites}
including the possibility of using random walks driven by statistics
on graph expansion; adaptive query processing techniques to react to
network dynamism; and multi-query optimizations motivated by more
complex overlay networks. Second, we plan to incorporate
negation into our model and implementation~\cite{ullmanNegation},
which raises interesting challenges
for pipelining and dynamic data. Third, a key selling point of
declarative languages in the networking community is the promise of
static program checks for desirable network protocol properties;
% such as convergence and stability; 
we are considering techniques from the
Datalog literature in this regard (\eg ~\cite{krs}) and expect that
the particulars of link-restricted rules can be of use as well.
Finally, we intend to aggressively pursue these ideas in the context of
serious networking applications, \eg overlay networks like distributed hash
tables, application-level multicast protocols, and virtual private networks.

%Second, we will revisit some of \Dlog's restrictions, in particular
%link-restricted rules for specifying more complex networks. So far our programs
%have performed global recursive computations along network links; we
%wish to study more execution semantics in the presence of locally
%recursive rules. 

We have been pleased in this work to see that the enthusiasm in the
networking community for declarative languages can provide more than
just a well-motivated application area for recursive queries; it
appears to spark a host of new database research challenges in what
was considered a very mature area.  We are optimistic about the
potential for additional significant results in this domain, in terms
both of theoretical work and systems challenges.

%% Use the next sentence of Stonebraker's quote in the intro, which is:
%% ``If a market develops for [recursive query] technology, the
%% commercial DBMSs will implement the ideas in a heartbeat.''  Well,
%% maybe not Oracle, but maybe Cisco instead!


%However, even with stratified negation, determining the right pipelining strategy in the asynchronous
%  distributed setting is a challenge. 
%Just as we relax the need to
%  perform semi-\naive evaluations in iteration, we intend to explore the relaxation
%  of evaluating programs with negation a stratum at a time, where evaluation results can be
%  corrected over time with updates, leading to eventual consistency.

%\end{mylist}

