\section{Abstract}

Diff applications need structuring: complex structuring of data, location independence, integrity and authentication

Needn't necessarily location specific
secur

Building distributed systems is a complex and time consuming task. It may take more than a month to transform a pseudo-code design into implementation. Further, the process is error-prone and often loses the {\em proved} properties of the system. The goal of this project is to present a new approach to building distributed system that reduces the design and implementation time and effort while providing better understanding of the important properties of the system. 

Our system enables specifying a distributed systems in declarative language called {\em syslog}. The powerful yet generic semantics of syslog enable concise specification of most complex systems in few lines of code. For e.g. the normal-case operation of Practical Byzantine Fault Tolerance, a system considered reasonably complex by the community, can be expressed in around 10 lines. The specification consists of just the ideas which makes the specification easier to understand and amenable to modifications. Declarative approach ensures safety in presence of design errors and change of environments. Finally, the hope is that reproducible and understandable system specification will replace the pseudo code from the literature and lead to newer insights and innovations that were difficult previously. 

Our system builds on two key components: Security and Distributed Objects.

Security forms the critical component of any real distributed system. One of the key goals of this project is to add support for security primitives that can abstract away the precise encryption algorithm from the specification. Thus, once expressed in form of our primitive, the specification could be implemented using a variety of different algorithms depending on the operating environment. This will allow the same specification to be used under multiple different environments.

The second important innovation of our work are the semantics and mechanism for distributed objects. Distributed objects are like conventional object, except that parts of a distributed object can be scattered throughout the network. The application specification is oblivious to the location of such objects. Thus, the applications can use these distributed objects without worrying about their location and the implementation can transparently export parts of these graphs to optimize performance. These exporting policies can be tuned by some global macros. These distributed objects expose many possibilities of optimizing. A lazy export policy will export only the requisite portions of the graph to the receiver to miniminze bandwidth whereas a pro-active export policy will serialize and export the graph together to reduce latency and overall bandwidth consumption.

There is another important benefit of supporting such distributed objects in our system. Our system is based on P2 which is declarative query processing engine. As a result, the basic unit of information in our system is tuple. However, programs with tuples generally tend to be flat and tedious to write and comprehend. These distributed objects, represented in form of relational tuple graph, yields the addition benefit of adding structure to the program making them easier to understand without voilating the relational semantics of the P2 world.

We consider the specification of PBFT, Secure Aggregation, and S-BGP to illustrate the power of our declarative approach. 