\documentclass{acm_proc_article-sp-sigmod09}

%%\usepackage{amsthm}


\usepackage[usenames, dvipsnames]{color}
\usepackage{times}
%%\usepackage{url}
%%\usepackage{graphicx}
%%\usepackage{boxedminipage}
\usepackage{xspace}
\usepackage{textcomp}
\usepackage{wrapfig}
\usepackage{url}
%%\usepackage{verbatim}
%%\usepackage{latexsym}
\usepackage{amsmath, amssymb}
%%\usepackage{amsthm}

\usepackage{alltt}
\usepackage{appendix}

\usepackage{txfonts}
\newcommand{\Tau}{\mathcal{T}}

\newcommand{\jmh}[1]{{\textcolor{red}{#1 -- jmh}}}
\newcommand{\paa}[1]{{\textcolor{blue}{#1 -- paa}}}
\newcommand{\rcs}[1]{{\textcolor{green}{#1 -- rcs}}}
\newcommand{\nrc}[1]{{\textcolor{magenta}{#1 -- nrc}}}
\newcommand{\wrm}[1]{{\color{BurntOrange}{#1 -- wrm}}}
\newcommand{\smallurl}[1]{{\small \url{#1}}}

%dedalus environment for code
\newenvironment{Dedalus}{
\vspace{0.5em}\begin{minipage}{0.95\textwidth}%\linespread{1.3}
\begin{alltt}\fontsize{9pt}{9pt}\selectfont}
{\end{alltt}\end{minipage}\vspace{0.5em}}

\newcommand{\dedalus}[1]{\texttt{\fontsize{9pt}{9pt}\selectfont #1}}


\begin{document}

\conferenceinfo{ACM PODS}{'10 Indianapolis, IN, USA}
\title{Dedalus
%%\titlenote{\small
%%Dedalus is intended as a precursor language for \textbf{Bloom}, a high-level language for programming distributed systems that
%%will replace Overlog in the \textbf{BOOM} project~\cite{boom-techr}.  
%%As such, it is derived from the character Stephen Dedalus in James Joyce's \emph{Ulysses}, whose dense and precise chapters 
%%precede those of the novel's hero, Leopold Bloom.  The character Dedalus, in turn, was partly derived from Daedalus, the greatest
%%of the Greek engineers and father of Icarus.  Unlike Overlog, which flew too close to the sun, Dedalus remains firmly grounded.
%%} 
: Datalog in Space and Time} 
%%Format\titlenote{(Produces the permission block, copyright information and page numbering). For use with ACM\_PROC\_ARTICLE-SP.CLS V2.6SP. Supported by ACM.}}
%
% You need the command \numberofauthors to handle the "boxing"
% and alignment of the authors under the title, and to add
% a section for authors number 4 through n.

\numberofauthors{5}

\author{Peter Alvaro, Neil Conway, William R. Marczak, Joseph M. Hellerstein, David Maier}
%%\author{Neil Conway}

\maketitle

\begin{abstract}
We present \textbf{Dedalus}, a syntactic subset of Datalog~\cite{ullmanbook} with negation, aggregate functions and choice.  Dedalus extends Datalog with a
notion of time -- every Dedalus relation has an integer attribute that intuitively represents when a given tuple is true.  We propose a sugared syntax in which rule head annotations indicate whether their conclusions hold at the same time as their
premises, in the immediate next timestep, or at some unknown timestep.  
We introduce the notion of \emph{temporal stratification}, a syntactically-checkable property of Dedalus programs
similar to stratification, and show that the class of temporally stratifiable programs in Dedalus corresponds
to the class of modularly stratified~\cite{modular,ross-syntactic} programs in Datalog.  By admitting the explicit representation of time into the logic language,
Dedalus captures practical details like persistence, updatable state, and message delay and loss in a purely declarative fashion.
This allows us to express distributed systems whose state evolves over time in a high-level language with an unambiguous, 
model-theoretic semantics.
\end{abstract}



\section{\large \bf \lang}
\label{sec:foundation}

\subsection{Preliminary Definitions}

We assume an infinite universe $\mathcal{U}$.

A {\em relation schema} is a pair consisting of a relation name and its arity.  A {\em schema} $\mathcal{S}$ is a set of relation schemas.  As in \wrm{cite Immerman}, we assume the existence of an order.  Thus, all schemas contain the following relational schemas: \dedalus{lt} -- a relation of arity two that is true whenever the second argument is greater than the first (we will write this relation in infix form using the \dedalus{<} symbol) -- \dedalus{suc} -- a relation of arity two that defines a total order over $\mathcal{U}$ \wrm{different from succ}.

A {\em rule} over a schema $\mathcal{S}$ is a clause of the form:

\begin{Dedalus}
p(\(\bar{X_0}\)) <- b_1(\(\bar{X_1}\)), \ldots, b_n(\(\bar{X_n}\)), !c_1(\(\bar{Y_1}\)), \ldots, !c_m(\(\bar{Y_m}\)).
\end{Dedalus}

where \dedalus{h}, \dedalus{\(b_1\), \ldots, \(b_n\)}, \dedalus{\(c_1\), \ldots, \(c_m\)} are relations in $\mathcal{S}$, and $\bar{X_i}$ and $\bar{Y_i}$ denote a tuple (of the appropriate arity) consisting of variable symbols, constants from $\mathcal{U}$.  Note that we maintain the usual safety restrictions of Datalog rules: any variable symbol $V$ that appears in $\bar{Y_i}$ for some $1 \leq i \leq m$ must also appear in $\bar{X_j}$ for some $1 \leq j \leq n$, but only if $V$ appears in $\bar{X_0}$ or $V$ appears in $\bar{Y_k}$ for some $k \neq i$ -- i.e., variable symbols that only appear in a single negated atom and do not appear in the head need not also appear in a positive atom \wrm{cite ullman}.

\wrm{describe recursion, rule out recursion through negation, define EDB}

A {\em fact} over a relation schema of arity $n$ is a pair consisting of the relation name and an $n$-tuple $(c_1,\ldots,c_n)$, where each $c_i \in \mathcal{U}$.  An instance $\mathcal{I}$ over a schema $\mathcal{S}$ is a set of facts.

Given a schema $\mathcal{S}$, we use $\mathcal{S}^*$ to denote the extension of $\mathcal{S}$ obtained by adding two columns to each relation schema in $\mathcal{S}$, and adding three additional relation schemas to $\mathcal{S}$.  The first additional column, the {\em location specifier}, indicates the ``location'' of the tuple, and the second additional column, the {\em timestamp}, is a natural number representing a logical time.  \wrm{As in...} We call $\mathcal{S}^*$ a {\em spatio-temporal} schema.  The three relations we add are: \dedalus{time}, a unary relation equivalent to $\mathbb{N}$, \dedalus{succ}, a binary relation representing the natural successor relation over $\mathbb{N}$, and \dedalus{node}, a unary relation that represents all locations (the meaning of ``location'' will become clear later).

A {\em spatio-temporal} fact over a relation schema of arity $n$ is a pair consisting of the relation name and an $n+2$ tuple $(l,t,c_1,\ldots,c_n)$ where each $c_i \in \mathcal{U}$, $l \in \dedalus{node}$, and $t = 0$ (all spatio-temporal facts must be supplied with timestamp 0).

A {\em spatio-temporal rule} over a spatio-temporal schema $\mathcal{S}^*$ is a rule of one of the following three forms:

A {\em deductive} rule:
\begin{Dedalus}
p(L,T,\(\bar{X_0}\)) <- b_1(L,T,\(\bar{X_1}\)), \ldots, b_n(L,T,\(\bar{X_n}\)), !c_1(L,T,\(\bar{Y_1}\)), \ldots, !c_m(L,T,\(\bar{Y_m}\)).
\end{Dedalus}

An {\em inductive} rule:
\begin{Dedalus}
p(L,S,\(\bar{X_0}\)) <- b_1(L,T,\(\bar{X_1}\)), \ldots, b_n(L,T,\(\bar{X_n}\)), !c_1(L,T,\(\bar{Y_1}\)), \ldots, !c_m(L,T,\(\bar{Y_m}\)), succ(T,S).
\end{Dedalus}

An {\em asynchronous local} rule:
\begin{Dedalus}
p(L,S,\(\bar{X_0}\)) <- b_1(L,T,\(\bar{X_1}\)),\ldots,b_n(L,T,\(\bar{X_n}\)), !c_1(L,T,\(\bar{Y_1}\)), \ldots, !c_m(L,T,\(\bar{Y_m}\)), time(S), S > T, choice((L, T, \(\bar{B}\)),(S)).
\end{Dedalus}

An {\em asynchronous communication} rule:
\begin{Dedalus}
p(D,S,\(\bar{X_0}\)) <- b_1(L,T,\(\bar{X_1}\)),\ldots,b_n(L,T,\(\bar{X_n}\)), !c_1(L,T,\(\bar{Y_1}\)), \ldots, !c_m(L,T,\(\bar{Y_m}\)), time(S), S > T, choice((L, T, \(\bar{B}\)),(S)), node(D).
\end{Dedalus}

where all symbols are as defined before, $\bar{B}$ is a list of all of the distinct variable symbols in $\bar{X_1}, \ldots, \bar{X_n}, \bar{Y_1}, \ldots, \bar{Y_m}$; \dedalus{D} and \dedalus{L} are variable symbols that may also appear in $\bar{B}$, \dedalus{T} and \dedalus{S} are variable symbols that may not appear in $\bar{B}$, and \dedalus{choice} is the construct of Sacc\`{a} and Zaniolo~\cite{sacca-zaniolo}, which we will presently describe.

A \lang {\em program} is a set of spatio-temporal rules over some spatio-temporal schema $\mathcal{S}^*$.  Note that a given set of rules over a schema may give rise to many different \lang programs, depending on which type of spatio-temporal rule each rule is converted into, and depending on where or if the variable $D$ appears in the body in asynchronous communication rules.

A \lang {\em instance} is a program with a set of spatio-temporal facts specified for EDB relations.

\noindent
\textbf{Syntactic sugar for space-time in \lang:}
The restrictions on temporal and location attributes suggest a natural syntactic sugar to improve readability.  Given the unification requirements for location and temporal attributes in rule bodies, we can omit these attributes from predicates without risk of ambiguity.  

The three temporal classes of rules listed above can be distinguished by annotating inductive head predicates with \dedalus{@next}, and asynchronous head predicates with \dedalus{@async}; simple deductive rules have no head annotation. 
Communication rules must include the location attribute of their head predicate; by definition this is a different variable than that of the body predicates.
 Given these conventions, the presence of appropriate \dedalus{successor}, \dedalus{choice} and \dedalus{node} predicates in the body are implicit, and can be omitted.  The result is a syntax that reads like a simple temporal variant of Datalog.

Deductive:
\begin{Dedalus}
p(\(\bar{X_0}\)) <- b_1(\(\bar{X_1}\)), \ldots, b_n(\(\bar{X_n}\)), !c_1(\(\bar{Y_1}\)), \ldots, !c_m(\(\bar{Y_m}\)).
\end{Dedalus}

Inductive:
\begin{Dedalus}
p(\(\bar{X_0}\))@next <- b_1(\(\bar{X_1}\)), \ldots, b_n(\(\bar{X_n}\)), !c_1(\(\bar{Y_1}\)), \ldots, !c_m(\(\bar{Y_m}\)).
\end{Dedalus}

Asynchronous local:
\begin{Dedalus}
p(\(\bar{X_0}\))@async <- b_1(\(\bar{X_1}\)), \ldots, b_n(\(\bar{X_n}\)), !c_1(\(\bar{Y_1}\)), \ldots, !c_m(\(\bar{Y_m}\)).
\end{Dedalus}

Asynchronous communication:
\begin{Dedalus}
p(#D,\(\bar{X_0}\))@async <- b_1(#L,\(\bar{X_1}\)), \ldots, b_n(#L,\(\bar{X_n}\)), !c_1(#L,\(\bar{Y_1}\)), \ldots, !c_m(#L,\(\bar{Y_m}\)).
\end{Dedalus}

Asynchronous communication rules require the body location specifier to be explicit, because it may be unified with other body attributes.


%\vspace{1em}
%\noindent 
%\textbf{Predicate Dependence Graphs in \lang}: 
%A \lang program's {\em predicate dependency graph}~\cite{ullmanbook} PDG is 
%a directed graph that has one node per predicate, and an edge from \dedalus{q} to \dedalus{p} if predicate \dedalus{p} appears in the head of a rule with \dedalus{q} in its body.  If \dedalus{q} is negated, or a \dedalus{count} appears in the head of the rule, we mark the edge as negated.  If the rule is inductive or asynchronous, we annotate the edge with the rule type.  The PDG omits the \dedalus{time} and \dedalus{successor} predicates, and the nodes and edges for \dedalus{choice}.  We write $\dedalus{q} \to \dedalus{p}$ if there exists a path following forward edges from \dedalus{q} to \dedalus{p}.  We write $\dedalus{q} \nrightarrow \dedalus{p}$ if there exists a forward path from \dedalus{q} to \dedalus{p} that crosses a negated edge.  We write $\dedalus{q} \circleright \dedalus{p}$ if all forward paths from \dedalus{q} to \dedalus{p} traverse an asynchronous or inductive edge.  We write $\dedalus{q} \Diamondright \dedalus{p}$ if at least one forward path from \dedalus{q} to \dedalus{p} traverses an async edge. 



\subsection{Semantics}
One interpretation of a \lang instance is given by the possible worlds of the stable model semantics~\cite{stable-model}.  We do not review the stable model semantics here -- the only salient detail is the interaction of\dedalus{choice} with the stable model semantics.  \dedalus{choice} induces a separate stable model foreach possible sequence of choices of timestamps.  Note that without \dedalus{choice} -- i.e., without any asynchronous local or asynchronous communication rules -- a \lang program has a unique stable model, because it is locally stratified.

There are two potential problems with considering a stable model as the meaning of a \lang instance.  First, every program with at least one asynchronous rule has infinitely many stable models.  Not all of these stable models may be meaningfully different.  Second, a stable model of a \lang program may itself be infinite.  We address both concerns in our definition of an {\em ultimate model}.

\begin{example}
\label{ex:flipflop}
A \lang program with an infinite stable model.

\begin{Dedalus}
flipflop(Y,X)@next \(\leftarrow\) flipflop(X,Y);
flipflop(1,2)@1;
\end{Dedalus}

\dedalus{flipflop(1,2)} is true at all odd times, and \dedalus{flipflop(2,1)} is true at all even times.  Thus, \dedalus{flipflop(1,2)} and \dedalus{flipflop(2,1)} are each cyclic with period 2.
\end{example}

\wrm{explain the ultimate model in terms of defining an output schema, and everything that is ever asserted into that output schema is part of the output}.



\subsection{Operational Interpretation}
\label{sec:operational}

Our goal in defining an operational formalism is to demonstrate that our model-theoretic view of distributed systems corresponds to the real-world behaviors of such systems.

\wrm{describe what kind of system this models, how it would be executed}
\wrm{describe failure model -- all messages eventually delivered}

\section{Evaluation models}

Consider the \emph{successor} relation described above.  According to our intuitive interpretation, this relation models
the passage of time, in order to establish a temporal order among ground atoms.  More formally, we expect of a successor
relation $S$ that

$\forall A,B (successor(A, B) \rightarrow B > A) \land \forall A \exists B (successor(A, B))$

This implies that successor is infinite (as we'd expect time to be).  This is problematic because it leads to unsafe programs.

\newtheorem{example}{Example}
\begin{example}
Consider the program and EDB below.

\begin{Dedalus}
r1
p_pos(A, B)@next \(\leftarrow\)
  p_pos(A, B),
  \(\lnot\)p_neg(A, B);
  
p_pos(A, B)  \(\leftarrow\)
  p(A, B);
  
p(1, 2)@123;
  
\end{Dedalus}

The single ground fact will, due to \emph{r1}, cause as many deductions as there are tuples in the successor relation.

\end{example}

But if \emph{successor} is infinite, many of mthese are in some sense \emph{void deductions}, functionally determined based on the EDB.
The EDB determines a window over successor that is relevant to any computation that must be performed.  It is easy
to see that in this example, we need only consider a successor relation that contains a single tuple \{123, 124\}.

Consider the given EDB extended with two more facts:

\begin{Dedalus}
delete p(1, 2)@456;
p(?, ?)@789;
\end{Dedalus}

Evaluating this program and EDB will require a \emph{successor} relation with values that range from 123 - 789.

\begin{definition}
A \emph{post-hoc} evaluation is an evaluation of a Dedalus program in which the EDB is given, \emph{successor} is derived from it
as part of a fixpoint computation.
\end{definition}

In a post-hoc evaluation, we may use the given EDB to populate the successor relation in the following way:

Define first a second order predicate called \emph{event\_time} 
that contains the union of the time attributes from the trace of events. Let \emph{Trace} be the set of $n$ EDB predicates.  
Then \emph{event\_time} is defined as

$event\_time(\Tau) \leftarrow \displaystyle\bigcup_{i}^n \pi_{\Tau}Trace_{i}$

\begin{Dedalus}
smax(max<N>) \(\leftarrow\) event\_time(N);
smin(min<N>) \(\leftarrow\) event\_time(N);

successor(N, N + 1) \(\leftarrow\) smin(N);

successor(S, S + 1) \(\leftarrow\) 
    successor(N, S),
    smax(M),
    N <= M;
\end{Dedalus}

In a post-hoc evaluation, time is in some sense ``instantaneous" in that all values of the successor relation are considered in a single
fixpoint computation.  
\paa{notes follow}


define a trace as a list of events ordered by time.  then the property we want to prove is about prefix computations.  take a function FP that returns an IDB.  We'll use 
the notation $\Gamma$ to indicate a trace, and $\alpha_{1} \ldots \alpha_{n}$ to indicate prefixes of  $Gamma$. $\alpha_{n} = \Gamma$, and since every
$\alpha$ is an increasing subset of the EDB, we have 

$\forall \alpha_{i}, \alpha_{j} \in \Gamma ((i < j) \to (\alpha_{i} \subset \alpha_{j}))$.

We would like to show that 

$FP(\alpha_{k}) =  \displaystyle \bigcup_{i=0}^{k} FP(\alpha_{i})$ for any $k$.  

this could (with some work) lead to an inductive proof
that an infinite model is minimal.


%%   FP(alpha_{k} \union FP(alpha_{k-1} \union

if we have this, then we have:

Any posthoc evaluation is equivalent to any 


\subsection{Stratification in {\large{\bf\slang}}}
\label{sec:strat}
We first turn our attention to the semantics of programs with negation.  As we
will see, the inclusion of time enables a syntactic stratification condition
for programs, and the existence of a unique model that corresponds to
intuition~\cite{local-strat}.






\begin{lemma} \label{lemma:no-neg-unique}
A \slang program without negation 
has a unique minimal model.
\end{lemma}

\begin{proof} 
A \slang program without negation 
is a pure Datalog
program.  Every pure Datalog program has a unique minimal model. 
\end{proof}



We define syntactic stratification of a \slang program the same way it is
defined for a Datalog program:

\begin{definition}
A \slang program is \emph{syntactically stratifiable} if there
exists no cycle with a negative edge 
in the program's
predicate dependency graph.
\end{definition}


We may evaluate such a program in {\em stratum order} as described in the
Datalog literature~\cite{ullmanbook}.
It is easy to see that any syntactically stratified \slang instance has a
unique perfect model~\cite{local-strat} because it is a syntactically stratified Datalog program.




However, many programs we are interested in expressing are not syntactically
stratifiable.  Fortunately, we are able to define a syntactically checkable
notion of {\em temporal stratifiability} of \slang programs that maps to a
subset of locally stratifiable Datalog programs.








\begin{definition} 
The \emph{deductive reduction} of a \slang program $P$ is
the subset of $P$ consisting of exactly the deductive rules in $P$.
\end{definition}

\begin{definition} 
A \slang program is \emph{temporally stratifiable} if its deductive
reduction is syntactically stratifiable.
\end{definition}

\begin{lemma}
\label{lemma:temp-strat-uniq}
Any temporally stratifiable \slang instance $P$ has a unique perfect model.
\end{lemma} 

\begin{proof}
Every temporally stratifiable \slang instance is locally
stratifiable~\cite{local-strat}, and thus has a unique perfect model.



\end{proof}


\begin{example}
A simple temporally stratifiable \slang program that is not syntactically stratifiable.

\begin{Dedalus}
persist[p\pos, p\nega, 3]  
  
p_pos(A, B, T) \(\leftarrow\)
  insert\_p(A, B, T);

p_neg(A, B, T) \(\leftarrow\)
  p_pos(A, B, T),
  delete\_p(T);
\end{Dedalus}

In the \slang program above, \emph{insert\_p} and \emph{delete\_p} are captured
in EDB relations.  This reasonable program is unstratifiable because $p\pos \succ
p\nega \land p\nega \succ p\pos$.  But because the successor relation is
constrained such that $\forall A,B, successor(A, B) \rightarrow B > A$, any
such program is locally stratified on time suffixes.  Therefore, we have
$p\pos_{n} \not\succ^+ p\_neg_{n} \not\succ^+ p\pos_{n+1}$; informally, earlier values
do not depend on later values.
\end{example}





\section{Time, State and Order}
\label{sec:stateupdate}

%%\linebreak
\begin{quote}
%
\emph{Time is a device that was invented to keep everything from
happening at once.}\footnote{Graffiti on a wall at Cambridge
University~\cite{scheme}.}
%
\end{quote} 

%Recall that by an event, we mean a \lang fact.
%The transitive consequences
%(via deductive rules) of events are likewise events and hold atomically in the
%same timestep with their premises.  However,

As we showed in the previous section, logical time may be used as a 
``source of monotonicity'' to restore a meaningful, temporal interpretation to otherwise
semantically ambiguous constructs like updateable state and orderly processing.  
Many common motifs in systems programming
(distributed and otherwise) follow a similar pattern of restricted nonmonotonicity, in 
which consequences of deductions are deferred in time.  In this section, we demonstrate
the expressivity of \lang by building a collection of stateful and orderly contructs,
most of which employ the careful use of nonmonotonic reasoning deferred in time.
We also introduce a convenience notation in the form of a simple macro language
for many of these common patterns.

%%\noindent{}In an asynchronous system, the programmer will in general not be able to
%%predict when, or in what order, events arrive from other nodes.  Additionally,
%%some events may need to be handled over time \jmh{vague}, requiring state-oriented motifs
%%such as persistence and mutability.  In this section, we construct a library of
%%\lang constructs to capture these two uses of order.

%Given our definition of \lang, we now address the persistence and mutability
%of data across time: a signature feature of distributed systems---and systems
%in general.
%---for which we provide a model-theoretic foundation.

%The intuition behind \lang's \dedalus{successor} relation is that it models the
%passage of (logical) time.  In our discussion, we will say that facts with
%lower time suffixes occur ``before'' atoms with higher ones.  The constraints
%we imposed on \lang rules restrict how deductions may be made with respect to
%time.  First, rules may only refer to a single time suffix variable in their
%body, and hence {\em cannot join across different ``timesteps''}.  Second,
%rules may specify deductions that occur concurrently with their ground facts,
%\wrm{define ground fact somewhere} in the next timestep, or at some arbitrary
%time, including times before their ground facts.

%This notion of time allows us to consider the contents of the EDB---and hence
%a model of an instance---with respect to an ``instant in time'': we simply
%bind the time suffixes ($\DT$) of all body predicates to a constant.  Because
%this produces a sequence of models (one per timestep), it gives us an intuitive
%and unambiguous way to declaratively express persistence and state changes
%across time.  In this section, we give examples of language constructs
%that capture state-oriented motifs such as persistent relations,
%deletion and update, sequences, and queues.

\subsection{State in Logic}

\jmh{Back this up with formalism: can't do a flip/flop in Datalog (Chandra/Harel, right?).  This discussion does tee up the question of whether something more traditional like Datalog-neg would have been enough for us ... might be sufficient to simply toss in a result about Turing completeness of Dedalus.  I realize your point here is more practically-minded and illustrative, but then maybe this intro is off target for this section.  Still I think an expressivity subsection in the paper would be nice.}


Logic languages naturally model the accumulation of information: deduction in 
the broadest sense tells us, given what we already know, what follows from it.  However, systems
programming frequently requires us to model information that may disappear or change
over time.  In this section we model persistence, both immutable and dynamic, as
\emph{induction} over time, and provide a convenience notation for declaring certain
relations as ``persistent.''


\subsubsection{Simple Persistence}
%
A fact in predicate $p$ at time $\DT$ may provide ground for deductive rules at
time $\DT$, but may only provide ground for deductive rules in timesteps
greater than $\DT$ if it is persisted.  One way to persist all facts in a
predicate $p$ is to use a {\em simple persistence rule}:

\dedalus{p\pos($A_1$,$A_2$,[...],$A_n$)@next $\leftarrow$
p\pos($A_1$,$A_2$,[...],$A_n$);}

\noindent A rule of this form ensures that a $p$ fact true at time $i$ will be
true $\forall j \in \mathbb{N} : j >= i$.


\subsubsection{Mutable State}
\label{sec:mutable}

Simple persistence rules cannot model deletions and updates of a fact, because
they express an unbroken induction over time.  One way to allow the induction
to be broken is to add a \dedalus{p\nega} subgoal to the body of a simple
persistence rule:

\begin{dedalus}
p\_pos($A_1,A_2,[...],A_n$)@next $\leftarrow$
\end{dedalus}

\hspace{5mm}
\begin{dedalus}
p\_pos($A_1,A_2,[...],A_n$),
\end{dedalus}

\hspace{5mm}
$\lnot$
\begin{dedalus}
p\_neg($A_1,A_2,[...],A_n$);
\end{dedalus}

\noindent If, at any time $k$, we have a fact
\dedalus{p\nega($\overline{C}$)@k}, then we do not deduce a
\dedalus{p\pos($\overline{C}$)@k+1} fact.  Furthermore, we do not deduce a
\dedalus{p\pos($\overline{C}$)@j} fact for any $j > k$, unless this
\dedalus{p\pos} fact is re-derived at some timestep $j > k$ by another rule.
This corresponds to the intuition that a persistent fact, once stated, is true
until it is retracted.

%%\newtheorem{example}{Example}
\begin{example}
Consider the following \lang instance:\rcs{introduce p predicate?}

%%p\pos(A, B) \(\leftarrow\) p(A, B);
\begin{Dedalus}
p\pos(A, B)@next \(\leftarrow\) p\pos(A, B), \(\lnot\)p\nega(A, B);

p(1,2)@101;
p(1,3)@102;
p\nega(1,2)@300;
\end{Dedalus}

It is easy to see that the following facts are true: \dedalus{p(1,2)@200},
\dedalus{p(1,3)@200}, \dedalus{p(1,3)@300}.  However, \dedalus{p(1,2)@301} is
false because it was ``deleted'' at timestep \dedalus{300}.
\end{example}

Since mutable persistence occurs frequently in practice, we provide the
\dedalus{persist} template, which takes two arguments: a predicate name and
its arity.  The macro expands to the corresponding mutable persistence rule,
and rewrites the current program in such a way that any references to the given
predicate (say \dedalus{p}) in rule bodies or heads are replaced by references
to its positive relation (e.g., \dedalus{p\_pos}), except for references in the
head of a rule which prefix \dedalus{p} with the distinguished \dedalus{delete}
keyword---these are replaced with \dedalus{p\_neg}.  The above
\dedalus{p\_pos} persistence rule may be equivalently specified as
\dedalus{persist[p,  2]}.

Mutable persistence rules enable {\em updates}.  For some time $\DT$, an update
is any pair of facts:

\begin{dedalus}
p\nega($\overline{C})@\DT;$
\end{dedalus}

\begin{dedalus}
p\pos($\overline{D})@\DT+1$;
\end{dedalus}


\noindent Intuitively, an update represents deleting an old value of a
tuple and inserting a new value.  Every update is {\em atomic across
  timesteps}, meaning that the old value exists during timestep $\DT$
when the new value is derived.  During the evaluation of timestep
$\DT+1$ the new value exists, and the old does not.

\subsubsection{Assignment and Committed Choice}

The assignment primitive provided by most imperative languages is a special case
of update without deletion.  We can model the (destructive) assignment of sets
of values to keys in the following way:

\begin{Dedalus}
log(A, B)@next \(\leftarrow\) condition(A, B);
log(A, B)@next \(\leftarrow\) log(A, B), \(\lnot\)condition(A, _);
\end{Dedalus}

The pair of rules above will cause {\em log} to associate with $A$ the ``most
recent'' set of $B$ values appearing in {\em condition}.  If {\em condition(A,
B)} respects the functional dependency $A \to B$, then \dedalus{log} will
associate only the ``most recent'' $B$ value with each $A$.

The mirror image of assignment is committed choice~\cite{committedchoice},
which associates the first
value(s) of $B$ with $A$.  Committed choice ``seals'' the value of \dedalus{B} such that ``future''
insertions into \dedalus{condition} cannot cause new rows with the same
\dedalus{A} value to be inserted.

\begin{Dedalus}
log(A, B)@next \(\leftarrow\) log(A, B);
log(A, B)@next \(\leftarrow\) condition(A, B), \(\lnot\)log(A, _);
\end{Dedalus}
%%\subsection{``At Most Once'' event relations}

Assignment and committed choice implement ``last write wins'' and 
``first write wins'' semantics, respectively.

\subsubsection{``At Most Once``}
A common requirement for programs with side-effects outside the control
of the system is ensuring that certain events occur ``at most once.''
Consider a requirement for our shopping cart application 
that only a single checkout response should be generated, even if subsequent
inputs cause the totals to be recalculated.  Hence we want to ensure that the
predicate {\em response} ``fires'' only once, regardless of the number of times 
that {\em status} fires.  This feature can be expressed as a 
specialization of the committed choice pattern.   

\begin{Dedalus}
response(Cli, Ses, Item, Amt) \(\leftarrow\)
  amo\_event(Cli, Ses, Item, Amt);

amo\_event(Cli, Ses, Item, Amt) \(\leftarrow\)
  status(Cli, Ses, Item, Amt), 
  \(\lnot\) amo\_log(Cli, Ses, Item, _);

amo\_log(Cli, Ses, Item, Amt)@next \(\leftarrow\) 
  amo\_event(Cli, Ses, Item, Amt);

amo\_log(Cli, Ses, Item, Amt)@next \(\leftarrow\) 
  amo\_log(Cli, Ses, Item, Amt);
\end{Dedalus}

In the subprogram above, the (immutable) predicate {\em amo\_log} serves as a guard
for the predicate {\em amo\_event}, which is true only for the ``first'' assignment
of an $Amt$ value to a grouping of client, session and item identifiers.  Subsequent
occurrences of the {\em status} event for the same group will never fire the second rule.

%But perhaps surprisingly,
%``closing a world'' in this fashion ensures that \dedalus{log} has strictly
%monotonic behavior in all models
%\wrm{commenting out fancy pants diction, so neil doesn't have to}

\paa{introduce 'at most once' as a specialization of committed choice, and
present the macro, which expands to a table (say foo), a log (foo\_log) and an
event (foo\_event) which occurs once if foo occurs at all.}

\subsection{Order in Logic}

\paa{some intro text: pure logic has no notion of order.  distributed systems programming
frequently requires ordering constructs to cope with indeterminacy in message ordering
and to achieve synchronization}

\noindent{}In an asynchronous system, the programmer will in general not be able to
predict when, or in what order, events arrive from other nodes.  When the timing and 
ordering of message arrival affects program results, it may be necessary to instrument
programs with constructs that preserve or restore order at communication boundaries,
or to ensure that simultaneous arrival of messages has the same effect as serial arrival.

\subsubsection{Priority Queues}

\paa{whack this ponderous subsection.  present instead \emph{serializers},
which do what so-called queues below do: enforce 'associativity' by preventing
more that one tuple from being considered per fixpoint.  then present
\emph{ordered queues} (or perhaps the same thing with a better name)
after the discussion of sequences (since OQs use sequences), as a mechanism
for maintaining ordering across async boundaries (familiar from TCP, fifo broadcast,
etc)}

%\paa{shorten this section} While a sequence is a useful construct for
%generating or imposing an ordering on tuples \wrm{seems a bit fishy.  seems
%like it might only be useful for transferring a given order thru async,
%assuming the tuples are already ordered at sender. plus, seqs appears after
%this now}, 

\wrm{i think this might be the wrong way to present queues.  queues don't
necessarily guarantee that ``all things of priority X happen before all things
of priority Y>X''.  the two high order bits of queues are: they prevent things
of different priorities from simultaneously executing, and they enforce kind of
a loose order ``dequeue the lowest priority thing i have thus far''.  ordering
by itself isn't a high-order bit though, because i can sort in one stratum.
it's more this ``online loose order'' which is important.  didn't want to do
too much damage to this section, so i didn't rewrite it yet to conform with
this.}

Some programs will require tuples to be processed in a particular (partial)
data-dependent order, rather than all-at-once, as a set.  For example,
consider a predicate \dedalus{priority\_queue} that represents a series of
tasks.
%to be performed in some predefined order.
Its attributes are two strings---a user and a job---and an integer indicating
the priority of the job in the queue:

\begin{Dedalus}
priority\_queue('bob', 'bash', 200)@123;
priority\_queue('eve', 'ls', 1)@123;
priority\_queue('alice', 'ssh', 204)@123;
priority\_queue('bob', 'ssh', 205)@123;
\end{Dedalus}

A program may desire to serialize the jobs, despite the coincidence of the
\dedalus{priority\_queue} events in logical time.

%Depending on the program that implements the balance update, several behaviors
%are possible.
%Given this schema, we note that a program would likely want to process
%\dedalus{priority\_queue} events individually in a data-dependent order, in
%spite of their coincidence in logical time.  

%%It is difficult to express general
%%in-order tuple processing in Datalog, in part because the language does not
%%admit sequences.  \jmh{Huh?  I don't see the last clause there.  Maybe say simply that Datalog is set-oriented, but what we want here is precisely to impose an ordering on the elements of the set, which seems unnatural.  There's maybe a connection to expressibility and aggregation or arithmetic or something, but let's not try to sort that out for now.}
%above is really what we want to say, right? -wrm
%has so
%notion of order of evaluation (except the implicit ordering implied by
%stratification).

In the program below, \dedalus{priority\_queue} stores the current contents of
the queue at any given time.  The queue must persist across timesteps, as
multiple timesteps may be necessary to drain the queue.  At each timestep, for
each value of \dedalus{A}, all tuples of minimum priority are stored in
\dedalus{priority\_queue\_out} and deleted (atomic with the storage).  Note
that this will change the value of the aggregate calculated at the subsequent
timestamp, assuming no new tuples are inserted at the next timestamp with a
just-dequeued priority:

\begin{Dedalus}
persist[priority\_queue, 3]

// find the min priorities
omin(A, min<C>) \(\leftarrow\)
  priority\_queue(A, _, C);

// output min priority elements
priority_queue_out(A, B, C)@next \(\leftarrow\)
  priority\_queue(A, B, C), omin(A, C);

// delete min priority elements
delete priority\_queue(A, B, C) \(\leftarrow\)
  priority\_queue(A, B, C), omin(A, C);
\end{Dedalus}

In this example, deductive rules that depend on \dedalus{priority\_queue\_out}
are constrained to consider only min-priority tuples at each timestep per value
of the variable \dedalus{A}, thus implementing a per-user FIFO discipline.  To
enforce a FIFO ordering over all users, we may remove the \dedalus{A} column
from \dedalus{omin}.

%A queue establishes a functional dependency between a \lang timestamp and a
%given priority.

By doing so, we take advantage of the monotonic property of timestamps to enforce an ordering property over our input that is otherwise 
very difficult to express in a logic language.
%We return to this idea in our discussion of temporal ``entanglement'' Section~\ref{sec:entangle}.

%Priority queues were developed in a similar fashion in~\cite{greedybychoice}.

\subsubsection{Entanglement}
\label{sec:entangle}

It is sometimes necessary to {\em entangle} the \dedalus{successor} relation
with attributes other than the time suffix, for example to express unbounded
sequences, or to establish a global order (such as through Lamport Clocks).
Consider the asynchronous rule below:

\begin{Dedalus}
p(A, B, N)@async \(\leftarrow\)
  q(A, B)@N;
\end{Dedalus}
\noindent

Due to the \dedalus{async} keyword in the rule head, each \dedalus{p} tuple
will take some unspecified time suffix value.  Note however that the time
suffix \dedalus{N} of the rule body appears also in an attribute of \dedalus{p}
other than the time suffix, recording a binding of both the time value of the
deduction and the time value of its consequence.  We call such a binding an
{\em entanglement}.   Note that in order to write the rule it is necessary to
not sugar away the time suffix in the rule body.  

\rcs{the above discussion obscures a crucial detail: entanglement doesn't allow arbitrary access to the timestamp.  Instead, it provides a one way information flow ``out of'' the timestamp field}

\subsubsection{Sequences}
%\wrm{Maybe somehow work in the fact that sequences are really about preserving
%an already-established order (at a sender) through asynchrony at the receiver.
%Connect to entanglement}

One may represent a sequence---an object that maintains a monotonically
increasing counter value---with a pair of inductive rules.  One rule
increments the current counter value when some condition is true, while the
other persists the value of the sequence when the condition is false.  We can
capture the increase of the sequence value without using arithmetic by
entangling \dedalus{successor}:

\begin{Dedalus}
seq(B)@next \(\leftarrow\) seq(A), successor(A,B), event(_);  
seq(A)@next \(\leftarrow\) seq(A), \(\lnot\)event(_);
\end{Dedalus}

\noindent Note that these two rules produce only a single value of
\dedalus{seq} at each timestep---assuming that the sequence was originally
instantiated with a single value---but they do so in a manner slightly
different than our standard persistence template.

Sequences are useful in general for preserving an established ordering on a set
when communicating between nodes.  As a shorthand we provide the {\em sequence}
macro, which takes three arguments (sequence name, a ``trigger'' predicate
which, when true, should cause the sequence to be incremented, and the
trigger's arity) and expands them to a pair of definitions of a unary predicate
like the one defined above (e.g., \dedalus{sequence[seq, event, 1]}).

\subsubsection{Lamport Clocks}
\label{sec:lamport}
%\wrm{Clean this up and make it jibe better with sec 5}
%Recall that asynchrony allows program executions to order message timestamps
%arbitrarily, violating intuitive notions of causality by allowing deductions to
%``affect the past.'' This section explains how to implement Lamport
%clocks~\cite{timeclocks} atop \lang, which allows programs to ensure temporal
%monotonicity by reestablishing a causal order despite derivations that flow
%backwards through time.  \wrm{we haven't yet introduced stratification and all
%that.  maybe there's some better way to write the above, like ``it is often
%nice to have a global partial order for X reasons''.  then later, we can say
%``aha! a majorly important utility of a global partial order is to avoid
%contradiction.}
It is often necessary to ensure some loose synchronization between clocks of
different nodes in an asynchronous distributed system.  One way to do this is
through Lamport clocks~\cite{timeclocks}.

Consider a rule \dedalus{p(A,B)@async \(\leftarrow\) q(A,B)}.  By rewriting it
to:

\begin{Dedalus}
persist[p, 2]
p\_wait(A, B, N)@async \(\leftarrow\) q(A, B)@N;
p\_wait(A, B, N)@next \(\leftarrow\) p\_wait(A, B, N)@M, N \(\ge\) M;
p(A, B)@next \(\leftarrow\) p\_wait(A, B, N)@M, N < M;
\end{Dedalus}

\noindent we place the derived tuple in a new relation \dedalus{p\_wait} that
stores any tuples that were ``sent from the future,'' according to their
entangled time; these tuples stay in the \dedalus{p\_wait} predicate until the
point in time at which they were derived.  
%Conceptually, this causes the system to evaluate a potentially large number of
%timesteps (if N is significantly less than the timestamp of the system when
%the tuple arrives).  However, if the runtime is able to efficiently evaluate
%timesteps when the database is quiescent, then instead of ``waiting'' by
%evaluating timesteps, it will simply increase its logical clock to match that
%of the sender.  \wrm{don't think we need to be getting into the efficiency of
%evaluation of the language this early...} In contrast, if the tuple is ``sent
%into the future,'' then it is processed using the timestep that receives it.
%\wrm{yes!  we delete the above thing about efficiency, and just keep the
%below}
This manipulation of timesteps and clock values is equivalent to conventional
descriptions of Lamport clocks, except that our Lamport clock implementation
effectively ``advances the clock'' by preventing derivations until the clock is
sufficiently advanced, by temporarily storing incoming tuples in the
\dedalus{p\_wait} relation.\footnote{For ease of exposition, we elide one
detail here: Lamport clocks rely upon a ``tie-breaking'' function to ensure
that no two events have the same timestamp.  We can implement such a discipline
using queues.}

Although annotating a program execution with logical clock values has
a number of practical runtime applications (such as debugging), in our
setting it is primarily useful as a way to reconcile physical
constraints (a given program execution on real hardware will obey
causality) with our expressive language model (which is able to model
temporal paradoxes).  Crucially, we do so in a purely logical manner,
without resorting to imperative constructs outside of Datalog.  In
Section~\rcs{sec:fixme}, we take this idea a step further, and explain
how \lang programs can be restricted to treat events as serializable
transactions.  This allows us to model well-studied runtime
optimizations such as parallelizing compilers and database lock
managers with little additional complexity.

\paa{clean up, and cite netdb and the TR as examples of more complicated
synchronization constructs expressible in logic (consensus and reliable broadcast,
respectively)}

%\wrm{why can't
%we just combo a Lamport clock with a priority queue?  i don't think we need
%choice here, so i commented it out.}
%In \lang, such a function could be implemented via another use of
%\dedalus{choice}, or by a program convention like appending a unique node
%identifier to each timestamp to prevent ``ties.''

\section{Related Work}

\subsection{Concurrency control}
The importance of commutative operations has long been recognized by work in the
distributed systems, data management, and groupware communities.

\subsection{Non-monotonicity in deductive databases}
Adding non-monotonic operators (e.g., aggregation and negation) to Datalog
increases the expressiveness of the language but introduces significant
complexities: care must be taken to ensure that the resulting language has a
semantics that is well-defined, intuitive to the user, and amenable to efficient
evaluation. A straightforward approach is to disallow recursion through
aggregation or negation, which admits only the class of so-called ``stratified
programs''~\cite{Apt1988}. Many attempts have been made to assign a semantics to
larger classes of programs (e.g.,~\cite{Gelfond1988,Ross1990,VanGelder1991}).

The observation that many uses of aggregation and negation have a ``monotonic''
flavor has been made before.

% CRDTs
% Work on semantics-aware concurrency control
% Operational transformations
% Ross & Sagiv on lattices
% Kostler et al. on differential fixpoint + subsumption

\bibliographystyle{abbrv}
\bibliography{pods}

\end{document}
