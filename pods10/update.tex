\section{Implications for State Update}
%%\newdef{definition}{Definition}



\subsection{Persistent Storage Implementations}

\paa{help!}

%%\subsubsection{We need only store the event tuple with max(Timestamp) for each projection of the other columns to query the present time}
\subsubsection{Cost Model}

\begin{figure}[t]
\begin{tabular}{ll} \hline
%%Rule Pattern & Idiom & Prepare & Propose & Election \\ \hline \hline
$d$ & Cost of a deductive step \\
$s$ & Cost of storing a tuple \\
$r$ & Cost of reading a tuple \\ 
$t$ & Number of tuple derivations from deductive rules \\ 
\hline
$S$ & Set of tuples inserted \\
$U$ & Set of tuples updated \\
$P$ & Set of stored tuples, with time projected out \\ 
$T$ & Set of stored tuple timestamps \\ 
$Q$ & Set of query timestamps \\ \hline 
\end{tabular}
\caption{Cost model.}
\label{fig:breakdown}
\end{figure}


\subsubsection{Naive Deductive Implementation}

To evaluate a schedule consisting of inserts, updates and queries, the naive deductive implementation must,
at a minimum,
\begin{enumerate}
\item Write every event once.  We may think of this as sequential log appends.  An update entails two writes: one for the deletion
and one for the insertion of the new tuple.
\item Read the entire log at least once.
\item For each query, perform the necessary number of inductive steps to determine if relevant predicates hold at query time.
On the surface, this would appear to be the interval between the ground fact time and the query time, if that interval is unbroken
by a deletion predicate.
\end{enumerate}

$(|S|+2|U|)s + (|S|+2|U|)r + t + (\displaystyle\sum_{i=0}^{|Q|-1} \displaystyle\sum_{j=0}^{|T|-1} Q_{i} - T_{j})d$

\subsubsection{Naive Overwriteable Storage Implementation}

In this case, every update entails a read and then a write.

$|S|s + |U|s + |U|r + t$


%%\subsubsection{perhaps we can admit queries over the past that are bounded and pre-stated, and do GC}
